\documentclass{amsart}[12pt]
%\pdfoutput=1

\usepackage{tikz}
\usepackage[sc]{mathpazo}
\usepackage{hyperref}
\usepackage{microtype}
\usepackage{tikz}
\usetikzlibrary{snakes, arrows}
\usepackage{graphicx}
\usepackage{mathtools} %for \mathclap, fixing long subscripts in a summation
\linespread{1.2}

\usepackage{xcolor}
\hypersetup{
    colorlinks,
    linkcolor={red!80!black},
    citecolor={blue!80!black},
    urlcolor={blue!80!black}
}


\theoremstyle{definition}

\newtheorem*{theorem*}{Theorem}
\newtheorem{dummy}{}[section]
\newtheorem{theorem}[dummy]{Theorem}
\newtheorem{lemma}[dummy]{Lemma}
\newtheorem{example}[dummy]{Example}
\newtheorem{corollary}[dummy]{Corollary}
\newtheorem{definition}[dummy]{Definition}
\newtheorem{remark}[dummy]{Remark}
\newtheorem{proposition}[dummy]{Proposition}
\newtheorem{observation}{Observation}
\newtheorem{question}{Question}
\newtheorem{conjecture}[dummy]{Conjecture}


\newcommand{\Z}{\mathbb{Z}}
\newcommand{\C}{\mathbb{C}}
\newcommand{\N}{\mathbb{N}}
\newcommand{\R}{\mathbb{R}}
\newcommand{\proj}{\mathbb{P}}
\newcommand{\core}{\mathbf{core}}

\DeclareMathOperator{\Hilb}{Hilb}
\DeclareMathOperator{\Ext}{Ext}
\DeclareMathOperator{\Hom}{Hom}

\begin{document}
\title{Orbifold Hilbert schemes and a generalization of cores and quotients}


\author{Paul Johnson}
\address{University of Sheffield}
\email{paul.johnson@sheffield.ac.uk}


\maketitle

\section{Introduction}
\cite{GLM}

\section{Hilbert schemes of points on surfaces}



\subsection{Colored square counting}

The core construction helps us with the colored box square counting:

Consider the function
$$P_G(q_0,\dots,q_{r-1})=\sum_{\lambda\in\mathcal{P}} \mathbf{q}^{|\lambda|^G}$$
where 
$$\mathbf{q}^{|\lambda|_G}=\prod_{i=0}^{r-1} q_i^{|\lambda|^G_i}$$
that counts partitions according to their full colored square count as opposed to just their size; alternatively, the coefficient of $\mathbf{q}^v$ is the euler characteristic of $\Hilb_v([\C^2/G])$.

What can we say about $P_G$?  First, we consider the case where $G\subset SL_2$.  Let $Q=q_0q_1\cdots q_{r-1}$.  The core construction gives
$$P_G=\prod_{i=1}^\infty\frac{1}{(1-Q^i)^r} \sum_{w} Q^{A(w)}\mathbf{q}^w $$
Thus, we see in this case $P_G$ is a multivariable theta function.

In case $r=2$, this has an infinite product expansion using the Jacobi identity.  In fact, work of Boulet shows that actually the case $G=\Z_2\times \Z_2$ has an infinite product expansion.  Letting $q_{00}, q_{01}, q_{10}, q_{11}$ denote the variables, we have

$$P_{\Z_2\times\Z_2}=\prod_{i=1}^\infty \frac{(1+q_{00}^jq_{01}^{j-1}q_{10}^{j-1}q_{11}^{j-1})(1+q_{00}^jq_{01}^jq_{10}^jq_{11}^{j-1})}{(1-q_{00}^jq_{01}^jq_{10}^jq_{11}^j)(1-q_{00}^jq_{01}^jq_{10}^{j-1}q_{11}^{j-1})(1-q_{00}^jq_{01}^{j-1}q_{10}^{j}q_{11}^{j-1})}$$

However, even the one variable specialization $P_{\Z_3}(q,1,1)$ does not have a nice infinite product expression, as observed by Bal\'azs Szendr\"oi \cite{Bmo}; as it has a root at $-e^{\pi/\sqrt{3}}$ \cite{Borwein2}.


\subsection{Hilbert schemes of points in the plane}


\begin{theorem}[Ellingsrud and Str\o mme, 1987]
$$\sum_{k,n \geq 0} b_k(\Hilb_n(\C^2))t^k q^n=\prod_{\ell=1}^\infty \frac{1}{1-t^{2\ell-2}q^\ell}$$
\end{theorem}



\subsubsection{Bia\l ynicki-Birula decomposition}
Proved using the Bia\l ynicki-Birula decomposition.  The Bia\l ynicki-Birula decomosition should be understood in analogy with Morse theory.  

The role of the Morse flow will be played by the flow $x\mapsto \varepsilon x$ as $\varepsilon\in\C^*$ tends toward zero.  We assume that the $\C^*$ action on $X$ is such that this limit point exists for all $x\in X$.  

Let $p$ be a fixed point of the $\C^*$ action.  Then linearizing the $\C^*$ action on $X$ gives a $\C^*$ action on $T_pX$, and so $T_pX$ is not just a vector space but a $\C^*$ representation, and hence decomposes into a direct sum of irreducible representations.  Let $V_a$ denote the irreducible representation of $\C^*$ where $\varepsilon\in\C^*$ acts as $\varepsilon^a$.  

Let $T_p^+X$ (respectively $T_p^-X$) denote the subspace of $T_pX$ where the $\C^*$ action acts with a positive (respectively negative) exponent.  I.e., if

$$T_pX=\bigoplus_{n\in\Z} V_n^{e_n}$$
then

Let 
$$\mathcal{S}_p=\{x\in X|\lim_{\varepsilon\to 0} \varepsilon x=p\}$$
Then clearly we have $X=\sqcup_{p}\mathcal{S}_p$; the point is that $\mathcal{S}_p$ is a subvariety isomorphic to $\C^{\iota(p)}$.



We first observe that $V_0$ cannot occur in $T_pX$, as 

We saw in Section [REFER BACK TO ] that if $X$ is a variety with a $\C^*$ action with $k$ isolated fixed points, then $\chi(X)=k$.  If we know the weights of the $\C^*$ action on the tangent spaces of the fixed points, the Bia\l ynicki-Birula decomposition leverages this result to give the betti numbers of $X$, or even further the class of $X$ in the Grothendieck ring of varieties.

\subsubsection{Tangent space to a monomial ideal}
To apply the Bia\l ynicki-Birula decomposition to $\Hilb_n(\C^2)$, we need to calculate the weights of the $\C^*$ action on $T_\lambda\Hilb_n(\C^2)$.

\begin{lemma}[Ellingsrud and Str\o mme, Cheah]
$$T_\lambda \Hilb_n(\C^2)=\sum_{\square\in\lambda} \left(x^{-\ell(\square)} y^{a(\square)+1}+x^{\ell(\square)+1}y^{-a(\square)}\right)$$
\end{lemma}

\begin{proof}

First, we use $T_\lambda\Hilb_n(\C^2)=\Hom_R(\mathcal{I}_\lambda,R/\mathcal{I}_\lambda)$.

Analogy with the Grassmannian makes this intuitively plausible; indeed, if $V\subset W$ is a $k$ dimensional subspace, then $T_V Gr_k(W)=\Hom(V, V^\perp)=\Hom(V, W/V)$.  Thus, we understand the deformations of $\mathcal{I}_\lambda$ as a vector space; for the deformation to remain an ideal it is plausible that we should require the deformation to be a map of $R$ modules and not just vector spaces.

More formally, it is a general fact that first order deformations of objects are given by $\Ext^1(\mathcal{F},\mathcal{F})$, and obstructions to these deformations are given by $\Ext^2(\mathcal{F},\mathcal{F})$; starting from this fact and considering the long exact sequences by taking $\Hom(\mathcal{I})$ to

$$0\to \mathcal{I}\to R\to R/\mathcal{I}\to 0$$
gives the result.




\end{proof}

\bibliographystyle{plain}
\bibliography{GHilb}

\end{document}
