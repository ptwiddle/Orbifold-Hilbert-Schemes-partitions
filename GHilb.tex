\documentclass{amsart}[12pt]
%\pdfoutput=1

\usepackage{tikz}
\usetikzlibrary{arrows, snakes} %for arrowtips  
\usepackage[sc]{mathpazo}
\usepackage{hyperref}
\usepackage{microtype}
\usepackage{tikz}
\usepackage{graphicx}
%\usepackage{mathtools} %for \mathclap, fixing long subscripts in a summation
\linespread{1.2}
\usepackage{xcolor}
\hypersetup{
    colorlinks,
    linkcolor={red!80!black},
    citecolor={blue!80!black},
    urlcolor={blue!80!black}
}


\theoremstyle{definition}

\newtheorem*{theorem*}{Theorem}
\newtheorem{dummy}{}[section]
\newtheorem{theorem}[dummy]{Theorem}
\newtheorem{lemma}[dummy]{Lemma}
\newtheorem{example}[dummy]{Example}
\newtheorem{corollary}[dummy]{Corollary}
\newtheorem{definition}[dummy]{Definition}
\newtheorem{remark}[dummy]{Remark}
\newtheorem{proposition}[dummy]{Proposition}
\newtheorem{observation}{Observation}
\newtheorem{question}{Question}
\newtheorem{conjecture}[dummy]{Conjecture}
\newtheorem{warning}[dummy]{Warning}

\newcommand{\Z}{\mathbf{Z}}
\newcommand{\C}{\mathbf{C}}
\newcommand{\N}{\mathbf{N}}
\newcommand{\R}{\mathbf{R}}
\newcommand{\A}{\mathbf{A}}
\newcommand{\OO}{\mathcal{O}}
\newcommand{\LL}{\mathbf{L}}
\newcommand{\PP}{\mathcal{P}} % partitions


\newcommand{\proj}{\mathbf{P}}
\newcommand{\core}{\mathbf{core}}
\newcommand{\quot}{\mathbf{quot}}
\newcommand{\irreps}{\text{irreps}}
\newcommand{\Var}{\mathbf{Var}}
\newcommand{\Top}{\mathbf{Top}}




\DeclareMathOperator{\Hilb}{Hilb}
\DeclareMathOperator{\Ext}{Ext}
\DeclareMathOperator{\Hom}{Hom}
\DeclareMathOperator{\DC}{DH}
\DeclareMathOperator{\Sym}{Sym}
\DeclareMathOperator{\Ker}{Ker}
\DeclareMathOperator{\cdim}{cdim}


\newcommand{\HG}{\Hilb}

\begin{document}
\begin{abstract}
We study the connection between the combinatoric of certain partition statistics and the topology of the Hilbert schemes of certain orbifold surfaces.
\end{abstract}


\title{Orbifold Hilbert schemes and a generalization of cores and quotients}


\author{Paul Johnson}
\address{University of Sheffield}
\email{paul.johnson@sheffield.ac.uk}


\maketitle
\setcounter{tocdepth}{2}
\tableofcontents
\section{Introduction}
\cite{GLMpower}

This paper is written for two largely distinct audiences.  On the one hand, we write for algebraic geometers interested in Hilbert schemes of points and orbifold surfaces.  On the other hand, combinatorialists studying partitions.  As such, there is more expository material than perhaps is standard. 

For geometers: there is much structure in the topology of Hilbert schemes of point on a smooth surface $S$: Gottsche, building on the work of Ellingsrud and Str\o me for $\C^2$, found product formulas for generating functions of their cohomology.  Later, Grojnowski and Nakajima \cite{grojnowski, nakajimaheisenberg} explained these product formulas for geometric representation theory.  An important example here is when $S$ is the resolution of an ADE quotient signularity -- 

Following and extending work of , we investigate the topology of the structure of Hilbert schemes of points on the local models $\C^2/\Z_r$.  We conjecture explicit product formulas in this case, reminiscent of Gottsche's formula, in terms of the Chen-Ruan cohomology of $[\C^2/\Z_r]$.  

Turning to the connected components, we prove a stability result for the cohomology of orbifold Hilbert schemes, reminiscent of G\"ottsche's, as more and more \emph{smooth} points (equivalently, copies of the regular representation) are added.  We investigate components stuck of the origin, and show their dimension and the dimension of the corresponding ideals are given by piecewise quadratic functions, where the quadratic part is the intersection pairing on the minimal resolution.  We conjecture the action of a Heisenberg group modeled on the minimal resolution.

For combinatorialists: Cores and quotients of partitions initially arose in the study of the modular representation theory of the symmetric group, but have found roles in number theory and representation theory, among others.  A lesser known occurence is in the geometry of Hilbert schemes of points.  

Using this geometry as motivation, we introduce here some new partitions statistics, and use a generalization of cores and quotients of partitions to study them.  Our first statistic turns out out to be the size of our generalized quotient, and it counts cells where certain linear combinations of the arm and leg lengths satisfy a congruence condition.  We conjecture that $(q,t)$ counting partitions with respsect to size and this new statistic satisfy explicit product formulas that are $t$-analogs of the Euler product (or more generally, $t$-analogs of the formula for partitions coming from cores and quotients).  

We prove that ``$G$-core'' partitions are in bijection with lattice points, and their size is given by a piecewise quadratic function.  




Buryak conjectured \cite{Buryak}, and later proved with Feigin \cite{BF}, and simplified the proof with Nakajima \cite{BFN}, the following version. 

\begin{theorem}[Buryak-Feigin-Nakajima] \label{thm:BFN}
Let $\C^*$ act on $\C^2$ by $t\cdot (x,y)=(t^ax, t^by), a,b\geq 0$.  Then

$$\sum_{k,n} h^k(\Hilb_n(\C^2)^{\C^*})q^nt^k=\prod_{a+b } \frac{1}{1-q^\ell}\prod_{(a+b)|\ell} \frac{1}{1-q^\ell t}$$
\end{theorem}

The proofs given by Buryak-Feigin, and Buryak-Feigin-Nakajima are algebraic geometric, and go through quiver varieties.  An obvious first step in trying to prove our conjectures combinatorially would be to prove Theorem \ref{thm:BFN} combinatorially.  We give some evidence in Section \ref{sec:dyson} that maps generalizing Dyson's map might be the approach for this.



\begin{tikzpicture}[scale=.4]
\begin{scope}
\draw (3.5,8.5) node{(1/5,1/5)};
       \clip (0,0) rectangle (7,7);
\foreach \x in {0,...,6}{%
    \foreach \y in {-2,...,2}{%
       \draw[fill=red!20]  (\x, -\x+5*\y) rectangle (\x+1, -\x+5*\y+1); 
       \draw[fill=green!20]  (\x, -\x+5*\y+1) rectangle (\x+1, -\x+5*\y+2); 
       \draw[fill=blue!20]  (\x, -\x+5*\y+2) rectangle (\x+1, -\x+5*\y+3); 
       \draw[fill=black]  (\x, -\x+5*\y+3) rectangle (\x+1, -\x+5*\y+4);
        \draw  (\x, -\x+5*\y+4) rectangle (\x+1, -\x+5*\y+5); 
}}
\end{scope}     

\begin{scope}[xshift=10cm]
\draw (3.5,8.5) node{(1/5,2/5)};
\begin{scope}[xscale=-1, rotate=90]
       \clip (0,0) rectangle (7,7);
\foreach \x in {0,...,6}{%
    \foreach \y in {-4,...,2}{%
       \draw[fill=red!20]  (\x, 3*\x+5*\y) rectangle (\x+1, 3*\x+5*\y+1); 
       \draw[fill=green!20]  (\x, 3*\x+5*\y+1) rectangle (\x+1, 3*\x+5*\y+2); 
       \draw[fill=blue!20]  (\x, 3*\x+5*\y+2) rectangle (\x+1, 3*\x+5*\y+3); 
       \draw[fill=black]  (\x, 3*\x+5*\y+3) rectangle (\x+1, 3*\x+5*\y+4);
        \draw  (\x, 3*\x+5*\y+4) rectangle (\x+1, 3*\x+5*\y+5); 
}}
\end{scope}
\end{scope}     

\begin{scope}[xshift=20cm]
\draw (3.5,8.5) node{(1/5,-1/5)};
\begin{scope}[xscale=-1, rotate=90]
       \clip (0,0) rectangle (7,7);
\foreach \x in {0,...,6}{%
    \foreach \y in {-2,...,2}{%
       \draw[fill=red!20]  (\x, \x+5*\y) rectangle (\x+1, \x+5*\y+1); 
       \draw[fill=green!20]  (\x, \x+5*\y+1) rectangle (\x+1, \x+5*\y+2); 
       \draw[fill=blue!20]  (\x, \x+5*\y+2) rectangle (\x+1, \x+5*\y+3); 
       \draw[fill=black]  (\x, \x+5*\y+3) rectangle (\x+1, \x+5*\y+4);
        \draw  (\x, \x+5*\y+4) rectangle (\x+1, \x+5*\y+5); 
}}
\end{scope}     
\end{scope}

\end{tikzpicture}

By adding a $1\times r$ strip to the bottom right corner of a partition, it is immediate that if a class $v\in K(G)$ is represented by a partition, then $v+V_G$ is as well. 

\begin{definition}
The $\overline{K}(G)=K(G)/V_G$ 
\end{definition}

It is not hard to see that any class in $\overline{K}(G)$ is represented by a partition: a block staircase partition with blocks of size $|G|$ will be a multiple of $V_G$, and then modifications on each step can add a multiple of $G$ plus one of any given color.  Below, the bottom to stairs each have the effect of adding one white box in $\overline{K}(G)$, while the top two stairs each add one red box.

\begin{tikzpicture}[scale=.3]
\begin{scope}[xscale=-1, rotate=90]
       \clip (0,0) rectangle (12,12);
\foreach \x in {0,...,11}{%
    \foreach \y in {-4,...,4}{%
       \filldraw[red!20]  (\x, \x+3*\y) rectangle (\x+1, \x+3*\y+1); 
       \filldraw[green!20]  (\x, \x+3*\y+1) rectangle (\x+1, \x+3*\y+2); 
       \draw[gray] (0,0) grid (12,12);
}}
\end{scope}     
\draw[very thick] (0,9)--(3,9)--(3,6)--(6,6)--(6,3)--(9,3)--(9,0);
\draw[very thick, dashed] (0,10)--(1,10)--(1,9);
\draw[very thick, dashed] (3,7)--(4,7)--(4,6);
\draw [very thick, dashed] (6,6)--(8,6)--(8,4)--(9,4)--(9,3);
\draw [very thick, dashed] (9,3)--(11,3)--(11,1)--(12,1)--(12,0);

\end{tikzpicture}


\begin{question}
Given a class $\overline{v}\in \overline{K}(G)$, what is the smallest class $v\in K(G)$ represented by a partition?
\end{question}


\subsection{Acknowledgements}

Research on this work was supported by NSF grant NUMBER HERE, and CITE TOM's GRANT?

Thanks to Amin Gholampour, Yunfeng Jiang, and Martijn Kool for asking a question that began this work, and to Tom Bridgeland for useful conversations.



\section{Background: Hilbert schemes of points on smooth surfaces}

\subsection{Partitions}


A partition is of $n$ is a nondincreasing sequence of numbers $\lambda_1\geq \lambda_2\geq \cdots \geq \lambda_\ell > 0$ with $\sum \lambda_i=n$  The \emph{length} $\ell(\lambda)$ of a partition is the number of parts, the \emph{size} $|\lambda|$ is the sum of the parts.  We use $\PP$ to denote the set of all partitions, and $\PP_n$ to denote the partitions of $n$.

Rather than a list of numbers, we usually view partitions in terms of their \emph{Young diagrams}.  We draw our Young diagrams as subsets of a grid in the first quadrant, with the square in the corner being $(0,0)$, and with the columns being the parts of $\lambda$.  

We refer to a cells of a partition as $\square\in\lambda$.  The \emph{arm} of a square $\square\in\lambda$ is the number of cells above $\square$ and in $\lambda$; the \emph{leg} $\ell(\square)$ of a cell is the number of cells contained in $\lambda$ and to the right of $\square$.

\begin{example}
Below is the Young diagram of $\lambda=3+2+2+1$.  The cell $(2,1)$ is marked $s$; the cells in the arm and leg of $s$ are labeled $a$ and $l$, respectively.
\begin{center}
\begin{tikzpicture}[scale=.5]
\draw[thin, gray] (0,0) grid (1,3);
\draw[thin, gray] (1,0) grid (3,2);
\draw[thick] (0,0)--(0,3)--(1,3)--(1,2)--(3,2)--(3,1)--(4,1)--(4,0)--cycle;
\draw (1.5,.5) node{$s$};
\draw (1.5,1.5) node{$a$};
\draw (2.5,.5) node{$l$};
\draw (3.5,.5) node {$l$};
\draw (8.5,1.5) node[align=left] {$a(s)=\# a=1$ \\ $\ell(s)=\# l=2$};
\end{tikzpicture}
\end{center}
\end{example}

The other combinatorial lens we will use to view partitions is the fermionic viewpoint, so named in .  

The boundary path is the directed lattice path starting at postive infinity on the $y$ axis, descending along the $y$ axis, tracing the boundary of $\lambda$, and then going along the $x$-axis to positive infinity.  One way to view this is as a bi-infinte word consisting of the letters $S$s and $E$; we will use the \emph{Maya diagram}, this is, on the simplest level, just translating $S$'s into empty circles, and the $E$'s into filled in circles, or stones.  



The cells $\square\in\lambda$ are in bijection with the
\emph{inversions} of the boundary path; that is, by pairs of segments
$(\text{step}_1, \text{step}_2)$, where $\text{step}_1$ occurs before $\text{step}_2$, but $\text{step}_1$ is traveling E and $\text{step}_2$ is traveing S.  The bijection sends $\square$ to the segments at the end of its arm and leg.





\subsection{Introduction to Hilbert schemes}

Throughout, $R=\C[x,y]$ will be the polynomial ring in two variables.  

The Hilbert scheme of points in the plane parameterizes ideals $\mathcal{I}$ of $R$ of codimension $n$.
$$\Hilb_n(\C^2)=\{\mathcal{I}\subset R | \dim_\C R/\mathcal{I}=n\}$$
The space $\Hilb_n(\C^2)$ is smooth and connected of dimension $2n$.
d
Geometrically, $\Hilb_n(\C^2)$ should be thought of as follows.  Let $\Sym^n(\C^2)=(\C^2)^n/S_n$ be the set of $n$ onordered points in $\C^2$.  $\Sym^n(\C^2)$ is singular for $n>1$, with singularities occuring where the points are not unique.  
If $P=\{p_1,p_2,\dots, p_n\}$ is a set of $n$ distinct points in the plane, thenon the one hand the ideal $\mathcal{I}_{P}=\{f\in R| f(p)=0 \text{ for }p\in \mathbf{p}\}$ of functions vanishing on $P$ is in $\Hilb_n$.  If two of the points collide, there is still a limiting ideal $\mathcal{I}$ consisting of.  The resulting ideal is not reduced, and the non-reduced structure ``remembers'' some of how they collide.

The ring $R/\mathcal{I}$ is the structure ring of $n$ points, counted with multiplicity, in the plane; when points collide, they have a non-reduced scheme structure.  
v
\begin{example}
Let $(a,b)\neq (0,0)\in\C^2$, and let $P_t$ be the pair of distinct points $(0,0)$ and $(at,bt)$.  What is the limit of $\mathcal{I}_{P_t}$ as $t\to 0$?


The ideal $\mathcal{I}_{P_t}$ of functions vanishing on these points is 
$$(\{x,y\})(\{x-at,y-bt\})=(\{x(x-at),x(y-bt), y(x-at), y(y-bt)\}$$
Setting $t=0$, we see that the limiting ideal should contain $x^2, xy$ and $y^2$.  If it were just the ideal generated by these three monomials, it would have codimension 3, not two, so it must contain something else.

Taking the difference of the the middle two generators, we see that $\mathcal{I}_{P_t}$ contains the element $btx-aty$, and since it is an ideal and $t$ is a nonzero scalar, it must contain $bx-ay$.  Since this element is independent of $t$, we see that it certain should be contained in the limit as well; including this along with all monomials of degree two or higher gives an ideal of codimension 2 as desired.

Another way of writing the ideal $\mathcal{I}_{P_0}$ is as 
$$\mathcal{I}_{P_0}=\{f\in R| f(0)=\partial_vf(0)=0\}$$
where here $v=(a,b)$ is the direction the two points collided in.  Thus, the non-reduced scheme structure at $\mathcal{I}_{P_0}$ \emph{remembers} the direction the in which the points collided. 
\end{example}

\begin{definition}
The \emph{Hilbert-Chow morphism} $HC:\Hilb_n(S)\to\Sym^n(S)$ sends an ideal $\mathcal{I}$ to 
$$HC:\mathcal{I}\mapsto \sum_{p\in S} \dim_p (R/\mathcal{I})$$
\end{definition}


For $n=1$, the Hilbert-Chow morphism is n isomorphism $\Hilb_1(S)\to S$.  When $n=2$, $HC$ is an isomorphism over the locus of distinct points, while over any point $2p\in\Sym^2(S)$, the fiber of $HC$ is a $\proj^1$.

The Hilbert-Chow morphism is a resolution of singularities.


The torus $(\C^*)^2$ acts on the plane $\C^2$ in the obvious way, $(s,t)\cdot (\alpha,\beta)=(s\alpha,t\beta)$ and hence on $R=\C[x,y]$ and $\Hilb_n(\C^2)$.  

\begin{warning} \label{warning:action-sign}

 The elements of $R$ are functions on $\C^2$, and thus $(\C^*)^2$ acts with \emph{opposite} weights as might naively be expected, that is 
$$(s,t)\cdot x^ny^m=s^{-n}t^{-m}x^ny^m$$

\end{warning}

\begin{definition}
The \emph{tautological bundle} $\mathbb{E}$ over the Hilbert scheme of points has fiber $R/\mathcal{I}$ over $\mathcal{I}$; it is a rank $n$ bundle on $\Hilb_n$.
\end{definition}


\subsection{Structure in the Topology of Hilbert schemes}


The main motivation of this work is the structure found in the topology of Hilbert schemes of points in surfaces.  This section recalls this structure.

Perhaps the key point in these results is the appears simpler when looking at the Hilbert schemes together for all $n$, rather than just for any fixed $n$.  

\begin{definition}
A \emph{graded space} $X$ is just a disjoint union $$X=\amalg_{n\in\Z} X_n$$
Often our spaces will be positively graded, and we will denote the graded space as formal power series with coefficients in $\Var, \Top$, etc.

$$X=\oplus X_nq^n$$
\end{definition}


This is manifested on the first level by product formulas.  
\begin{definition}
Let 
$$\HG_S=\bigoplus_{n=0}^\infty \Hilb_n(S)q^n$$
\end{definition}

\begin{definition}
For a space $X$, the Poincare polynomial 
$$P_t(X)=\sum_{k=0}^\infty b_k(X)$$
If $X$ is a graded topological space, with finite dimensional graded pieces, then $P_t(X)$ is naturally an element of $\Z[t][[q]]$.
\end{definition}


\begin{definition}
The \emph{Grothendieck ring of varieties} $K_0(\Var_k)$ is a quotient of the free abelian group on the set of isomorphism classes of varities over $k$.  We quotient out by relations of the form
$$[X]=[Y]+[X\setminus Y]$$
whenever $Y$ is a closed subvariety of $X$.   

The product structure is given by
$$[X]\times [Y]=[X\times Y]$$

Again, if $X$ is a graded space with positively graded pieces, then $[X]$ will be an element of $K_0(\Var_k)[[q]]$.

In $K_0(\Var_k)$, affine spaces are denoted $\LL^k=[\C^k]$.

\end{definition}

\begin{example}[Projective space]
The fact that the Riemann sphere is obtained from $\C$ by adding a point at infinity translates to the identity $[\proj^1]=\LL+1$.

More generally, the decomposition $\proj^n=\C^n\cup \proj^{n-1}$ gives
$$[\proj^n]=\LL^n+[\proj^{n-1}]=\LL^n+\LL^{n-1}+\cdots 1=\frac{1-\LL^{n+1}}{1-\LL}$$


If we let $\proj=\sum \proj^nq^n$, then a short computation with the previous line gives
$$[\proj]=\frac{1}{(1-q)(1-q\LL)}$$
\end{example}


The first such product formula was the following result of 

The study of the topology of $\Hilb_n(S)$ began with the following result:
\begin{theorem}[Ellingsrud and Str\o mme \cite{ES}]
$$P_t(\HG_S)=\prod_{\ell=1}^\infty \frac{1}{1-t^{2\ell-2}q^\ell}$$
$$[\HG_S]=\prod_{\ell=1}^\infty \frac{1}{1-\LL^{\ell +1}q^\ell}$$
\end{theorem}



Ellingsrud and Str\o mme actually focused more on $\Hilb_n(\proj^2)$.  We recount their proof in the next two sections as the geometric content of our work is essentially a small continuation of their methods.



G\"ottsche extended 

\begin{theorem}[G\"ottsche, \cite{gottsche}]
Let $S$ be a smooth quasiprojective surface, and when $b_i$ appears in isolation, let it be $b_i(S)$.  Then:
$$P_t(\HG_S)=\prod_{\ell\geq 1} \frac{(1+t^{2\ell-1}q^\ell)^{b_1}(1+t^{2\ell+1}q^\ell)^{b_3}}{(1-t^{2\ell-2}q^\ell)^{b_0}(1-t^{2\ell}q^\ell)^{b_2}(1-t^{2\ell+2}q^\ell)^{b_4}}$$
\end{theorem}

G\"ottsche's original proof used the Weil conjectures to reduce to the local model of the smooth surfaces, $S=\C^2$, where his formula is exactly Ellingsrud and Str\o mme's result.

Later proofs used power structures on the Grothendieck ring of varieties to  ...
.  This power structure was extended to orbifolds in \cite{}, and so the missing piece is an analogous result to Ellingsrud-Str\o mme for the local models of orbifold surfaces, namely $\C^2/G$ for $G$ a finite group.  

\subsubsection{Corollaries and extensions of G\"ottsche}
We will also investigate orbifold analogs of several results closely related to G\"ottsche's formula.

First, G\"ottsche observed the following easy corollary to his formula:
\begin{corollary}  Suppose $S$ is connected.
Then for fixed $k$ and large $n$, $b_{k}(S^{[n]})$ stabilizes
\end{corollary}

\begin{proof}
The point is that there is exactly one factor in the product formula that has no $t$'s, namely $1/(1-q)$.  If we remove this term, we may expand the rest of the product as a series \emph{in $t$}, and the coefficient of $t^k$ will be a polynomial $p_k(q)$ in $q$.  

Adding the $1/(1-q)$ term back in means that once a monomial $q^kt^m$ occurs, it will now also occur for all higher powers of $k$.  Hence, we see that once $n\geq\text{deg}(p_k)$, we will have $b_k(\Hilb_n(S))=p_k(1)$.
\end{proof}

Grojnowski and Nakajima \cite{grojnowski, nakajimaheisenberg} gave another proof by categorifying G\"ottsche's formula.  The power series on the right hand side is the $(q,t)$ character of the highest weight representation of the Heisenberg algebra modeled on $H^*(S)$.  Thus, one is lead to hope that the vector space 
$$H^*(\HG_S)=\bigoplus_{k\geq 0} H^k()$$
can naturally be given an action of this Heisenberg algebra.  This is exactly what Grojnowski and Nakajima did, using nested Hilbert schemes of points.  This result categorifies G\"ottsche's formula, and gives another proof of it.

In case the quasiprojective surface $S$ is the minimal resolution of the ADE singularity, this Heisenberg action is part of a quantumn group action for the corresponding semisimple quantum group.

Further categorifications of this action are areas of active study; the Heisenberg action on $\HG_{\C^2}$ is related to, while in case $S$ is the resolution of an ADE singularity the action has been further categorified by Cautis and Licata \cite{CL} to a ``categorical action'' on the derived category of Coherent sheaves $D(\Hilb_S)$; the quantum group action is categorified in \cite{CL2}.

In another direction, Heisenberg actions on $K^*(\HG_{\C^2})$ have been studied and related to shuffle algberas in \cite{FT, SV}.




\subsection{Proof of \texorpdfstring{Proof of Ellingsrud and Str\o mme}{Proof of Ellingsrud and Stromme}}

In this section we present a detailed proof of Ellingsrud and Str\o mme's result.  Many proofs of this are in the literature; for instance \cite{cheah} (IS IT IN THIS OR ONLY THE THESIS?) gives a largely combinatorial one, and \cite{nakajimabook}   Our proof is essentially the combinatorial one in \cite{cheah}, or , but made more so in that we use the boundary path of the partition instead of ``systems of arrows''.  

The main tool we use to understand the topology of $\Hilb_n(\C^2)$ is the $(\C^*)^2$ action induced by the $(\C^*)^2$ action on $\C^2$.  




\subsubsection{Warm-up: Euler Characteristic of $\Hilb_n(\C^2)$}

Setting $t=-1$ in Lemma  gives
$$\chi(\Hilb_{\C^2})=\prod_{k\geq 0} \frac{1}{1-q^k}$$
and so $\chi(\Hilb_n(\C^2))=p(n)$.

The pertinent fact about the Euler characteristic that makes it easy to compute is that it is additive over subvarieties: if $X\subset Y$ closed, then $\chi(Y)=\chi(X)+\chi(Y\setminus X)$.

Note that this is not true for real manifolds; take, the example of a point in $S^1$, both the point and its complement have euler characteristic 1, while $S^1$ has euler characteristic 0.  We briefly sketch the proof in case $X$ is a complex \emph{submanifold}.  If $U$ and $V$ are an open covering, then the Mayer-Vietoris sequence and the definition of $\chi$ give
$$\chi(Y)=\chi(U)+\chi(V)-\chi(U\cap V)$$
Taking $U=Y\setminus X$, and $V$ a tubular neighborhood of $X$, we see that $U\cap V$ we will be a homotopy equivalent to the normal sphere bundle of $X$ in $Y$.  Each sphere we will be the unit sphere sitting inside the normal bundle of $X$ -- since $X$ is a complex submanifold, this will be an odd dimensional sphere, having euler characteristic zero.  

\begin{lemma}
Suppose that $X$ has a $\C^*$ action, with fixed point set $X^{\C^*}$.  Then $\chi(X)=\chi(X^{\C^*})$.
\end{lemma}

\begin{proof}[idea]
There is a stratification of $X$ according to its stabilizer subgroup.  For a subgroup $H\subset G$, we let $X^{(H)}$ denote the subset with stabilizer group $H$.  Then $X^{(H)}$ has a free action of $G/H$, and so is a $G/H$ bundle over the quotient $X^{(H)}/(G/H)$.  Since 

We apply $\chi(X)=\chi(X^T)+\chi(X\setminus X^T)$.  The action of $\C^*$ on $X\setminus X^T$ will further decompose
\end{proof}


\begin{lemma}
The $T$-fixed points on $\Hilb_n(\C^2)$ are the monomial ideals
\end{lemma}

\begin{proof}
A monomial $x^\alpha y^\beta$ is just scaled by the action of $T$, and hence an ideal generated by monomials will be $T$ fixed.  

In the other direction, suppose that an ideal is $T$ fixed, and $f\in\mathcal{I}$ is a generator that is not a monomial; we show all monomials in $f$ are actually in $\mathcal{I}$.  The idea is that $t\cdot f\in\mathcal{I}$ for any $t\in T$.  If $f$ is the sum of $k$ monomials, then there are $t_1,\cdots, t_k\in\C^*$ so that $t_i f$ are linearly independent over $\C$; inverting the matrix that expresses $t_i f$ in terms of the monomials $m_i$ we see that $m_i$ is a linear combination of the $f_i$ and hence in $\mathcal{I}$.
\end{proof}

But there is a bijection between monomial ideals with $\dim R/\mathcal{I}=n$ and partitions of $n$ -- the monomials not in $\mathcal{I}$ will form a basis for $\dim R/\mathcal{I}$, and will form the squares of a partition.

\begin{center}
\begin{tikzpicture}
\draw[fill=red!20] (0,2) rectangle (1,3);
\draw[fill=red!20] (1,1) rectangle (2,2);
\draw[fill=red!20] (3,0) rectangle (4,1);


\draw[thin, gray] (0,0) grid (4.5,3.5);
\foreach \x in {0,...,4}
   \foreach \y in {0,...,3}
       \node at (\x+.5,\y+.5) {$x^\x y^\y$};
\draw[ultra thick, black] (0,0)--(0,2)--(1,2)--(1,1)--(3,1)--(3,0)--cycle;

\draw node at (7.5, 2) {$\begin{array}{cc} \mathcal{I} & \lambda \\ (x^3,xy,y^2) & (2,1,1) \end{array}$};
\end{tikzpicture}
\end{center}





\subsubsection{Bia\l ynicki-Birula decomposition}

To capture the Poincare polynomial instead of just the Euler characteristic, Ellingsrud and Str\o mme make use of the Bia\l ynicki-Birula decomposition of $\Hilb_n(\C^2)$ \cite{BB}, which is closely related to Morse theory. 




The role of the Morse flow will be played by the flow $x\mapsto \varepsilon x$ as $\varepsilon\in\C^*$ tends toward zero.  We assume that the $\C^*$ action on $X$ is such that this limit point exists for all $x\in X$.  

Let $p$ be a fixed point of the $\C^*$ action.  Then linearizing the $\C^*$ action on $X$ gives a $\C^*$ action on $T_pX$, and so $T_pX$ is not just a vector space but a $\C^*$ representation, and hence decomposes into a direct sum of irreducible representations.  Let $V_a$ denote the irreducible representation of $\C^*$ where $\varepsilon\in\C^*$ acts as $\varepsilon^a$.  


Let $T_p^+X$ (respectively $T_p^-X$) denote the subspace of $T_pX$ where the $\C^*$ action acts with a positive (respectively negative) exponent, and let $T_p^0(X)$ denote the subspace where $\C^*$ acts trivially.  Thus, if we consider the flow $X$ sending $\lambda$ toward 0, $T_p^+X$ are the directions that are flowing twoard $p$, and $T_p^-X$ are the directions flowing away from $p$.  

Note that if $p$ is an isolated fixed point, then $T_p^0(X)=0$.

$$T_pX=\bigoplus_{n\in\Z} V_n^{e_n}$$
then
$$T_p^+X=\bigoplus_{n>0} V_n^{e_n}$$

Let 
$$\mathcal{S}_p=\{x\in X|\lim_{\varepsilon\to 0} \varepsilon x=p\}$$
Then the Bia\l ynicki-Birula decomposition states that $X=\sqcup_{p}\mathcal{S}_p$; the point is that $\mathcal{S}_p$ is a subvariety isomorphic to $A_k$.





We saw in Section [REFER BACK TO ] that if $X$ is a variety with a $\C^*$ action with $k$ isolated fixed points, then $\chi(X)=k$.  If we know the weights of the $\C^*$ action on the tangent spaces of the fixed points, the Bia\l ynicki-Birula decomposition leverages this result to give the betti numbers of $X$, or even further the class of $X$ in the Grothendieck ring of varieties.

\subsubsection{Tangent space to a monomial ideal}
To apply the Bia\l ynicki-Birula decomposition to $\Hilb_n(\C^2)$, one must have a $\C^*$-action with eights of the $\C^*$ action on $T_\lambda\Hilb_n(\C^2)$.

\begin{lemma}[Ellingsrud and Str\o mme \cite{ES}] \label{lem:torus-weights}
$$T_\lambda \Hilb_n(\C^2)=\sum_{\square\in\lambda} \left(x^{-\ell(\square)} y^{a(\square)+1}+x^{\ell(\square)+1}y^{-a(\square)}\right)$$
\end{lemma}

Before giving a proof, we show how Lemma \ref{lem:torus-weights} implies Ellingsrud and Str\o mme's results. 

To apply Bia\l ynicki-Birula, one needs a $\C^*$ action with isolated fixed points, not a $(\C^*)^2$ action.  The solution is to pick any generic subtorus.  We will pick the action on $\C^2$ with weights $(\varepsilon,1)$, so that in the flow $\lambda\to 0$ the positive directions are the stable directions.  The choice of subtorus weights should really be integer numbers, and by $(1,\varepsilon)$ we mean an action $(N,1)$ with $N>>n$. Since $a(\square)$ and $\ell(\square)$ are bounded by $n$, this means that we will always have $a(\square)\varepsilon<<1$, and similarly with $\ell(\square)$.

Recall Warning \ref{warning:action-sign} that $\C^*$ will act with \emph{opposite} weights on $x$ and $y$, so that $t\cdot(x^\alpha y^\beta)=t^{-\alpha\varepsilon-\beta}x^\alpha y^\beta$.  

Restricted to this subtorus, the term $x^{-\ell(\square)} y^{a(\square)+1}$ has weight $\varepsilon\ell(\square)-a(\square)-1$, which will never be positive or zero.  The term $x^{\ell(\square)+1}y^{-a(\square)}$ has weight $a(\square)-\varepsilon(\ell(\square)+1)$, which will be positive if and only if $a(\square)>0$.  $a(\square)=0$ exactly on the top square of each column of $\lambda$, and so viewing the columns as the parts of $\lambda$, we see that the $i$th column contributes $\lambda_i-1$ to $\dim^+(\lambda)$, as we meant to show.



\begin{center}
\begin{tikzpicture}[scale=.5]
\draw[fill=red!20] (0,0) rectangle (1,2);
\draw[fill=red!20] (0,0) rectangle (3,1);


\draw[thin, gray] (0,0) grid (1,3);
\draw[thin, gray] (1,0) grid (3,2);

\draw[thick] (0,0)--(0,3)--(1,3)--(1,2)--(3,2)--(3,1)--(4,1)--(4,0)--cycle;

\draw (2,-1) node {$\dim^+(\lambda)$ is number of shaded squares};
\end{tikzpicture}
\end{center}



It should be noted that although any generic $\C^*$ action will have isolated fixed points, and hence can be used to compute the cohomology of $\Hilb_n(\C^2)$, different $\C^*$ action will provide different statistics.  Since all these statistics compute the same thing, they must be equidistributed.  This was apparently first observed geometrically by Haiman, and Loehr and Warrington gave a combinatorial proof in \cite{LW}, furthermore they extend the statistics to the non-isolated case.  It would be interesting to have a geometric understanding of the bijections in \cite{LW}.




We begin with the following the description of the the tangent space to the Hilbert scheme:


\begin{lemma} \label{lem:tangent-hom}
$$T_{\mathcal{I}}\Hilb_n(\C^2)=\Hom_R(\mathcal{I},R/\mathcal{I})$$
\end{lemma}

Lemma \ref{lem:tangent-hom} actually holds in any dimension, for any smooth variety.

Before giving a quick proof, we give a plausibility argument in analogy with the Grassmannian.  If $V\subset W$ is a $k$ dimensional subspace, then $T_V Gr_k(W)=\Hom(V, V^\perp)=\Hom(V, W/V)$.  This gives a description of the deformations of $\mathcal{I}$ as a vector subspace of $R$.  For the deformation to remain an ideal it is plausible that we should require the deformation to be a map of $R$ modules and not just vector spaces.

\begin{proof}
More formally, it is a general fact that first order deformations of objects are given by $\Ext^1(\mathcal{F},\mathcal{F})$, and obstructions to these deformations are given by $\Ext^2(\mathcal{F},\mathcal{F})$; starting from this fact and considering the long exact sequences by taking $\Hom(\mathcal{I}, -)$ to
$$0\to \mathcal{I}\to R\to R/\mathcal{I}\to 0$$
gives 
$$0\to\Hom(\mathcal{I},\mathcal{I})\to \Hom(\mathcal{I}, R)\to\Hom(\mathcal{I},R/\mathcal{I})\to\Ext^1(\mathcal{I}, \mathcal{I})\to\Ext^1(\mathcal{I},R)$$
Since $\mathcal{I}$ is an ideal that eventually contains $x^n, y^m$, we have $\Hom(\mathcal{I},\mathcal{I})\cong\Hom(\mathcal{I}, R)$ and $\Ext^1(\mathcal{I},R)=0$, and so indeed we have $\Hom(\mathcal{I},R/\mathcal{I})\cong\Ext^1(\mathcal{I}, \mathcal{I})$.
\end{proof}

We can now complete the proof of the tangent weight statement.


Let $T_\lambda^{a,b}$ denote the $(a,b)$-isotypical component of $T_{\mathcal{I}_\lambda}\Hilb_n(\C^2)$.  Let $f\in\Hom(\mathcal{I}_\lambda, R/\mathcal{I}_\lambda)$ have weight $(a,b)$.
$$f(x^{\alpha}y^\beta)=cx^{\alpha-a}y^{\beta-b}$$
for some constant $c_{\alpha, \beta}$ (that can be different for different monomials).  


Let $P_\lambda$ be the boundary path of $\lambda$; for $(a,b)$ in $\Z^2$, let $P_\lambda(a,b)$ denote the boundary path of $\lambda$ shifted to the right by $a$ and up by $b$.

\begin{lemma} Let $B_\lambda(a,b)$ be the set of bounded regions above $P_\lambda$ and below $P_\lambda(a,b)$.  For $U\in B_\lambda(a,b)$, let $f_U$ denote the map that multiplies monomials in $U$ by $x^{-a}y^{-b}$ and sends monomials not in $B_\lambda(a,b)$ to $0$.  Then the $f_U$ form a basis for $T_\lambda^{a,b}$.
\end{lemma}

\begin{proof}
To be a map of $R$-modules the map must commute with multipication by $x$ and $y$.  Thus, if $x^{\alpha-a+1}y^{\beta-b}\notin \mathcal{I}$, we must have $c_{\alpha+1,\beta}=c_{\alpha,\beta}$, and similarly with $y$.  Thus, to be a map of $R$-modules, we see that monomials in the same component must get multiplied by the same constant.

If a region is not bounded, then one of the generators of $\mathcal{I}$ contained in that region would be mapped to a monomial with negative exponents, which it can't do; therefore it must be multiplied by 0.
\begin{center}
\begin{tikzpicture}[scale=.5]

\fill[red!20!white] (0,5)--(0,3)--(1,3)--(1,5)--cycle;
\fill[red!20!white] (1,3)--(2,3)--(2,2)--(1,2)--cycle;
\fill[red!20!white] (3,2)--(4,2)--(4,1)--(3,1)--cycle;
\fill[red!20!white] (4,1)--(5,1)--(5,0)--(4,0)--cycle;
\draw[thin, gray] (0,0) grid (5,5);
\draw (.5,3.5) node{$y^3$};
\draw (1.5,2.5) node{$xy^2$};
\draw (3.5,1.5) node{$x^3y$};
\draw (4.5,.5) node{$x^4$};

\begin{scope}[xshift=-.1cm, yshift=-.1cm]
\draw[thick, red,  triangle 60 reversed-triangle 60] (0,5)--(0,3)--(1,3)--(1,2)--(3,2)--(3,1)--(4,1)--(4,0)--(5,0);
\end{scope}
\begin{scope}[xshift=1.1cm, yshift=.1cm]
\draw[thick, blue, dashed, triangle 60 reversed-triangle 60]  (0,5)--(0,3)--(1,3)--(1,2)--(3,2)--(3,1)--(4,1)--(4,0)--(5,0);
\end{scope}

\begin{scope}[xshift=8cm]
\fill[red!20!white] (0,5)--(2,5)--(2,3)--(3,3)--(3,2)--(1,2)--(1,3)--(0,3)--cycle;
\fill[red!20!white] (3,2)--(5,2)--(5,1)--(6,1)--(6,0)--(4,0)--(4,1)--(3,1)--cycle;

\draw[thin, gray] (0,0) grid (5,5);
\draw (.5,3.5) node{$y^3$};
\draw (1.5,2.5) node{$xy^2$};
\draw (3.5,1.5) node{$x^3y$};
\draw (4.5,.5) node{$x^4$};
\begin{scope}[xshift=-.1cm, yshift=-.1cm]
\draw[thick, red,  triangle 60 reversed-triangle 60] (0,5)--(0,3)--(1,3)--(1,2)--(3,2)--(3,1)--(4,1)--(4,0)--(5,0);
\end{scope}
\begin{scope}[xshift=2.1cm, yshift=.1cm]
\draw[thick, blue, dashed, triangle 60 reversed-triangle 60]  (0,5)--(0,3)--(1,3)--(1,2)--(3,2)--(3,1)--(4,1)--(4,0)--(5,0);
\end{scope}
\end{scope}



\end{tikzpicture}
\end{center}
The picture on the left shows $(1,0)$ is three dimensional.  The region containing $y^3$ is unbounded; indeed, $y^3$ would have to map to $x^{-1}y^3\notin R/\mathcal{I}$.  

The picture on the right shows $(2,0)$ is one dimensional; $y^3$ ad $xy^2$ would each map to things not in $R/\mathcal{I}$ because they are in an unbounded region.  We also have that $x^3y$ and $x^4$ must get multiplied by the same constant, since $xf(x^3y)=x^2y=yf(x^4)$ and $x^2y\in R\mathcal{I}$.
\end{proof}


We now prove Lemma \ref{lem:torus-weights}.  First, observe that $T_\lambda^{(a,b)}$ is empty if $(a,b)$ are both negative or both non-negative.  If both are non-negative, then there are no cells at all below $P_\lambda(a,b)$ and above $P_\lambda$.  If both are positive, then the cells below $P_\lambda(a,b)$ and $P_\lambda$ form a single unbounded region.  
\begin{center}
\begin{tikzpicture}[scale=.5]
\fill[blue!20!white] (0,5)--(1,5)--(1,4)--(2,4)--(2,3)--(4,3)--(4,2)--(5,2)--(5,0)--(4,0)--(4,1)--(3,1)--(3,2)--(1,2)--(1,3)--(0,3)--cycle;
\draw (.5,3.5) node{$y^3$};
\draw (1.5,2.5) node{$xy^2$};
\draw (3.5,1.5) node{$x^3y$};
\draw (4.5,.5) node{$x^4$};
\draw[thin, gray] (0,0) grid (5,5);
\begin{scope}[xshift=-.1cm, yshift=-.1cm]
\draw[thick, red,  triangle 60 reversed-triangle 60] (0,5)--(0,3)--(1,3)--(1,2)--(3,2)--(3,1)--(4,1)--(4,0)--(5,0);
\end{scope}
\begin{scope}[xshift=1.1cm, yshift=1.1cm]
\draw[thick, blue, dashed, triangle 60 reversed-triangle 60]  (0,5)--(0,3)--(1,3)--(1,2)--(3,2)--(3,1)--(4,1)--(4,0)--(5,0);
\end{scope}

\begin{scope}[xshift=9cm]
\draw[thin, gray] (0,0) grid (5,5);
\draw (.5,3.5) node{$y^3$};
\draw (1.5,2.5) node{$xy^2$};
\draw (3.5,1.5) node{$x^3y$};
\draw (4.5,.5) node{$x^4$};
\begin{scope}[xshift=-.1cm, yshift=-.1cm]
\draw[thick, red,  triangle 60 reversed-triangle 60] (0,5)--(0,3)--(1,3)--(1,2)--(3,2)--(3,1)--(4,1)--(4,0)--(5,0);
\end{scope}
\begin{scope}[xshift=-.9cm, yshift=.1cm]
\draw[thick, blue, dashed, triangle 60 reversed-triangle 60]  (0,5)--(0,3)--(1,3)--(1,2)--(3,2)--(3,1)--(4,1)--(4,0)--(5,0);
\end{scope}
\end{scope}

\end{tikzpicture}
\end{center}
The left shows that for $(1,1)$ are both positive, then there is a single unbounded region.  The right shows that for $(-1,0)$ there are no cells above $P_\lambda$ and below $P_\lambda(-1,0)$.
Thus, exactly one of $a,b$ is negative; assume it is $a$.  We must show that the elements of $B_\lambda(a,b)$ are in bijection with the cells $\square\in\lambda$ having  $a=\ell(\square)$ and $b=-a(\square)-1$ (recall the sign from Warning \ref{warning:action-sign}.

Consider a region of cells above $P_\lambda$ and below $P_\lambda(-a,-b)$.  There are two possible directions this region could be unbounded -- along the positive $y$ or $x$-axes.  Since $b$ is positive, there are no squares far along the $x$-axis both below $P_\lambda(-a,-b)$  and above $P_\lambda$, and so the region is automatically bounded in that direction.  

We must therefore guarantee the region is bounded at the bottom right.  This will happen when a South step of $P_\lambda$ starts at the same point as East step of $P_\lambda(-a,-b)$.  




Bounded cells well thus be in bijection with pairs of such cells.  Translating the East step back $(a,b)$ to its original place on the boundary strip, we find these two steps give an inversion in the $P_\lambda$ and hence a cell of $\lambda$.  The arm length of this cell will be $b+1$, and the leg will have size $a$. 

An analogous argument shows that tangent directions with $b$ negative correspond to the other terms in the sum.




\section{Background: Hilbert schemes of points on orbifolds}


\subsection{Orbfiolds and stacks}
We turn now to Hilbert schemes on orbifolds.  An orbifold is, first of all, a space where every point has a neighborhood isomorphic to $\C^n/G$, where $G$ is a finite group.  Viewed this way, orbifolds are mildly singular spaces.  

Instead, it is often better to take the stacky point of view, in which we change categories, giving 


In the naive point of view, the local model is $G$-invariant objects of $\C^n$; in the stacky point of view, this is replaced with $G$-equivariant behavior.  Rather than 



\subsection{Orbifold Hilbert schemes}




Thus, the Hilbert scheme of $n$ points on $\C^2/G$ should parameterize $G$-equivariant ideals of $R$, which are $G$-invariant ideals.  
$$\Hilb_n(\C^2/G)=\{\mathcal{I}\subset R | \} =\Hilb_n(\C^2)^G$$

Viewing $\Hilb_n([\C^2/G])$ as the fixed point set of the $G$ action on $\C^2$ is much simpler than the stack theoretic viewpoint, and will be our min way of dealing with the stacks.  The stack theoretic viewpoint provides our motivation -- from this point of view, orbifolds should be have like smooth spaces, and hence one is lead to look for analogs of G\"ottsche's formula for them.


As the fixed point of a finite group acting on a smooth space, we immediately see that $\Hilb_n([\C^2/G])$ are smooth.



\begin{example}
$\Hilb_1([X/G])=X^G$, the fixed point set of $G$.  This is distractly different behavior than Hilbert schemes on spaces, where $\Hilb_1(S)=S$.
\end{example}

\begin{example}
We now look at $\Hilb_2([\C^2/G])$ for $G=\Z_2, \Z_3$.  

First, we examine $\Z_2$, where the nontrivial elements acts on $\C^2$ by multiplication by $-1: (s,t)\mapsto (-s,-t)$.  First, we look at the fixed points where the two points are distinct: it sends a pair of distinct pounts $\{p,q\}$ to $\{-p,-q\}$, and so this is fixed if and only if $p=-q$.  As long as $p\neq 0, -p\neq p$, and so we see that this locus is isomrophic to $\C^2\setminus\{(0,0)/\Z^2$.

Now we examine this action over the locus where the two points are identical.  First, if the scheme is supported at $p$, then its image under the $\Z_2$ action is supported at $-p$, and so to be fixed it must be supported over $0$.  Furthermore, any ideal supported at $0$ will be $\Z_2$ invariant -- the group action just multiplies $x, y$ by $-1$, and we had a basis that was

There are two $\Z_3$ actions on $\C^2$ with an isolated fixed point -- the diagonal action and the anti-diagonal action.

In either case, there are no fixed points on $\Hilb_2(\C^2)$ away from the locus where both points are supported at 0.  

In the diagonal case, the same argument as in the $\Z_2$ case shows that any ideal supported over the origin will be invariant, and we have
$$\Hilb_2(S_{1/3})=\proj^1$$.

In the anti-diagonal case, $x$ and $y$ are scaled by different numbers.  So the only linear generator of $\mathcal{I}, bx-ay$, will not be fixed unless $b=0$ or $a=0$, and so we see that $\Hilb_2(S_{2/3})$ consists of two points.  

Here we have another marked difference from the case of non-orbifold surfaces: Hilbert schemes of points on orbifolds need not be connected.

\end{example}

The fact that orbifold Hilbert schemes are disconnected is easily explained by a discrete invariant.  Since $R, \mathcal{I}$ both have $G$ actions, the module $R/\mathcal{I}$ is not just a vector space but a representation of $G$.  These are discrete invariants, and so obviously one representation of $G$ cannot deform into another.  

\begin{definition}
For $v$ a representation of $G$, we define
$$\Hilb_v(\C^2/G)=\{\mathcal{I}|R/\mathcal{I}=v\}$$
\end{definition}

It turns out the $\Hilb_v$ are connected.  

The tautological bundle on $\Hilb_n$ splits into $r$ distinct tautological bundles, according to how the group $G$ acts 


This discussion also helps explain why increasing $n$ a point does not correspond to a smooth point of $\C^2/G$.  Smooth points of $\C^2/G$ correspond to points in $\C^2$ where $G$ acts freely.  If $\mathcal{O}_{Gp}$ is the structure sheaf of the orbit of such a point, then $G$ acts on $\mathcal{O}_{Gp}$ as the regular representation.  Thuss

Thus, the generic point of $\Hilb_{\C[G]}({\C^2/G})$ -- often called $G-\Hilb$ in the literature -- will correspond to a smooth point of $\C^2/G$.  As $G-\Hilb$ is smooth from our discussion before, and since the image of the Hilbert-Chow morphism is the $G$ invariant sets, we really have a map $G-\Hilb\to \C^2/G$ that is an isomorphism away from the singular points, and so we see $G-\Hilb$ is a resolution of $\C^2/G$. 

\begin{center}
\begin{tikzpicture}[scale=.5]
\draw(0,-4) node{$\widetilde{S}_{1/2}$};
\draw[-latex] (3,0)--(9,0);

\draw (0,2.236) ellipse (2 and 1);
\draw (0,-2.236) ellipse (2 and 1);
\draw[red] (0,0) ellipse (2 and 1);
\fill[white] (-2,0)--(2,0)--(2,1.1)--(-2,1.1)--cycle;
\fill[white] (-2,-2.336)--(2,-2.336)--(2,-1.136)--(-2,-1.136)--cycle;
\draw[dashed] (0,-2.236) ellipse (2 and 1);
\draw[red,dashed] (0,0) ellipse (2 and 1);

\draw (2,2.236)--(2,-2.236);
\draw (-2,2.236)--(-2,-2.236);

\begin{scope}[xshift=12cm]
\draw (0,-4) node{$S_{1/2}$};
\draw (0,2.236) ellipse (2 and 1);
\draw (0,-2.236) ellipse (2 and 1);
\fill[white] (0,0)--(1.7888, -1.7888)--(-1.7888,-1.7888)--cycle;
\draw (-1.7888,-1.7888) -- (1.78888,  1.7888);
\draw (1.7888,-1.7888) -- (-1.78888,  1.7888);
\draw[dashed] (0,-2.236) ellipse (2 and 1);
\end{scope}
\end{tikzpicture}
\end{center}

In fact, singularities of the form $\C^2/G$ have a unique minimal resolution, and $G-\Hilb$ is this resolution:





\begin{theorem} \label{thm:GHilb-resolution}
The map $G-\Hilb\to \C^2/G$ is the minimal resolution.
\end{theorem}

Theorem \ref{thm:GHilb-resolution} is useful because it gives a modular interpretation of the minimal resolution.  It also shows the geometry of the minimal resolution appearing.

The resolution of the $\C^2/G$ singularity is rational; hence the exceptional locus is a tree of $\proj^1$'s joining together.  In case $G$ is cyclic, the resolution is actually a chain of rational curves with negative self intersection.  See Chapter 10 of \cite{toric} for a thorough discussion of this.

\begin{center}
\begin{tikzpicture}[xscale=.75]
\draw plot [smooth, tension=1] coordinates {(0,0) (2,.5) (4,0)};
\draw plot [smooth, tension=1] coordinates {(3,0) (5,.5) (7,0)};
\draw plot [smooth, tension=1] coordinates {(6,0) (8,.5) (10,0)};
\draw plot [smooth, tension=1] coordinates {(9,0) (11,.5) (13,0)};

\draw (2,1) node{$-2$};
\draw (5,1) node{$-4$};
\draw (8,1) node{$-3$};
\draw (11,1) node{$-3$};

\end{tikzpicture}
\end{center}

$$
\begin{bmatrix}
-2 & 1 & 0 & 0 \\
1 & -4 & 1 & 0 \\
0 & 1 & -3 & 1 \\
0 & 0 & 1 & -3 
\end{bmatrix}
$$

The number and self-intersection of the components of the exception divisor are given by the \emph{Hirzebruch-Jung} continued fraction expansion of $r/a$.  That is, if we write
$$
\frac{r}{a}=b_1-\cfrac{1}{b_2-\cfrac{1}{b_3-\cfrac{1}{\cdots - \cfrac{1}{b_r}}}}
$$  
then the exception divisor as $r$ components in a chain, and the $i$th component has self-intersection $-b_i$. 

\begin{example}[Diagonal action]
Consider the diagonal action of $\Z_r$ on $\C^2$; this corresponds to $a=1$, and so the Hirzebruch-Jung continued fraction is simply $r/1=r$.  The exceptional divisor consists of one component with self intersection $-r$.

\end{example}


\begin{example}[The anti-diagonal case]  
In this case, the Hirzebruch-Jung continued fraction is
$$r/(r-1)=\underbrace{2-\cfrac{1}{2-\cfrac{1}{\cdots-\cfrac{1}{2}}}}_{\text{$r-1$ times}}$$
Thus, the resolution consists of a chain $r-1$ -2 curves.
\end{example}

\begin{example}[The other $\Z_5$ case]
As a final example, consder $\Z_5$ acting with weights $(1,2)$.  Switching the role of $x$ and $y$, this is equivalent to $\Z_5$ acting with weights $(1,3)$.

We have the continued fraction expansions
$$5/2=3-\cfrac{1}{2}$$
$$5/3=2-\cfrac{1}{3}$$
The minimal resolution thus consists of a -3 curve meeting a -2 curve; the choice of which direction we choose as $x$ and $y$ corresponds to which edge of the chain we start at.
\end{example}

The intersection pairing of the minimal resolution $\widetilde{S}_{a/r}$ will play an important role later.  The components of the exception curve form a basis for 


\subsection{McKay Correspondence}
The McKay correspondence is relates the geometry of $\widetilde{S}_G$ to the representation theory of $G$.

\begin{definition}
Let $G$ be a finite group, and $W$ a representation of $G$.  The \emph{McKay graph} $\Gamma_V$ is the directed graph with vertex set $\irreps(G)$, and if $U$ and $V$ are two irreps of $G$, the multiplicity of the edges from $U$ to $V$ is the multiplicity of $V$ in $U\otimes W$.
\end{definition}

The subgroups of $SU_2$ have an ADE classification.  There are two infinite series: cyclic groups, binary dihedral groups, and double covers of the isometries of the symmetries of the platonic solids.

McKay observed \cite{McKay} that when $G\subset SL_2(\C)$, the McKay graph of the defining two dimension representation was the \emph{affine} version of the corresponding ADE Dynkin diagram.  Removing the vertex corresponding to the trivial representation gives the usual ADE Dynkin diagram.  Furthermore, the exceptional locus of the minimal resolution $\widetilde{S}_G$ of the corresponding singularity has all $-2$ curves, with dual graph the corresponding Dynkin diagram. 

A geometric explanation of this was first given by Gonzalez-Sprinberg and Verdier \cite{GSV}, by constructing vectors bundles $E_\rho$ on the minimal resolution $\widetilde{S}_G$, labeled by the irreducible representations $\rho$ of $G$, so that $c_1(E_\rho)$ was the corresponding component of the exception divisor.

Ito and Nakamura observed the $G\Hilb$ is the minimal resolution \cite{IN}.

The description of the minimal resolution $\widetilde{S}_G$ as $G-\Hilb$ gives a natural construction of these vector bundles: for each irrep $\rho$ of $G$, we have on the one hand a tautological bundle $\mathbb{E}_\rho$ over $G-\Hilb$, and on the other hand a vertex of the McKay graph.  

$$c_1(\mathbb{E}_\rho)=E_\rho$$

Furthermore, viewing the minimal resolution as a moduli space of objects on the orbifold gives rise to powerful tools using the derived category.  

In dimensions higher than two, Hilbert schemes of points on a smooth surface are not smooth, and we no longer have any reason to expect that $G-\Hilb$ would be smooth, or a crepant resolution of $\C^n/G$ -- indeed, in dimensions 4 or higher, gorenstein quotient singularities need not have crepant resolutions.

It turns out that in many cases of interest, $G-\Hilb$ is smooth in this case, and a crepant resolution of $\C^2/G$.  This was proven by Bridgeland, King and Reid in \cite{BKR} for dimension three, and Haiman's work on $\Hilb_n(\C^2)$ is essentially showing that it is $S_n-\Hilb(\C^n)$.



\subsection{The special McKay correspondence}

When $G$ is not in $SL_2$, the number of components in the exception divisor of $\widetilde{S}_G$ is strictly less than the number of nontrivial irreducible representations of $G$.  The \emph{special McKay correspondence} started by Wunram \cite{wunram} picks out a subset of the irreps of $G$, called the \emph{special} representations, and gives a labeling of the irreducible components of the exception divisor of $G$ by the special representations, so that we still have $c_1(\mathbb{E}_\rho)=E_\rho$.

Kidoh \cite{Kidoh} in the cyclic case and Ishii \cite{ishii} in the general case proved that we still have $G\Hilb$ is the minimal resolution.

Ito \cite{ito} gave the following combinatorial description of the special representations.  

\begin{theorem}[\cite{ito}, Theorem 3.7]
Defines $B(G)$ to be the set of monomials which are not divisible by any $G$-invariant monomial, and defines $L(G)$ to be the set of monomials not divisible by $xy, x^{|G|}, y^{|G|}$.  Then a representation $\rho$ is special if and only if the corresponding monomial is not contained in $B(G)\setminus L(G)$.
\end{theorem}


Although we will work mainly at the level of homology, it is worth noting the McKay and special McKay correspondences hold at the level of derived categories.  (KAPRANOV-VASSEROT, ISHII)


\subsection{Colored square counting}

The core construction helps us with the colored box square counting:

Consider the function
$$P_G(q_0,\dots,q_{r-1})=\sum_{\lambda\in\mathcal{P}} \mathbf{q}^{|\lambda|^G}$$
where 
$$\mathbf{q}^{|\lambda|_G}=\prod_{i=0}^{r-1} q_i^{|\lambda|^G_i}$$
that counts partitions according to their full colored square count as opposed to just their size; alternatively, the coefficient of $\mathbf{q}^v$ is the euler characteristic of $\Hilb_v([\C^2/G])$.

What can we say about $P_G$?  First, we consider the case where $G\subset SL_2$.  Let $Q=q_0q_1\cdots q_{r-1}$.  The core construction gives
$$P_G=\prod_{i=1}^\infty\frac{1}{(1-Q^i)^r} \sum_{w} Q^{A(w)}\mathbf{q}^w $$
Thus, we see in this case $P_G$ is a multivariable theta function.

In case $r=2$, this has an infinite product expansion using the Jacobi identity.  In fact, work of Boulet shows that actually the case $G=\Z_2\times \Z_2$ has an infinite product expansion.  Letting $q_{00}, q_{01}, q_{10}, q_{11}$ denote the variables, we have

\begin{theorem}[Boulet \cite{boulet}]
$$P_{\Z_2\times\Z_2}=\prod_{i=1}^\infty \frac{(1+q_{00}^jq_{01}^{j-1}q_{10}^{j-1}q_{11}^{j-1})(1+q_{00}^jq_{01}^jq_{10}^jq_{11}^{j-1})}{(1-q_{00}^jq_{01}^jq_{10}^jq_{11}^j)(1-q_{00}^jq_{01}^jq_{10}^{j-1}q_{11}^{j-1})(1-q_{00}^jq_{01}^{j-1}q_{10}^{j}q_{11}^{j-1})}$$
\end{theorem}

However, even the one variable specialization $P_{\Z_3}(q,1,1)$ does not have a nice infinite product expression, as observed by Bal\'azs Szendr\"oi \cite{Bmo}; as it has a root at $-e^{\pi/\sqrt{3}}$ \cite{Borwein2}.

\subsection{Dijkgraaf, orbifold partitions, and the topology of Quot schemes}

Specialization where one $q_i$ is set to $q$ and all others are set to one is studied in \cite{DS}, where they are all called \emph{orbifold partitions of the first type}.  There, they show that these specializations are characters of the affine Kac-Moody lie algebras.

We briefly mention that they also introduce \emph{orbifold partitions of the second type}, where only cells of the given color are coordinated.  Combinatorially, this can be viewed as putting an equivalence relation on partitions, so that two partitions are equivalent if they contain exactly the same boxes of color $i$.  Algebraically, instead of considering $G$-equivariant ideals of $R$, this is working $G$-invariantly.  That is, we have the ring $R^G$ of $G$ invariant elements of $R$.  The ring $R$ is a module over $R^G$, and splits into a direct sum of modules indexed by the irreducible representations of $R$.  Rather than counting ideals in $R$ with quotient having length $n$, they are counting submodules of $R_i$ of colength $n$.  

Geometrically, $R^G$ is the ring of functions on the singular space $\C^2/G$.  The modules $R_i$ are spaces of sections of sheave $\mathcal{F}_i$ that are .  These ideals are $T$ equivariant, and so they correspond to the euler characteristic of the quot scheme.  


\begin{question}
What can be said about the Quot?  Homology groups, decomposition in the grothendieck ring of varieites, etc.

How are these related to the topo

\end{question}



\subsection{Hilbert schemes of points in the plane}


That is, we look for 
$$\DC_G(q,t):=\sum_{n,k\geq 0 } b_k(\Hilb_n([\C^2/G])) t^kq^n$$







\subsubsection{Homological stability}

The analogs of stabilization and geometric representation theory work on the level of connected Hilbert scheme.

\begin{theorem} 
$P_t(\Hilb^{\delta+nG}_G)$ stabilizes to $1/(t,t)_\infty^{|G|}$
\end{theorem}
Note that the right hand side is independent of $m$ and $\delta$.
\begin{proof} Combinatorics -- a generalization of cores and quotients of partitions \end{proof}

\begin{conjecture}
The stable cohomology of $\Hilb^{\delta+nG}$ is freely generated by the Chern classes of the $|G|$ tautological bundles.
\end{conjecture}



\subsubsection{Heisenberg Action}


\begin{conjecture}
Let $\delta\in K_0(G)$ be small, and $G$ cyclic.  Then
$$\bigoplus_{k\geq 0} H_*(\Hilb^{\delta+kG}_G)$$ admits the action of a Heisenberg algebra based on the cohomology of the minimal resolution of $\C^2/G$.
\end{conjecture}

Evidence:
Let $c$ be the number of rational curves in the minimal resolution of $\C^2/G$.  Then
$$\mathcal{CH}^\delta_G\cdot(q,qt^2)_\infty\cdot (qt^2,qt^2)_\infty^c$$
has positive coefficients; but higher powers start giving negative coefficients.


\subsubsection{Resolutions of $(\C^2/G)^n/S_n$}
One family of resolutions

Let $X_G$ be the minimal resolution of $\C^2/G$.  Then $\Hilb_n(X_G)$ is a resolution of $(\C^2/G)^n/S_n$.


Another family of resolutions

Let $\delta\in K^0(G)$ be such that $\Hilb^\delta([\C^2/G])=pt$.  Then $\Hilb^{\delta+nG}([\C^2/G])$ is a resolution of $(\C^2/G)^n/S_n$.  



Stabilization implies that for fixed $\delta$, and large $n$, this second resolution will be bigger than the first resolution.

Changing Stability?
But, it seems as $\delta\to\infty$, this second family of resolutions converges to the first.


\section{Warm-up: Cores, quotients, and the antidiagonal case}

In this section, we focus on the anti-diagonal case $-1/r$; that is, where $A=\Z_r$ acting with weights $(1,-1)$.  In this case, much is already well known: combinatorially, cores and quotients, together with the abacus construction, give us much information.  Topologically, we are in the $SL_2$ case, and the components of the Hilbert scheme have a descipription as Nakajima quiver varieties. 

Though there has been some work connecting the combinatorial and geometric viewpoints, we suspect more c

\subsection{Colored boxes}

In the anti-diagonal case, the coloring scheme of the cells is a familiar object in the student of partitions, but not one as frequently connected to core and quotient partitions.

\begin{definition}
The \emph{content} $c(\square)$ of a cell $\square=(i,j)\in\lambda$ is $c(\square)=i-j$
\end{definition}

\begin{tikzpicture}

\draw[thin, gray] (0,0) grid (4.5,4.5);
\foreach \x in {0,...,4}
   \foreach \y in {0,...,4}
       \pgfmathsetmacro\result{\x-\y}
       \node at (\x+.5,\y+.5) {\pgfmathprintnumber{\result} };
\node at (10,2.5) {Each cell is labeled by its content};

\end{tikzpicture}


It is immediate from the definition, that if we let $\C^*_{1,-1}$ be the torus that acts on $x$ with weight 1 and $y$ with weight $0$, then the content $c(\square)$ is just the weight of the $\C^*$ action on the corresponding monomial in $R/\mathcal{I}_\lambda$.

Furthermore, the multiset of the contents of $\lambda \mod r$ is just the class of $[R/\mathcal{I}]\in K^0(\Z_r)$.


\subsection{Topology of orbifold Hilbert schemes}

To determine the topology of $\Hilb([\C^2/G])$, it is convenient to view it as $\Hilb(\C^2)^G$.  When $G$ is abelian, it can be simultaneously diagonalized, and we can pick the torus $(\C^*)^2$ we have acting on $\C^2$ to be the one acting on the diagonal coordinates for $G$, so that the $G$ action commutes with the torus action.  

Since the torus commutes with $G$, the torus action will act on $\Hilb_n([\C^2/G])$.  Since $G$ is in fact a subgroup of the torus, we see that $\Hilb_n(\C^2)^{\C^*}\subset\Hilb_n(\C^2)^G$, and so all the monomials ideals are $G$ invariant.  Thus, we see that as long as $G$ is abelian, we have
$$\chi(\Hilb_{[\C^2/G]})=\chi(\Hilb_{\C^2})=\prod\frac{1}{1-q^n}$$

Similarly, we can use the Ellingsrud-Str\o mme's calculation of torus weights on $T_\lambda\Hilb_n$ to use the Bia\l ynicki-Birula decomposition to find the Betti numbers of $\Hilb_n([\C^2/G])$.  The tangent directions to $\Hilb_n(\C^2)^G$ within $\Hilb_n(\C^2)$ are simply the $G$ invariant directions.  If $G$ is a cyclic group, a term in Lemma \ref{lem:torus-weights} will be $G$ invariant if and only if the arm and leg satisfy some linear congruence relation; for $G$ a product of two cyclic groups, it will be a system of linear relations.

\begin{definition}
Let $\lambda$ be a partition of $n$, and $k<r$ relatively prime.  Then $\dim_{k/r}(\lambda)$ is the dimension of $T_\lambda\Hilb_n([\C^2/G_{k/r}])$.

Furthermore, let $\C^*$ act on $\C^2$ with weights $(\varepsilon, 1)$, and consider the induced $\C^*$ action on $\Hilb_n([\C^2/G_{k/r}])$.  We define $\dim^{\pm}_r \lambda$ to be the dimension of the positive and negative eigenspaces of this action on $T_\lambda(\Hilb_n([\C^2/G_{k/r}])$. 
\end{definition}

\begin{lemma} We have

\begin{align*}
\dim_{k/r}(\lambda)&=\#\left\{\square\in\lambda \Big | \ell(\square)-k a(\square)\in\{-1,k\} \mod r\right\} \\
\dim^+_{k/r}(\lambda)& =\#\left\{\square\in\lambda \Big | \ell(\square)-k a(\square)=-1 \mod r \text{ and } a(\square)>0\right\} \\
\dim^-_{k/r} &= \dim_{k/r}-\dim^+_{k/r}
\end{align*}
\end{lemma}
\begin{proof}
Let $L$ denote the representation where the standard generator of $G_{k/r}$ acts as $\exp(2\pi i/r)$. Under the inclusion $i:G_{k/r}\in T$, we have $i^*(T_1)=L, i^*(T_2)=L^k$.  Thus, we see that as a $G_{k/r}$-rep, we have
$$T_\lambda \Hilb_n(\C^2)=\sum_{\square\in\lambda} L^{-\ell(\square)+ka(\square)+k}+L^{\ell(\square)+1-ka(\square)}$$
The tangent space to $\Hilb_n(\C^2/G_{k/r})$, being the $G$ invariant directions, will correspond to the exponents that are divisible by $r$.  This immediately gives the first line.

To find the positive eigendirections, we intersect our answer to the first line with Prop whatever.

The last line is just saying that all fixed points are isolated.
\end{proof}

Combinatorially, it seems awkward that cells with $a(\square)=0$ are included in the negative eigenspace instead of the positive eigenspace, and it is natural to make the following definition:
\begin{definition}
The \emph{combinatorial} positive and negative dimensions, $\cdim^\pm_{k/r}(\lambda)$ are defined by
\begin{align*}
\cdim^+_{k/r}(\lambda)&=\#\left\{\square\in\lambda \Big | \ell(\square)-k a(\square)=-1 \mod r\right\} \\
\cdim^-_{k/r}(\lambda)&=\#\left\{\square\in\lambda \Big | \ell(\square)-k a(\square)=k \mod r\right\} \\
\end{align*}
\end{definition}

\begin{example}[Hook Lengths]
The case when $G\subset SL_2$ is $k=r-1$, or equivalently, $k=-1$.  In this case, we have
$$\cdim^+_{-1/r}(\lambda)=\cdim^-_{-1/r}(\lambda)=\#\left\{\square\in\lambda \Big | h(\square)=0\mod r\right\} \\$$
and so these statistics count the number of cells with hook lengths divisible by $r$.
\end{example}

The generating functions for $\cdim^+_{k/r}$ and $\dim^+_{k/r}$ are closely related.

\begin{lemma}
$$\prod_{m>0}(1-q^{mr}t^m)\sum_{\lambda\in\PP} q^{|\lambda|}t^{\cdim^+_{k/r}(\lambda)}=\prod_{m>0} (1-q^{mr}t^{m-1}) \sum_{\lambda\in\PP} q^{|\lambda|}t^{\dim^+_{k/r}(\lambda)}$$
\end{lemma}
\begin{proof}
The two statistics $\cdim^+_{k/r}$ and $\dim_{k/r}$ both count cells $\square$ satisfying the congruence relation $\ell(\square)-ka(\square)=-1\mod r$; the combinatorial dimension includes those cells with $a(\square)=0$, while the usual dimension does not.  

If a square satifies $a(\square)=0$ it is on the top of its column; when $a(\square)=0$ the congruence relation becomes simply asking that $\ell(\square)+1$ is divisible by $r$. Since $\square$ is at the top of its column, that means the two differ whenever we have a part of $\lambda$ appearing at least $kr$ times.  In particular, the two statistics agree on partitions $\lambda$ where no part has multiplicity $r$ or greater.

If there are $r$ or more parts of size $m$ in $\lambda$, one may remove $r$ parts of them, resulting in removing an $r\times k$ rectangle of squares from $\lambda$.  Doing so will not change $\ell(\square)-ka(\square)\mod r$ for any square in $\lambda$ -- we haven't changed the arm or leg of any square to the right of the rectangle removed, while we have reduced the leg of any square to the right by $r$.  Furthermore, each row of squares we removed will contain exactly one square contributing to $\cdim^+_{k/r}(\lambda)$, since each square in a row has the same arm length, and as we move across the row from right to left the leg lengths decrease by 1.  

Thus, we have seen that if we let $\PP^{<r}$ to denote the set of partitions with multiplicities of all parts less than $r$, we have
$$\prod_{m>0}\frac{1}{1-q^{mr}t^m}\sum_{\lambda\in\PP} q^{|\lambda|}t^{\cdim^+_{k/r}(\lambda)}=\sum_{\lambda\in\PP^{<r}}q^{|\lambda|}t^{\cdim^+_{k/r}(\lambda)}$$
The same analysis holds for $\dim^+_{k/r}$, except now the top row of each rectangle removed does not contribute to $\dim^+$, explaining why the powers of $t$ in the product are one lower.

The shaded cells are the cells contributing to $\cdim^+_{1/3}(\lambda)$; that is, those cells $\square$ with $\ell(\square)-a(\square)=-1\mod 3$.  The regions contained in the dashed red lines correspond to the parts removed -- notice that each row of each region contains one shaded box, and that removing the dashed regions does not changed whether the leftover cells are shaded or not.


\begin{tikzpicture}[scale=.5]
\begin{scope}
\fill[red!20!white] (0,1) rectangle (2,2);
\fill[red!20!white] (2,2) rectangle (3,5);
\fill[red!20!white] (2,1) rectangle (3,0);
\fill[red!20!white] (3,1) rectangle (4,2);
\fill[red!20!white] (4,2) rectangle (5,4);
\fill[red!20!white] (4,0) rectangle (5,1);
\fill[red!20!white] (6,1) rectangle (9,2);
\fill[red!20!white] (9,1) rectangle (10,0);
\fill[red!20!white] (11,1) rectangle (12,2);
\fill[red!20!white] (12,1) rectangle (13,0);
\fill[red!20!white] (16,1) rectangle (17,0);

\draw[thick, dashed, red] (4.1,.1)--(4.1,3.9)--(6.9,3.9)--(6.9,.1)--cycle;
\draw[thick, dashed, red] (8.1,.1)--(8.1,1.9)--(10.9,1.9)--(10.9,.1)--cycle;
\draw[thick, dashed, red] (11.1,.1)--(11.1,1.9)--(13.9,1.9)--(13.9,.1)--cycle;
\draw[thick, dashed, red] (16.1,.1)--(16.1,.9)--(18.9,.9)--(18.9,.1)--cycle;


\path[clip, preaction={thick, draw}] (0,0)--(0,7)--(1,7)--(1,6)--(3,6)--(3,4)--(7,4)--(7,3)--(8,3)--(8,2)--(14,2)--(14,1)--(19,1)--(19,0)--cycle;
\draw[thin, gray] (0,0) grid (20,7);
\end{scope}

\begin{scope}[yshift=-10cm]
\fill[red!20!white] (0,1) rectangle (2,2);
\fill[red!20!white] (2,2) rectangle (3,5);
\fill[red!20!white] (2,1) rectangle (3,0);
\fill[red!20!white] (3,1) rectangle (5,2);


\path[clip, preaction={thick, draw}] (0,0)--(0,7)--(1,7)--(1,6)--(3,6)--(3,4)--(4,4)--(4,3)--(5,3)--(5,1)--(7,1)--(7,0)--cycle;
\draw[thin, gray] (0,0) grid (7,7);
\end{scope}

\end{tikzpicture}
\end{proof}


Combinatorially, our motivation lies in the notion of core partitions.
\begin{definition}
Let $\lambda$ be a partition.  The \emph{hook length} $h(\square)$ of $\square\in\lambda$ is $a(\square)+\ell(\square)+1$.

We say $\lambda$ is a \emph{$t$-core} if it has no cells $\square\in\lambda$ with $h(\square)=t$.
\end{definition}

There are many equivalent definitions of $t$-core partitions; when we generalize them some of these definitions will still hold and some will be different.

\begin{lemma}
The following are equivalent:
\begin{enumerate}
\item $\lambda$ is a $t$-core
\item $\lambda$ has no boundary strips of length $t$
\item $\lambda$ has no cells $\square$ with $t|h(\square)$
\item It is not possible to remove one square of each content from $\lambda$
\item $\lambda$ is the only partition with its set of contents
\item $\mathcal{I}_\lambda$ is an isolated point in $\Hilb_n([\C^2/G_{-1/t}])$
\end{enumerate}


\end{lemma}


Many of these

\subsection{Abacus construction}

The abacus construction builds on the boundary path description of partitions in Section .

Rather than working with bi-infinite strings of letters, we will prefer to work with \emph{Maya diagrams}, which we now describe.  The conventions in the following discussion are based on the ``fermionic'' viewpoint of partitions and lattice paths, based on Dirac's electron sea model, see \cite{KR} for more. 


\subsection{Paths}

We will use $\Z_{1/2}$ to denote the set of half integers $\Z+1/2$, i.e., $-1/2$ and $3/2$ are in $\Z_{1/2}$, but $2$ is not.  $\Z_{1/2}^+$ will denote the positive half integers, and $\Z_{1/2}^-$ will denote the negative.



\begin{definition}
A state $S$ is a subset $S\subset \Z^{1/2}$ so that the symmetric difference of $S$ with $\Z^-_{1/2}$ is finite; that is $S\cap\Z^+_{1/2}$ and $S^c\cap \Z^-_{1/2}$ are both finite.  We call $|S\cap \Z^-_{1/2}|-|S\cap \Z^+_{1/2}|$ the \emph{charge} of $S$.

We will typically represent a state by a \emph{Maya diagram} -- this is a sequence of circles labeled by $\Z_{1/2}$, with the positive entries going to the left and the negative entries to the right.  A black bead is placed at each of the entries of $S$, and the entries not in $S$ are displayed as white circles.
\end{definition}

We now describe a bijection between the set of partitions $\mathcal{P}$ to the set of charge 0 states/


\subsubsection{}
We draw partitions in ``Russian notation'' -- rotated $\pi/4$ radians counterclockwise and scaled up
by a factor of $\sqrt{2}$, so that each segment of the border path of $\lambda$ is centered above a half integer on the $x$-axis, with origin above the square 0.



\begin{example} \label{ex:electronstopartitions}

We illustrate the bijection in the case of $\lambda=3+2+2$.  The corresponding state $S_\lambda$ consists of two
electrons with energy $5/2$ and $1/2$, and two positrons with energy
$3/2$ and $5/2$.

\begin{center}
\begin{tikzpicture}
\begin{scope}[gray, very thin, scale=.6]
\clip (-5.5, 5.5) rectangle (5.5, -5);
\draw[rotate=45, scale=1.412] (0,0) grid (6,6);
\end{scope}

\begin{scope}[rotate=45, very thick, scale=.6*1.412]
\draw (0,5.5) -- (0, 3) -- (1,3) -- (1,2) -- (3,2) -- (3,0) -- (5.5,0);
\end{scope}

\begin{scope}[scale=.6, dotted]

\draw (-4.5,0) -- (-4.5, 4.5);
\draw (-3.5,0) -- (-3.5, 3.5);
\draw (-2.5,0) -- (-2.5, 3.5);
\draw (-1.5,0) -- (-1.5, 3.5);
\draw (-.5,0) -- (-.5, 3.5);
\draw (.5,0) -- (.5, 4.5);
\draw (1.5,0) -- (1.5, 4.5);
\draw (2.5,0) -- (2.5, 3.5);
\draw (3.5,0) -- (3.5, 3.5);
\draw (4.5,0) -- (4.5, 4.5);
\end{scope}


\begin{scope}[scale=.6, yshift=-.5cm]
\draw[solid] (0,-.5) -- (0,.5);
\draw (-5.5,0) node{$\cdots$};
\draw (-4.5,0) circle (.3) node[below=3pt]{$\frac{9}{2}$};
\draw (-3.5,0) circle (.3) node[below=3pt]{$\frac{7}{2}$};
\filldraw (-2.5,0) circle (.3) node[below=3pt]{$\frac{5}{2}$};
\draw (-1.5,0) circle (.3) node[below=3pt]{$\frac{3}{2}$};
\filldraw (-.5,0) circle (.3) node[below=3pt]{$\frac{1}{2}$};
\filldraw (.5,0) circle (.3) node[below=3pt]{$\frac{-1}{2}$};
\draw (1.5,0) circle (.3) node[below=3pt]{$\frac{-3}{2}$};
\draw (2.5,0) circle (.3) node[below=3pt]{$\frac{-5}{2}$};
\filldraw (3.5,0) circle (.3) node[below=3pt]{$\frac{-7}{2}$};
\filldraw (4.5,0) circle (.3) node[below=3pt]{$\frac{-9}{2}$};
\draw (5.5,0) node{$\cdots$};
\end{scope}
\end{tikzpicture}
\end{center}

\end{example}







The bijection between partitions and states of charge zero may be
modified to give a bijection between partitions and states of charge $c$ for any $c\in\Z$.   Simply translate the partition to the right by $c$.



\subsection{Abaci}

Rather than view the Maya diagram as a series of stones in a line, we
now view it as beads on the runner of an abacus.  Sliding the beads
to be right justified allows the charge of the state to be read off,
as it is easy to see how many electrons have been added or are missing
from the vacuum state.

In what follows, we mix our metaphors and talk about electrons and protons on runners of an abacus.

\begin{example} \label{ex:mayabijection}
Consider Example \ref{ex:particles}, where the Maya diagram consists of
two positrons and an electron.  Pushing the beads to be right
justified, we see the first bead is one step to the right of zero, and
hence the original state had charge 1.

\begin{center}
\begin{tikzpicture}[scale=.6]
\draw (0,-.5) to (0,.5);
\draw (-5.5,0) node{$\cdots$};
\draw (-4.5,0) circle (.3) node[below=3pt]{$\frac{9}{2}$};
\draw (-3.5,0) circle (.3) node[below=3pt]{$\frac{7}{2}$};
\draw (-2.5,0) circle (.3) node[below=3pt]{$\frac{5}{2}$};
\filldraw (-1.5,0) circle (.3) node[below=3pt]{$\frac{3}{2}$};
\draw (-.5,0) circle (.3) node[below=3pt]{$\frac{1}{2}$};
\draw (.5,0) circle (.3) node[below=3pt]{$\frac{-1}{2}$};
\filldraw (1.5,0) circle (.3) node[below=3pt]{$\frac{-3}{2}$};
\draw (2.5,0) circle (.3) node[below=3pt]{$\frac{-5}{2}$};
\filldraw (3.5,0) circle (.3) node[below=3pt]{$\frac{-7}{2}$};
\filldraw (4.5,0) circle (.3) node[below=3pt]{$\frac{-9}{2}$};
\draw (5.5,0) node{$\cdots$};

\begin{scope}[yshift=-2.5cm]
\draw[-triangle 45, snake=snake,line after snake=1mm] (-4,0)--(4,0) node [above,  text centered,midway] {Push beads};

\end{scope}

\begin{scope}[yshift=-4cm]
\draw (0,-.5) to (0,.5);
\draw (-5.5,0) node{$\cdots$};
\draw (-4.5,0) circle (.3) node[below=3pt]{$\frac{9}{2}$};
\draw (-3.5,0) circle (.3) node[below=3pt]{$\frac{7}{2}$};
\draw (-2.5,0) circle (.3) node[below=3pt]{$\frac{5}{2}$};
\draw (-1.5,0) circle (.3) node[below=3pt]{$\frac{3}{2}$};
\draw (-.5,0) circle (.3) node[below=3pt]{$\frac{1}{2}$};
\draw (.5,0) circle (.3) node[below=3pt]{$\frac{-1}{2}$};
\filldraw (1.5,0) circle (.3) node[below=3pt]{$\frac{-3}{2}$};
\filldraw (2.5,0) circle (.3) node[below=3pt]{$\frac{-5}{2}$};
\filldraw (3.5,0) circle (.3) node[below=3pt]{$\frac{-7}{2}$};
\filldraw (4.5,0) circle (.3) node[below=3pt]{$\frac{-9}{2}$};
\draw (5.5,0) node{$\cdots$};
\end{scope}


\end{tikzpicture}
\end{center}
\end{example}

\subsubsection{Cells and hook lengths}

The cells $\square\in\lambda$ are in bijection with the
\emph{inversions} of the boundary path; that is, by pairs of segments
$(\text{step}_1, \text{step}_2)$, where $\text{step}_1$ occurs before $\text{step}_2$,
but $\text{step}_1$ is traveling NE and $\text{step}_2$ is traveing SE.  The bijection
sends $\square$ to the segments at the end of its arm and leg.

Translating to the fermionic viewpoint, cells of $\lambda$ are in
bijection with pairs 
$$\left\{(e, e-k)\big | e\in\Z_{1/2}, k>0\right\}$$ of a filled energy level $e$ and an


The arm $a(\square)$ corresponds to the number of energy levels between $e$ and $e-k$ that are empty; the leg $\ell(\square)$ is the number of energy levels between $e$ and $e-k$ that are full.  The hook length $h(\square)$ of the corresponding cell is $k$, the distance between the two sets with 

Recall that a \emph{rim hook} of length $a$ of $\lambda$ is a size $a$ subset of the cells of $\lambda$ so that removing the rim hook gives a small partition, and the rim hook does not contain a $2\times 2$ box.  

\begin{lemma} \label{lem:rimhooks}
Rimhooks of size $a$ in $\lambda$ are in bijection with cells $\square\in\lambda$ with $h(\square)=a$.  Removing the rimhook corresponds to moving the stone at $e$ and playing at $e-k$ instead.
\end{lemma}

This is illustrated in the following picture:

\begin{tikzpicture}[scale=.8]
\begin{scope}[gray, very thin, scale=.6]
\clip (-5.5, 5.5) rectangle (5.5, -1);
\draw[rotate=45, scale=1.412, fill, red!20!white] (1,1)--(1,2)--(3,2)--(3,0)--(2,0)--(2,1)--cycle;
\draw[rotate=45, scale=1.412] (0,0) grid (6,6);
\end{scope}
\begin{scope}[rotate=45, very thick, scale=.6*1.412]
\draw (0,5.5) -- (0, 3) -- (1,3) -- (1,2) -- (3,2) -- (3,0) -- (5.5,0);
\end{scope}

\begin{scope}[scale=.6, yshift=-.5cm]

\draw (-5.5,0) node{$\cdots$};
\draw (-4.5,0) circle (.3);
\draw (-3.5,0) circle (.3);
\filldraw (-2.5,0) circle (.3);
\draw (-1.5,0) circle (.3);
\filldraw[gray] (-.5,0) circle (.3);
\filldraw (.5,0) circle (.3);
\draw (1.5,0) circle (.3);
\draw (2.5,0) circle (.3);
\filldraw (3.5,0) circle (.3);
\filldraw (4.5,0) circle (.3);
\draw (5.5,0) node{$\cdots$};

\node at (-.3, -.3) (start);
\node at (2.3, -.3) (finish);
\draw[->] (start) to [out=-30, in=210] (finish);

\end{scope}

\begin{scope}[xshift=8cm]
\begin{scope}[gray, very thin, scale=.6]
\clip (-5.5, 5.5) rectangle (5.5, -1);
\draw[rotate=45, scale=1.412] (0,0) grid (6,6);
\end{scope}
\begin{scope}[rotate=45, very thick, scale=.6*1.412]
\draw (0,5.5) -- (0, 3) -- (1,3) -- (1,1) -- (2,1) -- (2,0) -- (5.5,0);
\end{scope}

\begin{scope}[scale=.6, yshift=-.5cm]

\draw (-5.5,0) node{$\cdots$};
\draw (-4.5,0) circle (.3);
\draw (-3.5,0) circle (.3);
\filldraw (-2.5,0) circle (.3);
\draw (-1.5,0) circle (.3);
\draw (-.5,0) circle (.3);
\filldraw (.5,0) circle (.3);
\draw (1.5,0) circle (.3);
\filldraw (2.5,0) circle (.3);
\filldraw (3.5,0) circle (.3);
\filldraw (4.5,0) circle (.3);
\draw (5.5,0) node{$\cdots$};





\end{scope}
\end{scope}

\end{tikzpicture}



If $(e,e-k)$ is such a pair, reducing the energy of the electron from
$e$ to $e-k$ changes $\lambda$ by removing the
rim hook corresponding to the cell $\square$.  This rim-hook has
length $k$.


\begin{definition} 
The $r$-core of $\lambda$, denoted $\core_r(\lambda)$, the partition obtained from $\lambda$ by iteratively choosing a rim-hook of size $r$ and removing it.
\end{definition}

It is not clear from the above definition that $\core_r(\lambda)$ is well defined -- it seems possible that removing rim hooks in different orders could result in different partitions.  We will see later that it is in fact well defined.  Assuming that $\core_r(\lambda)$ is well defined, it follows from Lemma \ref{lem:rimhooks} that $\core_r(\lambda)$ is an $r$-core.


\begin{example}
The cell $\square=(2,1)$ of $\lambda=3+2+2$ (See Example \ref{ex:electronstopartitions}).
Here, $h(\square)=3$, and corresponds to the electron in energy state $1/2$
and the empty energy level $-5/2$; which are three apart.

\end{example}

\subsection{Bijections}

The essence of the cores and quotient construction is that, rather than place the electrons corresponding to $\lambda$ on one runner, we place them on $r$ different runners, putting the energy levels $ka-i-1/2$ on runner $i$ -- i.e., every $r$th bead goes to the same runner, as in Figure \ref{ }

The charge $c_i$ on the $i$th runner need not be 0, but using the change, we may still view the beads on the $i$th runner as a Maya diagram of a partition.  


\begin{definition} \label{def:quotients}
The \emph{$r$-quotient} of $\lambda$, denoted $\quot_r(\lambda)$ is an $r$-tuple of partitions, where the $i$th partition $\quot_r^i(\lambda)$ is obtained by reading off the $i$th runner of the $r$-abacus of $\lambda$ as a partition.

The size of the $r$-quotient is the sum of the sizes of the individual partitions: 
$$|\quot_r(\lambda)|=\sum_{i=1}^r |\quot_r^i(\lambda)|$$
\end{definition}

The abacus construction also gives us another way to view the $r$-core of a partition $\lambda$.  Since removing an $r$-border strip corresponds to moving a bead $r$ spots to the right to an empty position, on the $r$ abacus this corresponds to moving a bead one step to the right on its runner.  Thus, 


\begin{lemma}
We have
$$|\lambda|=|\core_r{\lambda}|+r|\quot_r{\lambda}|$$
$$\cdim^+_{-1/r}(\lambda)=|\quot_r{\lambda}|$$
\end{lemma}

\begin{proof}
To form the core of $\lambda$ from $\lambda$, we slide the beads on each runner of the abacus.  Each time we slide a bead one step, we are removing an $r$-strip from $\lambda$, and hence decreasing the size of $\lambda$ by one.  On the other hand, we are removing a single cell from $\tilde{\lambda}$.


\end{proof}


\begin{corollary} \label{cor:product-formulas-traditional-cores}

Let $\mathcal{C}_r$ denote the set of $r$-core partitions.  We have
$$\sum_{\lambda\in\mathcal{C}_a} q^{|\lambda|}=\frac{(q^r;q^r)_\infty^r}{(q;q)_\infty}$$
Moreover, we have 
$$\sum_{\lambda\in\mathcal{P}} q^{|\lambda|} t^{\cdim^+{-1/r}(\lambda)}=\frac{(q^r;q^r)_\infty^r}{(q;q)_\infty}\frac{1}{(q^rt;q^rt)_\infty^r}$$
And hence:
$$P_t(\Hilb_{[\C^2/G]})=\frac{(q^r;q^r)_\infty^r}{(q;q)_\infty}\frac{1}{(q^rt^2;q^rt^2)_\infty^{r-1}(q^r;q^rt^2)_\infty}$$
\end{corollary}





The process is reversible -- if we have $a$-different Maya diagrams whose charge sums to $0$, we may interleave the beads of them together as in Figure \ref{ }, and merge them to get a partition.  


The $a$-quotient of $\lambda$ records information about the $a$-hooks of $\lambda$.  If we have a cell $\square\in\lambda$,  with hooklength $h(\square)=ka$ divisible by $a$, then the two energy levels of $\textrm{inversion}(\square)$ lie on the same runner.  Similarly, any inversion of energy states on the same runner corresponds to a cell with hook length divisible by $a$.  

Thus, $\lambda$ is an $a$-core if and only if the beads on each runner of the $a$-abacus are right justified, that is, if the $a$-quotient of $\lambda$ is zero.  Moving a bead on the $a$-abacus, corresponds to removing a border strip of length $a$, and so sliding all the beads on each runner all the way to the right corresponds to removing all $a$-hooks in $\lambda$.  We see that the order we make the moves doesn't matter, as the end result will always be the same.

  Although the total charge of all the runners must be zero, the charge need not be evenly divided among the runners.  Let
$c_i$ be the charge on the $i$th runner; then we have $\sum c_i=0$, and the $c_i$ determine $\lambda$.

Similarly, given any $\mathbf{c}=(c_0,\dots,c_{a-1})\in\Z^a$ with $\sum c_i=0$, there is a unique right justified abacus with charge  $c_i$ on the $i$th runner.  The coresponding partition is an $a$-core which we denote $\core_a(\mathbf{c})$.

We have shown:

\begin{lemma}
There is a bijection $$\core_a:\{(c_0,\dots,c_{a-1}|c_i\in\Z, \sum c_i=0\}\to \{\lambda | \lambda \text{ is in $a$-core} \}$$
\end{lemma}


\begin{example}
We illustrate that $\core_3(0,3, -3)=7+5+3+3+2+2+1+1$.  The numbers on the boundary path illustrate which runner of the abacus that step belongs to.

\begin{center}
\begin{tikzpicture}

\begin{scope} %drawing of original partition
\draw (0,5) node{$\lambda=10+8+8+4+4+3+1+1$};


   \begin{scope}[gray, very thin, scale=.4]
      \clip (-11.5, 11.5) rectangle (11.5, -1);
      \draw[rotate=45, scale=1.412, gray!20!white, fill] (0,0)--(0,10)--(1,10)--(1,8)--(3,8)--(3,4)--(5,4)--(5,3)--(6,3)--(6,1)--(8,1)--(8,0)--cycle;
      \begin{scope}[rotate=45, scale=1.412, black!50!green] 
         \draw (.5,6.5)node{$\spadesuit$};
         \draw (1.5,6.5)node{$\spadesuit$};
      \end{scope}

      \begin{scope}[rotate=45, scale=1.412, blue] 
         \draw (2.5,5.5)node{$\clubsuit$};
         \draw (2.5,2.5)node{$\clubsuit$};
         \draw (4.5,2.5)node{$\clubsuit$};
      \end{scope}


      \draw[rotate=45, scale=1.412] (0,0) grid (12,12);
   \end{scope}

   \begin{scope}[rotate=45, ultra thick, scale=.4*1.412]
      \begin{scope}[blue]
         \draw (1,9.5) node{$1$};
         \draw (2.5,8)node{$1$};
         \draw (3,5.5)node{$1$};
         \draw (4.5,4)node{$1$};
         \draw (6,2.5)node{$1$};
         \draw (7.5,1)node{$1$};
         \draw (9.5,0)node{$1$};
      \end{scope}

      \begin{scope}[red]
         \draw (0,10.5)node{$2$};
         \draw (1,8.5)node{$2$};
         \draw (3,7.5)node{$2$};
         \draw (3,4.5)node{$2$};
         \draw (5,3.5)node{$2$};
         \draw (6,1.5)node{$2$};
         \draw (8,.5)node{$2$};
         \draw (10.5,0)node{$2$};
      \end{scope}

      \begin{scope}[black!50!green]
         \draw (.5,10)node{$3$};
         \draw (1.5, 8)node{$3$};
         \draw (3,6.5)node{$3$};
         \draw (3.5,4)node{$3$};
         \draw (5.5,3)node{$3$};
         \draw (6.5,1)node{$3$};
         \draw (8.5,0)node{$3$};
      \end{scope}

   \end{scope}

   \begin{scope}[scale=.4, yshift=-.5cm, blue]
      \draw(-11.5,0) circle(.3);
      \draw(-8.5, 0)  circle (.3);
      \filldraw(-5.5,0)  circle (.3);
      \draw(-2.5,0)  circle (.3);
      \filldraw(.5,0)  circle (.3);
      \draw(3.5,0)  circle (.3);
      \filldraw(6.5,0)  circle (.3);
      \filldraw(9.5,0)  circle (.3);
   \end{scope}

   \begin{scope}[scale=.4, yshift=-1.5cm, red]
      \draw(-10.5,0) circle(.3);
      \draw(-7.5, 0)  circle (.3);
      \draw(-4.5,0)  circle (.3);
      \draw(-1.5,0)  circle (.3);
      \draw(1.5,0)  circle (.3);
      \draw(4.5,0)  circle (.3);
      \draw(7.5,0)  circle (.3);
      \filldraw(10.5,0)  circle (.3);
   \end{scope}

   \begin{scope}[scale=.4, yshift=-2.5cm, black!50!green]
      \filldraw(-9.5,0) circle(.3);
      \filldraw(-6.5, 0)  circle (.3);
      \draw(-3.5,0)  circle (.3);
      \filldraw(-.5,0)  circle (.3);
      \filldraw(2.5,0)  circle (.3);
      \filldraw(5.5,0)  circle (.3);
      \filldraw(8.5,0)  circle (.3);
      \filldraw(11.5,0)  circle (.3);
   \end{scope}

   \begin{scope}[scale=.4]
      \draw (0,-3)--(0,0);
   \end{scope}
\end{scope}

\begin{scope}[yshift=-8cm] %drawing of the core
\draw (0,5) node{$\core_3(\lambda)=7+5+3+3+2+2+1+1$};


   \begin{scope}[gray, very thin, scale=.4]
      \clip (-11.5, 11.5) rectangle (11.5, -1);
      \draw[rotate=45, scale=1.412, gray!20!white, fill] (0,0)--(0,7)--(1,7)--(1,5)--(2,5)--(2,3)--(4,3)--(4,2)--(6,2)--(6,1)--(8,1)--(8,0)--cycle;
      \draw[rotate=45, scale=1.412] (0,0) grid (12,12);
   \end{scope}

   \begin{scope}[rotate=45, ultra thick, scale=.4*1.412]
      \begin{scope}[blue]
         \draw (0,8.5) node{$1$};
         \draw (1,6.5)node{$1$};
         \draw (2,4.5)node{$1$};
         \draw (3.5,3)node{$1$};
         \draw (5.5,2)node{$1$};
         \draw (7.5,1)node{$1$};
         \draw (9.5,0)node{$1$};
      \end{scope}

      \begin{scope}[red]
         \draw (0,10.5)node{$2$};
         \draw (0,7.5)node{$2$};
         \draw (1,5.5)node{$2$};
         \draw (2,3.5)node{$2$};
         \draw (4,2.5)node{$2$};
         \draw (6,1.5)node{$2$};
         \draw (8,.5)node{$2$};
         \draw (10.5,0)node{$2$};
      \end{scope}

      \begin{scope}[black!50!green]
         \draw (0,9.5)node{$3$};
         \draw (0.5, 7)node{$3$};
         \draw (0.5,7)node{$3$};
         \draw (1.5,5)node{$3$};
         \draw (2.5,3)node{$3$};
         \draw (4.5,2)node{$3$};
         \draw (6.5,1)node{$3$};
         \draw (8.5,0)node{$3$};
      \end{scope}

   \end{scope}

   \begin{scope}[scale=.4, yshift=-.5cm, blue]
      \draw(-11.5,0) circle(.3);
      \draw(-8.5, 0)  circle (.3);
      \draw(-5.5,0)  circle (.3);
      \draw(-2.5,0)  circle (.3);
      \filldraw(.5,0)  circle (.3);
      \filldraw(3.5,0)  circle (.3);
      \filldraw(6.5,0)  circle (.3);
      \filldraw(9.5,0)  circle (.3);
   \end{scope}

   \begin{scope}[scale=.4, yshift=-1.5cm, red]
      \draw(-10.5,0) circle(.3);
      \draw(-7.5, 0)  circle (.3);
      \draw(-4.5,0)  circle (.3);
      \draw(-1.5,0)  circle (.3);
      \draw(1.5,0)  circle (.3);
      \draw(4.5,0)  circle (.3);
      \draw(7.5,0)  circle (.3);
      \filldraw(10.5,0)  circle (.3);
   \end{scope}

   \begin{scope}[scale=.4, yshift=-2.5cm, black!50!green]
      \draw(-9.5,0) circle(.3);
      \filldraw(-6.5, 0)  circle (.3);
      \filldraw(-3.5,0)  circle (.3);
      \filldraw(-.5,0)  circle (.3);
      \filldraw(2.5,0)  circle (.3);
      \filldraw(5.5,0)  circle (.3);
      \filldraw(8.5,0)  circle (.3);
      \filldraw(11.5,0)  circle (.3);
   \end{scope}

   \begin{scope}[scale=.4]
      \draw (0,-3)--(0,0);
   \end{scope}

\end{scope}


\begin{scope}[yshift=-12cm] %the quotients

   \begin{scope}[xshift=-4cm] %blue quotient


\draw (0,2) node{$\quot_3^1(\lambda)=2+1$};
      \begin{scope}[gray, very thin, scale=.4]
         \clip (-5, 3.5) rectangle (5, -1);
         \draw[rotate=45, scale=1.412, gray!20!white, fill] (0,0)--(0,2)--(1,2)--(1,1)--(2,1)--(2,0)--cycle;
         \draw[rotate=45, scale=1.412] (0,0) grid (4,4);
      \end{scope}

      \begin{scope}[scale=.4, yshift=-.5cm, blue]
         \draw(-3.5,0) circle(.3);
         \draw(-2.5, 0)  circle (.3);
         \filldraw(-1.5,0)  circle (.3);
         \draw(-.5,0)  circle (.3);
         \filldraw(.5,0)  circle (.3);
         \draw(1.5,0)  circle (.3);
         \filldraw(2.5,0)  circle (.3);
         \filldraw(3.5,0)  circle (.3);
      \end{scope}

   \end{scope}



   \begin{scope} %red quotient
\draw (0,2) node{$\quot_3^2(\lambda)=\emptyset$};
      \begin{scope}[gray, very thin, scale=.4]
         \clip (-5, 3.5) rectangle (5, -1);
         \draw[rotate=45, scale=1.412] (0,0) grid (4,4);
      \end{scope}

      \begin{scope}[scale=.4, yshift=-.5cm, red]
         \draw(-3.5,0) circle(.3);
         \draw(-2.5, 0)  circle (.3);
         \draw(-1.5,0)  circle (.3);
         \draw(-.5,0)  circle (.3);
         \filldraw(.5,0)  circle (.3);
         \filldraw(1.5,0)  circle (.3);
         \filldraw(2.5,0)  circle (.3);
         \filldraw(3.5,0)  circle (.3);
      \end{scope}

   \end{scope}


   \begin{scope}[xshift=4cm] %green quotient 
\draw (0,2) node{$\quot_3^3(\lambda)=1+1$};

      \begin{scope}[gray, very thin, scale=.4]
         \clip (-5, 3.5) rectangle (5, -1);
         \draw[rotate=45, scale=1.412, gray!20!white, fill] (0,0)--(0,1)--(2,1)--(2,0)--cycle;
         \draw[rotate=45, scale=1.412] (0,0) grid (4,4);
      \end{scope}

      \begin{scope}[scale=.4, yshift=-.5cm, black!50!green]
         \draw(-3.5,0) circle(.3);
         \draw(-2.5, 0)  circle (.3);
         \draw(-1.5,0)  circle (.3);
         \filldraw(-.5,0)  circle (.3);
         \filldraw(.5,0)  circle (.3);
         \draw(1.5,0)  circle (.3);
         \filldraw(2.5,0)  circle (.3);
         \filldraw(3.5,0)  circle (.3);
      \end{scope}

   \end{scope}

\end{scope}


\end{tikzpicture}
\end{center}
\end{example}



\subsection{Leaving and arriving word description}

We now describe a slight shift in perspective on cores that will help when we generalize them.  One way to label the colors of the paths is in terms of the contents of the cells they border.  



\subsection{Size of an \texorpdfstring{$a$}{a}-core}
The core and quotient bijection above in particular gives a bijection between $a$-cores and points in a lattice $a-1$ dimensional lattice.  We have parametrized the lattice above in the \emph{charge} coordinates.  


\begin{definition}
The \emph{$r$-charge} of a partition $\lambda$ is the $c^G(\lambda)$ vector in $\Lambda_a=\{(c_i)|\sum c_i=0\}$ where $c_i^G(\lambda)$ is the charge of the $i$th runner of $\lambda$ when written on the $a$-abacus.
\end{definition}




\begin{theorem} \label{thm:quadratic-sl2}
The size of a core is a quadratic function in the charge coordinates;
$$|\core_a(\mathbf{c})|=\frac{a}{2}\sum_{k=0}^{a-1} c_k^2+ kc_k$$
In fact, each colored cell count is a quadratic function in the charge coordinates.
$$|\core_a(\mathbf{c})|^G_i=\frac{1}{2}\sum_{k=0}^{a-1} c_k^2+ \sum_{k>i}c_k$$
\end{theorem}

We are not sure where Theorem \ref{thm:quadratic-sl2} originates; it is proved independently in \cite{GKS, DS}.  The proof is not difficult and uses the interaction of Frobenius coordinates and the abacus, see ; the first statement is proved in \cite{jsimultaneous} and is easily extended to the second statement. 


Note that since the quadratic part of $|\core_a(\mathbf{c})|^G_i$ is independent of $i$, the quadratic part is adding a number of copies of the regular representation.  Hence, Theorem \ref{thm:quadratic-sl2} implies:




\begin{corollary} 
There is a linear relation between the charge functions $c_i(\lambda)$ the class $[\lambda]^G$.
\end{corollary}

In fact, an explicit form of this linear change of variables will hold in general, even though



 that there is a linear change of variables between the charge coordinates and the colored box coordinates.  This second fact we will prove this second fact directly in Theorem \ref{ }



\section{Chen-Ruan cohomology and the disconnected conjecture}


\subsection{Conjectural product formula}

To state our conjectural formula, we will use the Pochhammer symbol $(a;x)_\infty:=\prod_{\ell\geq 0} (1-ax^\ell)$.
\begin{example}
Using the Pochhammer symbol, G\"ottsche's formula becomes:
$$\sum_{n\geq 0} b_k(\Hilb_n(S))t^kq^n=\frac{(-qt;qt^2)_\infty^{b_1}(-qt^3;qt^2)_\infty^{b_3}}{(q;qt^2)_\infty^{b_0}}\frac{1}{(qt^2;qt^2)_\infty^{b_2}}\frac{1}{(qt^4;qt^2)_\infty^{b_4}}$$
\end{example}

In \cite{GLMequivariant}, we have the following conjectural product formula 
{Conjecture (Gusein-Zade, Luengo, Melle-Hern\'andez)}

$$\DC_{1/3}=\frac{1}{(q;t^2q^3)_\infty}\frac{1}{(q^2t^2;t^2q^3)_\infty}\frac{1}{(q^3;t^2q^3)_\infty}$$

Setting $t=1$ should just give the partition function; indeed, we see that this is just the Euler product expansion with some factors of $t$ inserted, with the power of $t$ multiplying $q^n$ depends on $n$ mod 3:

$$\DC_{1/3}=\frac{1}{(1-q)}\frac{1}{(1-qt^2)}\frac{1}{(1-q^3)}\frac{1}{(1-q^4t^2)}
\frac{1}{(1-q^5t^4)}\frac{1}{(1-q^6t^2)}\cdots$$

 It seems in general that if $G\cap SL_2=\{1\}$ then
$$\DC_{G}=\prod_{h=1}^r \frac{1}{(q^h t^{\epsilon(h)}; q^r t^2)_\infty}$$
with $\epsilon(h)$ either 2 or 0; the question is then to describe $\epsilon(h)$.

In G\"ottsche's formula, $\epsilon(h)=0$ corresponds to $b_0$, and $\epsilon(h)=2$ corresponds to $b_2$, and so we might hope for a description of the $\epsilon(h)$ in terms of the cohomology of the stack $[\C^2/G]$.  That is exactly what we propose, using Chen-Ruan cohomology.



\subsection{Chen-Ruan cohomology}

The Chen-Ruan cohomology $H_{CR}^*(\mathcal{X})$ of an orbifold $\mathcal{X}$ was discovered as a byproduct of defining the quantum cohomology of such orbifolds \cite{chenruancohomology}.  As a vector space, the Chen-Ruan cohomology is the usual cohomology of the inertial orbifold $\mathcal{IX}$ of $\mathcal{X}$.  As a set $\mathcal{IX}$ is the space of constant maps from $S^1$ to $\mathcal{X}$; more algebraically, 
$$\mathcal{IX}=\{(x, (g)| x\in \mathcal{X}, (g)\in\textrm{conj}(G_x)\}$$
Every isotropy group $G_x$ has an identity element $e_x$, the subset of $(x, e_x)$ naturally forms a copy of $X$.  Elements $(x,(g))$ with $(g)$ nontrivial form other components of $\mathcal{IX}$ called \emph{twisted sectors}.

Thus, for $\mathcal{X}=[\C^2/G]$, we see that 
$$H_{CR}^*(\mathcal{X}) =H^*(\mathcal{IX})=\oplus_{g\in\textrm{conj}(G)} (\C^2)^g$$
has a basis indexed by conjugacy classes of $G$.

However, Chen-Ruan cohomology has a different product and grading than $H^*(\mathcal{IX})$.  Each twisted sector has a a degree shifting number $\iota(g)$, obtained as follows.

Since $g\in G_x$, and $g$ acts on $T_x$.  It acts trivially on the tangent directions to fix($g$), and nontrivial on the normal directions.  Diagonalizing the action on the normal bundle, we see that $g$ is a diagonal matrix with entries 
$$(\exp(2\pi i a_1/r), \exp(2\pi i a_2/r), \cdots \exp(2\pi i a_m))$$
Then $\iota(g)=\sum a_i/r$; sometimes called the logarithmic trace of $g$.

The grading shift number is in general only a rational number -- $\iota(g)\in \Z$ if and only if the determinant of $g$ is trivial.  Thus, in case all isotropy groups $G_x$ are in $SL_2$, then the Chen-Ruan cohomology is integrally graded.

\begin{example}[Antidiagonal action]
When $G\subset SL_2$, we have that 
$$\iota(g)=\left\{\begin{array}{rl} 0 & g=0 \\
1 & g\neq 0 \end{array}\right.
$$

Thus, the dimension of $H_{CR}^*([\C^2/G])$ is equal to the number of conjugacy classes of $G$ (which is the number of irreps of $G$.  We have $H_{CR}^0$ is one dimensional, with the rest of the classes being two dimensional.  Thus we have

$$H_{CR}^*([\C^2/G])\cong H^*(\widetilde{S}_G)$$
as graded vector spaces (actually, as rings in this case).  
\end{example}


\begin{theorem}[Yasuda \cite{yasuda}] \label{thm:yasuda}
Let $\mathcal{X}$ an effective orbifold, and $\widetilde{X}\to|\mathcal{X}|$ a crepant resolution of the coarse moduli space.  Then :
$$H_{CR}^*(\mathcal{X})=H^*(\widetilde{X})$$
as graded vector spaces.  
\end{theorem}

Note that the products do not necessarily agree; quantumn corrections are needed; the \emph{Crepant Resolution Conjecture} states that, properly understood, the quantumn cohomology of $\mathcal{X}$ and $\widetilde{X}$ should agree.  See \cite{CoatesRuan} for discussion of the details and reference to other sources.


\begin{example}
Let $\Z_r$ act diagonally, with element $g_k$ acting as $(\exp(k2\pi i/r),\exp(k2\pi i/r))$.  Then we have $\iota(g_k)=2k/r$.
\end{example}




The Chen-Ruan cohomology of $[\C^2/G]$ is rationally graded, with $d$ with $0\leq d < 4$


$$0\to G\cap SL_2\to G \to \C^*$$








Our conjectural product formula for $\DC_G$ is easiest to state in case $G\cap SL_2=1$.  In this case, the action of $G$ on $\wedge T^*\C^2$ is faithful; taking $r$ times the logarithmic trace of this action gives a bijection between $G$ and $\{0,\dots, r-1\}$.


Let $F(g)$ and $I(g)$ denote the fractional and integral parts of $\iota(g)$. 

If $G\cap SL_2=\{1\}$, then $F(G)$ gives a bijection between $G$ and $\{0, 1/r,\dots, (r-1)/r\}$.
\begin{conjecture}[Johnson]
Let $G$ be cyclic, and define $k=|G\cap SL_2|$

$$\mathcal{H}_G(q,t)= \frac{(q^k;q^k)^k_\infty}{(q,q)_\infty} \prod_{g\in G}\frac{1}{(q^{r(1-F(g))} t^{2I(g)},q^rt^2)_\infty}$$

\end{conjecture}


\begin{example}[Diagonal action, $r$-odd]
Let $r=2k+1$.   The element of $\Z_r$ that acts on $K$ as $\exp(2\pi i/r)$ acts 1on the tangent space as $\exp(2\pi i k/r)$, and thus as $\iota=2k/(2k+1)<1$.  

Odd powers of this element will have $\iota<1$, while even powers will have $\iota>1$, giving 

In the limit as $r\to\infty$ odd, we get Buryak-Feigin's result

$$\prod_{n \text{ odd}} \frac{1}{1-q^n}\prod_{m \text{ even}}\frac{1}{1-tq^m}$$
\end{example}

In the examples that follow, we will use the following extension of the Pochhammer symbol:

$$(a_1,\dots, a_n; b)_\infty=\prod_{i=1}^n (a_i;b)_\infty$$

\begin{example}[$S_{2/5}$]
We have
$$\begin{array}{r|c|l}
g & r(1-F(g)) & I(g) \\
\hline
1 & 2 & 0 \\
2 & 4 & 1 \\
3 & 1 & 0 \\
4 & 3 & 1 \\
\end{array}
$$

Thus, we have


$$
\frac{1}{(q, q^2,q^3t, q^4t, q^5t^2; q^5t^2)_\infty}
$$
\end{example}


Appears Buryak-Feigin plays well with colored box counting from cores.



\section{Generalized cores and applications}

This section introduces the generalization of cores and quotients, with the goal of showing that it is a useful tool of studying $|\lambda|^G$ and $\cdim(\lambda)$ introduced.

In particular, we use the $G$-cores and quotients to prove two results: that the topology of $\Hilb_v$ stabilizes, and that the size and dimensions of the small components are piecewise quadratic functions.

\begin{lemma}
  The map class $[\lambda]^G\in\overline{K}(G)$ depends only on $c^G(\lambda)$.  
More specifically, let $e_i$ be the vector that adds 1 to the $i$th charge, $w_i=e_{i+a}-e_i\in\Lambda^G$ and $f_i\in\overline{K}(G)$ the vector that adds 1 to the $i$th color, and define an isomorphism of lattices $\varphi:\Lambda^G\to\overline{K}(G)$ by $\varphi(w_i)=f_i$.

Then $[\lambda]^G=\varphi(c^G(\lambda))$.
\end{lemma}

\begin{proof}
First, note $\varphi$ is an isomorphism of lattices, as the $w_i$ and $f_i$ generate each lattice subject to the relation $\sum w_i=0, \sum f_i=0$.

It suffices to show that if $\mu$ is obtained from $\lambda$ by removing a single box $\square$, then $[\lambda]^G-[\mu]^G=\varphi(c^G(\lambda)-c^G(\mu)$.  Suppose that $\square$ had color $i$; then in $\lambda$ the border path travels above $\square$ and has color $i$, while in $\mu$ the border path travels below $\square$ and has color $i+a$.  All other up-sloping steps of $\lambda$ and $\mu$ agree.  In removing the box, we deleted a box of color $i$ from $\lambda$, and thus $[\lambda]^G-[\mu]^G=f_i$.  We also removed one bead from the $i$th runner, increasing $c^G_i$ by 1, and added a bead to the $i+a$ runner, decreasing $c^G_{i+a}$ by 1.




\end{proof}


\subsection{ }

The construction of $G$-cores and quotients is exactly parallel to the 


\begin{center}
\begin{tikzpicture}

\begin{scope} %drawing of original partition
\draw (0,5) node{$\lambda=$};
\begin{scope}[scale=.4]

\begin{scope}[rotate=45, scale=1.412]
\clip (0,0)--(13,0)--(0,13)--cycle;
\foreach \x in {0,...,13}{%
    \foreach \y in {0,...,4}{%
       \draw[fill=green!7]  (\x, -\x+3*\y) rectangle (\x+1, -\x+3*\y+1); 
       \draw[fill=blue!7]  (\x, -\x+3*\y+1) rectangle (\x+1, -\x+3*\y+2); 
       \draw[fill=red!7]  (\x, -\x+3*\y+2) rectangle (\x+1, -\x+3*\y+3); 
}}

\begin{scope}
    \clip (0,0)--(0,10)--(1,10)--(1,9)--(3,9)--(3,5)--(4,5)--(4,3)--(6,3)--(6,2)--(7,2)--(7,1)--(8,1)--(8,0)--cycle;
\foreach \x in {0,...,13}{%
    \foreach \y in {0,...,4}{%
       \draw[fill=green!23]  (\x, -\x+3*\y) rectangle (\x+1, -\x+3*\y+1); 
       \draw[fill=blue!23]  (\x, -\x+3*\y+1) rectangle (\x+1, -\x+3*\y+2); 
       \draw[fill=red!23]  (\x, -\x+3*\y+2) rectangle (\x+1, -\x+3*\y+3); 
}}
\end{scope}


      \draw[gray, very thin] (0,0) grid (13,13);
\end{scope}


      \begin{scope}[rotate=45, scale=1.412, black!50!green] 
         \draw (.5,6.5)node{$\spadesuit$};
         \draw (1.5,6.5)node{$\spadesuit$};
      \end{scope}

      \begin{scope}[rotate=45, scale=1.412, blue] 
         \draw (2.5,7.5)node{$\clubsuit$};
         \draw (2.5,3.5)node{$\clubsuit$};
         \draw (3.5,3.5)node{$\clubsuit$};
      \end{scope}



   \end{scope}

   \begin{scope}[rotate=45, ultra thick, scale=.4*1.412]
      \begin{scope}[blue]
         \draw (0,10.5) node{$1$};
         \draw (1,9.5)node{$1$};
         \draw (2.5,9)node{$1$};
         \draw (3,7.5)node{$1$};
         \draw (3.5,5)node{$1$};
         \draw (4,3.5)node{$1$};
         \draw (5.5,3)node{$1$};
         \draw (6.5,2)node{$1$};
         \draw (7.5,1)node{$1$};
         \draw (8.5,0)node{$1$};
         \draw (11.5,0)node{$1$};
      \end{scope}

      \begin{scope}[red]
         \draw (0,11.5)node{$2$};
         \draw (3,8.5)node{$2$};
         \draw (3,5.5)node{$2$};
         \draw (4,4.5)node{$2$};
         \draw (6,2.5)node{$2$};
         \draw (7,1.5)node{$2$};
         \draw (8,.5)node{$2$};
         \draw (9.5,0)node{$2$};
      \end{scope}

      \begin{scope}[black!50!green]
         \draw (.5,10)node{$3$};
         \draw (1.5, 9)node{$3$};
         \draw (3,6.5)node{$3$};
         \draw (4.5,3)node{$3$};
         \draw (10.5,0)node{$3$};

      \end{scope}

   \end{scope}

   \begin{scope}[scale=.4, yshift=-.5cm, blue]
      \draw(-10.5,0) circle(.3);
      \draw(-8.5, 0)  circle (.3);
      \filldraw(-6.5,0)  circle (.3);
      \draw(-4.5,0)  circle (.3);
      \filldraw(-1.5,0)  circle (.3);
      \draw(.5,0)  circle (.3);
      \filldraw(2.5,0)  circle (.3);
      \filldraw(4.5,0)  circle (.3);
      \filldraw(6.5,0)  circle (.3);
      \filldraw(8.5,0)  circle (.3);
      \filldraw(11.5,0)  circle (.3);
   \end{scope}

   \begin{scope}[scale=.4, yshift=-1.5cm, red]
      \draw(-11.5,0) circle(.3);
      \draw(-5.5, 0)  circle (.3);
      \draw(-2.5,0)  circle (.3);
      \draw(-.5,0)  circle (.3);
      \draw(3.5,0)  circle (.3);
      \draw(5.5,0)  circle (.3);
      \draw(7.5,0)  circle (.3);
      \filldraw(9.5,0)  circle (.3);
      \filldraw(12.5,0)  circle (.3);
   \end{scope}

   \begin{scope}[scale=.4, yshift=-2.5cm, black!50!green]
      \draw(-12.5,0)  circle (.3);
      \filldraw(-9.5,0) circle(.3);
      \filldraw(-7.5, 0)  circle (.3);
      \draw(-3.5,0)  circle (.3);
      \filldraw(1.5,0)  circle (.3);
      \filldraw(10.5,0)  circle (.3);
   \end{scope}

   \begin{scope}[scale=.4]
      \draw (0,-3)--(0,0);
   \end{scope}
\end{scope}

\begin{scope}[yshift=-8cm] %drawing of the core
\draw (0,5) node{$\core_3(\lambda)=7+5+3+3+2+2+1+1$};



\begin{scope}[rotate=45, scale=.4*1.412]

\clip (0,0)--(13,0)--(0,13)--cycle;
\foreach \x in {0,...,13}{%
    \foreach \y in {0,...,4}{%
       \draw[fill=green!7]  (\x, -\x+3*\y) rectangle (\x+1, -\x+3*\y+1); 
       \draw[fill=blue!7]  (\x, -\x+3*\y+1) rectangle (\x+1, -\x+3*\y+2); 
       \draw[fill=red!7]  (\x, -\x+3*\y+2) rectangle (\x+1, -\x+3*\y+3); 
}}

\begin{scope}
    \clip (0,0)--(0,7)--(1,7)--(1,6)--(2,6)--(2,5)--(4,5)--(4,4)--(5,4)--(5,3)--(6,3)--(6,2)--(7,2)--(7,1)--(8,1)--(8,0)--cycle;
\foreach \x in {0,...,13}{%
    \foreach \y in {0,...,4}{%
       \draw[fill=green!23]  (\x, -\x+3*\y) rectangle (\x+1, -\x+3*\y+1); 
       \draw[fill=blue!23]  (\x, -\x+3*\y+1) rectangle (\x+1, -\x+3*\y+2); 
       \draw[fill=red!23]  (\x, -\x+3*\y+2) rectangle (\x+1, -\x+3*\y+3); 
}}
\end{scope}


      \draw[gray, very thin] (0,0) grid (13,13);
\end{scope}


   \begin{scope}[rotate=45, ultra thick, scale=.4*1.412]
      \begin{scope}[blue]
         \draw (0,10.5) node{$1$};
         \draw (0,7.5)node{$1$};
         \draw (1,6.5)node{$1$};
         \draw (2,5.5)node{$1$};
         \draw (3.5,5)node{$1$};
         \draw (4.5,4)node{$1$};
         \draw (5.5,3)node{$1$};
         \draw (6.5,2)node{$1$};
         \filldraw (7.5,1)node{$1$};
         \filldraw (8.5,0)node{$1$};
         \filldraw (11.5,0)node{$1$};
      \end{scope}

      \begin{scope}[red]
         \draw (0,11.5)node{$2$};
         \draw (0,8.5)node{$2$};
         \draw (4, 4.5)node{$2$};
         \draw (5, 3.5)node{$2$};
         \draw (6, 2.5)node{$2$};
         \draw (7, 1.5)node{$2$};
         \draw (8,.5)node{$2$};
         \draw (9.5,0)node{$2$};
      \end{scope}

      \begin{scope}[black!50!green]
         \draw (0,9.5)node{$3$};
         \draw (0.5, 7)node{$3$};
         \draw (1.5,6)node{$3$};
         \draw (2.5,5)node{$3$};
         \draw (10.5,0)node{$3$};
      \end{scope}

   \end{scope}

   \begin{scope}[scale=.4, yshift=-.5cm, blue]
      \draw(-10.5,0) circle(.3);
      \draw(-7.5, 0)  circle (.3);
      \draw(-5.5,0)  circle (.3);
      \draw(-3.5,0)  circle (.3);
      \filldraw(-1.5,0)  circle (.3);
      \filldraw(.5,0)  circle (.3);
      \filldraw(2.5,0)  circle (.3);
      \filldraw(4.5,0)  circle (.3);
      \filldraw(6.5,0)  circle (.3);
      \filldraw(8.5,0)  circle (.3);
      \filldraw(11.5,0)  circle (.3);
   \end{scope}

   \begin{scope}[scale=.4, yshift=-1.5cm, red]
      \draw(-11.5,0) circle(.3);
      \draw(-8.5, 0)  circle (.3);
      \draw(-.5,0)  circle (.3);
      \draw(1.5,0)  circle (.3);
      \draw(3.5,0)  circle (.3);
      \draw(5.5,0)  circle (.3);
      \draw(7.5,0)  circle (.3);
      \filldraw(9.5,0)  circle (.3);
      \filldraw(12.5,0)  circle (.3);
   \end{scope}

   \begin{scope}[scale=.4, yshift=-2.5cm, black!50!green]
          \draw(-12.5,0) circle(.3);
      \draw(-9.5,0) circle(.3);
      \filldraw(-6.5, 0)  circle (.3);
      \filldraw(-4.5,0)  circle (.3);
      \filldraw(-2.5,0)  circle (.3);
       \filldraw(10.5,0)  circle (.3);
   \end{scope}

   \begin{scope}[scale=.4]
      \draw (0,-3)--(0,0);
   \end{scope}

\end{scope}


\begin{scope}[yshift=-12cm] %the quotients

   \begin{scope}[xshift=-4cm] %blue quotient


\draw (0,2) node{$\quot_3^1(\lambda)=2+1$};
      \begin{scope}[gray, very thin, scale=.4]
         \clip (-5, 3.5) rectangle (5, -1);
         \draw[rotate=45, scale=1.412, gray!20!white, fill] (0,0)--(0,2)--(1,2)--(1,1)--(2,1)--(2,0)--cycle;
         \draw[rotate=45, scale=1.412] (0,0) grid (4,4);
      \end{scope}

      \begin{scope}[scale=.4, yshift=-.5cm, blue]
         \draw(-3.5,0) circle(.3);
         \draw(-2.5, 0)  circle (.3);
         \filldraw(-1.5,0)  circle (.3);
         \draw(-.5,0)  circle (.3);
         \filldraw(.5,0)  circle (.3);
         \draw(1.5,0)  circle (.3);
         \filldraw(2.5,0)  circle (.3);
         \filldraw(3.5,0)  circle (.3);
      \end{scope}

   \end{scope}



   \begin{scope} %red quotient
\draw (0,2) node{$\quot_3^2(\lambda)=\emptyset$};
      \begin{scope}[gray, very thin, scale=.4]
         \clip (-5, 3.5) rectangle (5, -1);
         \draw[rotate=45, scale=1.412] (0,0) grid (4,4);
      \end{scope}

      \begin{scope}[scale=.4, yshift=-.5cm, red]
         \draw(-3.5,0) circle(.3);
         \draw(-2.5, 0)  circle (.3);
         \draw(-1.5,0)  circle (.3);
         \draw(-.5,0)  circle (.3);
         \filldraw(.5,0)  circle (.3);
         \filldraw(1.5,0)  circle (.3);
         \filldraw(2.5,0)  circle (.3);
         \filldraw(3.5,0)  circle (.3);
      \end{scope}

   \end{scope}


   \begin{scope}[xshift=4cm] %green quotient 
\draw (0,2) node{$\quot_3^3(\lambda)=1+1$};

      \begin{scope}[gray, very thin, scale=.4]
         \clip (-5, 3.5) rectangle (5, -1);
         \draw[rotate=45, scale=1.412, gray!20!white, fill] (0,0)--(0,1)--(2,1)--(2,0)--cycle;
         \draw[rotate=45, scale=1.412] (0,0) grid (4,4);
      \end{scope}

      \begin{scope}[scale=.4, yshift=-.5cm, black!50!green]
         \draw(-3.5,0) circle(.3);
         \draw(-2.5, 0)  circle (.3);
         \draw(-1.5,0)  circle (.3);
         \filldraw(-.5,0)  circle (.3);
         \filldraw(.5,0)  circle (.3);
         \draw(1.5,0)  circle (.3);
         \filldraw(2.5,0)  circle (.3);
         \filldraw(3.5,0)  circle (.3);
      \end{scope}

   \end{scope}

\end{scope}


\end{tikzpicture}
\end{center}


\section{Riemann-Roch calculation}

\subsection{Riemann-Roch Theorem}
Although the ext groups $\Ext^i(\mathcal{F},\mathcal{G})$ between two sheaves depend delicately on the sheaves involved, the Euler pairing
$$\chi(\mathcal{F},\mathcal{G})=\sum_{i\geq 0} \Ext^i(\mathcal{F},\mathcal{G})$$
only depends upon the $K$-theory classes of $\mathcal{F}$ and $\mathcal{G}$.  

\begin{example}
Let $p$ and $q$ be two points of $\C^2$.  If $p\neq q$, then the structure sheaves $\mathcal{O}_p$ and $\mathcal{O}_q$ have disjoint support, and so $\Ext^i(\mathcal{O}_p,\mathcal{O}_q)=0$ for all $i$.

If $p=q$ we have:
\begin{eqnarray*}
\Ext^0(\mathcal{O}_p,\mathcal{O}_p)&=&1\\
\Ext^1(\mathcal{O}_p,\mathcal{O}_p)&=&2\\
\Ext^2(\mathcal{O}_p,\mathcal{O}_p)&=&1 
\end{eqnarray*}

In both cases, $\chi(\mathcal{O}_p,\mathcal{O}_q)=0$.
\end{example}

The Riemann-Roch formula gives an evaluation of the Euler pairing strictly in terms of the $K$-theory class.  Since our orbifolds are global quotients we can just use an equivariant Riemann-Roch formula, and since there are isolated fixed points it has a particularly nice form

In particular, for $\C^2/G$, the Euler pairing will be equivalent to 

The exterior algebra of $\C^2$ is a graded $G$ representation, with $\bigwedge^0\C^2$ the trivial representation, and $\bigwedge^2\C^2$ in piece 0, and $\bigwedge^1\C^2=\C^2$ in degree 1.  

\begin{definition}
The super McKay pairing is $Q_{SM}(V,W)=\dim (V\otimes W^\vee\otimes\bigwedge^*\C^2)^G$.
\end{definition}


\begin{theorem}[Equivariant Riemann-Roch]
$$\chi(\mathcal{F},\mathcal{G})=Q_{SM}([\mathcal{F}],[\mathcal{G}])$$
\end{theorem}


\subsection{Application to cores}

We now apply the Riemann-Roch theorem to the dimension and $K$-theory classes of cores.  Specifically, we will address the following question: given a class $v\in \overline{K}(G)$, what can we say about the dimension and $K$-theory class of the core Hilbert scheme in class $v$?

First, note that pairing with $[V_G]$ just counts the dimension, and since the super dimension of $\bigwedge^*\C^2$ is zero, it is in the kernel of the McKay pairing.  Geometrically, this corresponds to the fact that adding the structure sheaf of a smooth point on $[\C^2/G]$ will change the $K^0$ theory class by $V_G$, but the smooth point may be added disjoint from the support of the rest of the sheaf and the sheaf it is being paired with and hence not change any of the ext groups.


Thus, the Euler pairing $\chi$ descends to a pairing on $\overline{K}(G)=K^0(G)/V_G$.  Given a class $v\in \overline{K}(G)$, what is the minimum lift to $\tilde{v}\in K^0(G)$ with $[\tilde{v}]=v$ and $\Hilb_{\tilde{v}}$ nonempty?

As vector spaces, $\overline{K}(G)$ is isomorphic to the subset of $K^0(G)$ with entries summing to zero, but quite as lattices.  Therefore for $G$ abelian we introduce the space 
$$S^0_G=\left\{v\in K^0(G; \Z[1/|G|])\Bigg | \sum_{\chi\in\irreps(G)}(\chi, v)=0\right\}$$
and we have an isomorphism $S^0_G$ and $\overline{K}(G)$.  


\begin{lemma}
Let $k$ be the index of $K_{[\C^2/G]}\in K(G)$.

Then
$$\tilde{v}=v+\frac{1}{2}\left(Q_M(v)+\dim\Hilb_{\tilde{v}} -v_0-v_{k}\right)[V_G]$$
\end{lemma}



\begin{proof}
We calculate $\chi(\tilde{v},\tilde{v})$ in two different ways, once from the definition, and once using the Riemann-Roch theorem.

Let $\OO_{\tilde{v}}$ be any quotient sheaf with the right $K$-theory class.  

We first use
$$\chi(\tilde{v}, \tilde{v})=\dim\Ext^0(\OO_{\tilde{v}},\OO_{\tilde{v}})-\dim\Ext^1(\OO_{\tilde{v}},\OO_{\tilde{v}})+\dim\Ext^2(\OO_{\tilde{v}},\OO_{\tilde{v}})$$

By definition $\Ext^0(\OO_{\tilde{v}},\OO_{\tilde{v}})=\Hom_R(R/\mathcal{I},R/\mathcal{I})^G$.  Now since $\OO_v$ is a quotient of $R$ it is generated by $1$, and so to define a homomorphism we just need to say where $1$ maps. Since $\mathcal{I}$ is an ideal, it can map to any entry of $R/\mathcal{I}$.  Since one is fixed by $G$, for the homomorphism to be $G$ invariant we must preserve weight and map to something invariant.  So $\dim\Ext^0(\OO_{\tilde{v}},\OO_{\tilde{v}})=\tilde{v}_0$.  

Using Serre-duality, $\dim\Ext^2(\OO_{\tilde{v}},\OO_{\tilde{v}})=\dim\Ext^0(\OO_{\tilde{v}},K\otimes\OO_{\tilde{v}})$.  The effect of tensoring by $K$ is just to change the $G$ action -- an invariant vector of $\OO_{\tilde{v}}\otimes K$ is just a vector that transforms as $K^\vee$ in $\OO_{\tilde{v}}$.  
So $\dim\Ext^2(\OO_{\tilde{v}},\OO_{\tilde{v}})=\tilde{v}_{-k}$.  

Finally $\dim\Ext^1(\OO_{\tilde{v}},\OO_{\tilde{v}})=\dim\Hilb_{\tilde{v}}$.  So $\chi(\tilde{v},\tilde{v})=\tilde{v}_0+\tilde{v}_{-k}-\dim\Hilb_{\tilde{v}}$.  If we write $\tilde{v}=v+M [V_G]$, then $\tilde{v}_i=v_i+M$, and so  
\begin{equation} \label{eq:EXT-evaluation}
\chi(\tilde{v},\tilde{v})=v_o+v_{-k}+2M-\dim\Hilb_{\tilde{v}} 
\end{equation}

On the other hand, using Euler pairing using Riemann-Roch works exactly the same way, and we have
\begin{equation} \label{eq:RR-evaluation}
\chi(\tilde{v},\tilde{v})=Q_{SM}(v+M[V_G],v+M[V_G])=Q_M(v,v)
\end{equation}
using the fact that $[V_G]$ is in the kernel of $Q_{SM}$.  

Setting Equations \ref{eq:EXT-evaluation} and \ref{eq:RR-evaluation} equal and solving for $M$ gives the desired result.

\end{proof}


\begin{corollary}
The map $v\mapsto \dim\Hilb_{\tilde{v}}$ is a piecewise quadratic function.
\end{corollary}

\begin{proof}
Using cores, we have already seen that $\tilde{v}$ is a piecewise quadratic function of $v$.  
\end{proof}


\begin{corollary}  If $G\in SL_2$. Then
$$\tilde{v}=v-v_0+\frac{1}{2}Q_M(v,v)[V_G]$$
In particular, we see $\tilde{v}$ is a quadratic function of $v$.
\end{corollary}

\begin{proof}
When $G\in SL_2$, we the canonical bundle $K_{[\C^2/G]}$ is trivial, and so $k=0$.  Furthermore, when $G\in SL_2$, the core partitions are isolated points and hence have dimension 0.
\end{proof}






\section{Toward bijections: Generalized Dyson maps} \label{sec:dyson}

In this section we point to some partial Our main product formula 

In the case of diagonal $\C^*$ action, the $t=0$ coefficient of our general product formula reduces to Euler's theorem that the number of partitions of $n$ into distinct parts is the number of $n$ into odd parts.  Buryak, Feigin and Nakajima's result then gives a $t$ extension of this to all partitions: the number of partitions of $n$ with $d_{(1,1)}=k$ is the number of partitions of $n$ with $k$ even parts.

There are at least three bijective proofs of Euler's identity, see for instance \cite{pak}: Sylvester's bijection, Glaisher's bijection, and the iterated Dyson map.  This last bijection appears to be the one pertinent here.

In the next subsection we briefly recall this bijection, then in 
\subsection{Iterated Dyson map} \label{sec:review-dyson}


\subsection{Asymptotic behavior} \label{sec-asymptotic-dyson}

\subsection{Application to finite \texorpdfstring{$G$}{G}} 


\subsection{Generalizations} \label{sec:generalized-dyson}




\bibliographystyle{plain}
\bibliography{GHilb}

\end{document}
