\documentclass{amsart}[12pt]
%\pdfoutput=1



\usepackage{tikz}
\usetikzlibrary{arrows, snakes} %for arrowtips  
\usepackage[sc]{mathpazo}
\usepackage{hyperref}
\usepackage{microtype}
\usepackage{tikz}
\usepackage{graphicx}
%\usepackage{mathtools} %for \mathclap, fixing long subscripts in a summation
\linespread{1.2}
\usepackage{xcolor}
\hypersetup{
    colorlinks,
    linkcolor={red!80!black},
    citecolor={blue!80!black},
    urlcolor={blue!80!black}
}


\theoremstyle{definition}

\newtheorem*{theorem*}{Theorem}
\newtheorem{dummy}{}[section]
\newtheorem{theorem}[dummy]{Theorem}
\newtheorem{lemma}[dummy]{Lemma}
\newtheorem{example}[dummy]{Example}
\newtheorem{corollary}[dummy]{Corollary}
\newtheorem{definition}[dummy]{Definition}
\newtheorem{remark}[dummy]{Remark}
\newtheorem{proposition}[dummy]{Proposition}
\newtheorem{observation}{Observation}
\newtheorem{question}{Question}
\newtheorem{conjecture}[dummy]{Conjecture}
\newtheorem{warning}[dummy]{Warning}

\newcommand{\Z}{\mathbf{Z}}
\newcommand{\C}{\mathbf{C}}
\newcommand{\N}{\mathbf{N}}
\newcommand{\R}{\mathbf{R}}
\newcommand{\Q}{\mathbf{Q}}
\newcommand{\A}{\mathbf{A}}
\newcommand{\OO}{\mathcal{O}}
\newcommand{\LL}{\mathbf{L}}
\newcommand{\PP}{\mathcal{P}} % partitions
\newcommand{\II}{\mathcal{I}}

\newcommand{\coleg}{\text{co}\ell}


\newcommand{\proj}{\mathbf{P}}
\newcommand{\core}{\mathbf{core}}
\newcommand{\quot}{\mathbf{quot}}
\newcommand{\irreps}{\text{irreps}}
\newcommand{\Var}{\mathbf{Var}}
\newcommand{\Top}{\mathbf{Top}}
\newcommand{\Sur}{\mathcal{S}}


\DeclareMathOperator{\dusty}{pdim}
\DeclareMathOperator{\Hilb}{Hilb}
\DeclareMathOperator{\Ext}{Ext}
\DeclareMathOperator{\Hom}{Hom}
\DeclareMathOperator{\DC}{DH}
\DeclareMathOperator{\Sym}{Sym}
\DeclareMathOperator{\Ker}{Ker}
\DeclareMathOperator{\cdim}{cdim}


\newcommand{\HG}{\Hilb}
\newcommand{\Gcomb}{\mathcal{GC}}

\begin{document}



\begin{abstract}
We study the connection between the combinatoric of certain partition statistics and the topology of the Hilbert schemes of certain orbifold surfaces.
\end{abstract}


\title{Orbifold Hilbert schemes and a generalization of cores and quotients}


\author{Paul Johnson}
\address{University of Sheffield}
\email{paul.johnson@sheffield.ac.uk}


\maketitle
\setcounter{tocdepth}{1}
\tableofcontents
This paper is written for two largely distinct audiences: algebraic geometers interested in Hilbert schemes of points and orbifold surfaces, and combinatorialists studying partitions -- particularly the construction of cores and quotients.  As such, there is more expository material than perhaps is standard.  In particular, we have two introductions, one for geometers, and one for combinatorialists.


\section{Introduction for Geometers}
From the geometric point of view, this paper studies the topology of Hilbert schemes of points on $\C^2/G$, where $G$ is an abelian group.  

Our motivation is the structure found in the topology of Hilbert schemes of point on a smooth surface $S$.G\"ottsche, building on the work of Ellingsrud and Str\o me for $\C^2$, found product formulas for generating functions of their cohomology, which implied a stabilization result on their homology.  Later, Grojnowski and Nakajima \cite{grojnowski, nakajimaheisenberg} explained these product formulas using geometric representation theory.  This background is discussed at length in Section \ref{sec:smoothbackground.}

With the philosophy that, viewed as stacks, orbifolds should be behave just as well as smooth surfaces, it is a natural question to ask whether similar structure is found in the topology of Hilbert schemes on orbifold surfaces.  One then is lead to first start with the local models $[\C^2/G]$.  When $G$ is abelian, the techniques of Ellingsrud and Str\o mme adapt to give a combinatorial recipe for computing $H_k(\Hilb_n([\C^2/G]))$ in terms of the partitions of $n$.

This direction was taken by Gusein-Zade, Luengo, and Melle-Hern\'andez in \cite{GLM}, which we refer to as GLMZH from here.  In particular, they conjecture:

\begin{conjecture}{Gusein-Zade, Luengo, Melle-Hern\'andez}
Let $\Z/3\Z$ act on $\C^2$ diagonally.  Then
\begin{equation*}
\begin{split}
\sum_{k, n\geq 0} h_k(\Hilb_n([\C^2/\Z/3\Z]))q^nt^k =&\frac{1}{(1-q)}
\frac{1}{(1-t^2q^2)}\frac{1}{(1-q^3)} \cdot \\
&\frac{1}{(1-t^2q^4)}\frac{1}{(1-t^4q^5)}\frac{1}{(1-t^2q^6)}\cdot \\
&\frac{1}{(1-t^4q^7)}\frac{1}{(1-t^6q^8)}\frac{1}{(1-t^4q^9)}\cdots
\end{split}
\end{equation*}
\end{conjecture}

The right hand side is a $t$-analog of the Euler product, where the power of $t$ that is attached to $q^n$ is roughly $n/3$, with the some rounding up or down depending on the residue of $n$ mod 3.

It is implicit GZLMH that similar formulas should hold for any cyclic group $G$, but no explicit conjecture that holds for all $G$ was made.  Our first new contribution is to remedy this; we give an explicit product formula for.  Fittingly, our formula can be interpretted as being analogous to G\"ottsche's formula, but involving the Chen-Ruan cohomology of $[\C^2/G]$.

These product formulas are inadequate for fully understanding the topology of $\Hilb_n([\C^2/G])$.  When $X$ is a smooth surface, $\Hilb_n(X)$ is connected; however, $\Hilb_n([\C^2/G])$ is usually \emph{not} connected.  The tautological bundle $\C[x,y]/\II$ over a point $\II\subset\Hilb_n([\C^2/G])$ is a $n$-dimensional represention of $G$, and so its class in $K^0(G)$ is a discrete invariant.

For $v$ an $n$-dimensional representation of $G$, denote by $\Hilb_v([\C^2/G])\subset \Hilb_n([\C^2/G])$ the locus of ideals with $\C[x,y]/\II\cong v$ as $G$-representations.  Generalizing Evain \cite{evain1}, Maclaglan and Smith showed that $\Hilb_v([\C^2/G])$ is connected \cite{MS}.

We show in Proposition \ref{prop:connected} that in the cases we consider, $\Hilb_v([\C^2/G])$ are connected.  

In this point of view, the regular representation of $G$, which we will denote $R$, plays a special role.  Adding a smooth point of $[\C^2/G]$ corresponds to adding a copy of the regular representation.  Furthermore, for some representations $v$, the ideals in $\Hilb_v([\C^2/G])$ will be supported only over the singular points in $\C^2/G$.  For instance, $\Hilb_1([\C^2/G])$ will always be isomorphic to the $G$ fixed locus $(\C^2)^G$.

It dependence of $\Hilb_v([\C^2/G])$ on $v$ behaves very differently in the direction of the regular representation and the other directions.  In the direction of the regular representation, we prove:

\begin{theorem}
Let $v\in K^0(G)$ be arbitrary.  As $n\to \infty$, and $k$ fixed, the homology $h_k(\Hilb_{v+nR}([\C^2/\mathbb{Z}/r\mathbb{Z}]))$ stabilizes.  The generating function of the stable homology is
$$\prod_{m\geq 1} \frac{1}{(1-t^{2m})^{|r|}}$$
\end{theorem}

The proof is combinatorial, involving a generalization of the classical notion of ``cores and quotients'' of partitions.  A conjectural topological interpretation of the product formula is that all the cohomology of $\Hilb_v$ is \emph{tautological}, and that in the stable limit there are no relations among the tautological generators.


\begin{conjecture}
Let $v$ be any representation so that $\Hilb_v([\C^2/\Z_r])$ is concentrated over the origin.  Then the cohomology

$$\bigoplus_{m\geq 0} \Hilb_{v+mR} $$ 

carries an action of the Heisenberg algebra modeled on the cohomology of the minimal resolution of $\C^2/\Z_r$; it is an infinite sum of highest weight representations.
\end{conjecture}

\subsection{Piecewise quadratic behavior of ``small'' representations}


\begin{theorem}
For every class in $\overline{v}\in K^0(G)/R$ there is a unique representative $\tilde{v}\in K^0(G)$, so that $\Hilb_{\tilde{v}}([\C^2/\Z_r])$ is small.  

We have that $\dim \tilde{v}$ and $\dim \Hilb_{\tilde{v}}([\C^2/\Z_r])$ are piecewise quadratic functions on $K^0(G)/R$.
\end{theorem}

The geometric meaning or origin of $\dim \tilde{v}$ is piecewise quadratic is not clear to us; the proof and motivation for stating this theorem are both combinatorial.  That $\dim \Hilb_{\tilde{v}}$ is piecewise quadratic follows from Riemann-Roch and the fact that $\dim \tilde{v}$ is.    


We believe the quadratic part to be related to the intersection pairing of the minimal resolution of $\C^2/\Z_r$, and speculate that the piecewise behavior is related to stability conditions for the McKay quiver.




\section{Introduction for combinatorialists}
In this work, we introduce and study a generalization of cores and quotients of partitions.  Cores and quotients of partitions initially arose in the study of the modular representation theory of the symmetric group, but have found wider roles in combinatorics, number theory and representation theory, among others.  A lesser known occurence is in the geometry of Hilbert schemes of points on orbifolds; it is this occurence that leads to our generalization.  

\subsection{}
A partition is of $n$ is a nondincreasing sequence of numbers $\lambda_1\geq \lambda_2\geq \cdots \geq \lambda_\ell > 0$ with $\sum \lambda_i=n$  The \emph{length} $\ell(\lambda)$ of a partition is the number of parts, the \emph{size} $|\lambda|$ is the sum of the parts.  We use $\PP$ to denote the set of all partitions, and $\PP_n$ to denote the partitions of $n$.

Rather than a list of numbers, we usually view a partition as being equivalent to its \emph{Young diagrams}.  We draw our Young diagrams as subsets of a grid in the first quadrant, with the square in the corner being $(0,0)$, and with the columns being the parts of $\lambda$.

We write $\square\in\lambda$ to mean a cell contained in the $\lambda$.  The \emph{arm} $a(\square)$ of a square $\square\in\lambda$ is the number of cells above $\square$ and in $\lambda$; the \emph{leg} $\ell(\square)$ of a cell is the number of cells contained in $\lambda$ and to the right of $\square$.  We will occasionally need to refer to the \emph{coarm} and \emph{coleg} of a square, which we will denote $a^\prime(\square)$ and $\ell^\prime(\square)$, and which refer to the number of cells contained in $\lambda$ and below and to the left of $\square$, respectively. 

\begin{example}
Below is the Young diagram of $\lambda=4+4+3+2+2+2$.  The cell $(3,1)$ is marked $s$; the cells in the arm and leg of $s$ are labeled $a$ and $l$, respectively, and the cells in the coarm and coleg of $s$ are labeled $ca$ and $cl$, respectively/
\begin{center}
\begin{tikzpicture}[scale=.5]
\draw[thin, gray] (0,0) grid (2,4);
\draw[thin, gray] (0,0) grid (3,3);
\draw[thin, gray] (1,0) grid (6,2);

\draw[thick] (0,0)--(0,4)--(2,4)--(2,3)--(3,3)--(3,2)--(6,2)--(6,0)--cycle;
\draw (2.5,1.5) node{$s$};
\draw (2.5,2.5) node{$a$};
\draw (3.5,1.5) node{$l$};
\draw (4.5,1.5) node {$l$};
\draw (5.5,1.5) node {$l$};


\draw (0.5,1.5) node {$cl$};
\draw (1.5,1.5) node {$cl$};
\draw (2.5,0.5) node {$ca$};


\draw (10,2) node[align=left] {$a(s)=\# a=1$ \\ $\ell(s)=\# l=3$ \\ $a^\prime(s)=\# ca=1$ \\ $\ell^\prime(s)=\# cl=2$};
\end{tikzpicture}
\end{center}
\end{example}


Recall that the \emph{hook length} $h(\square)=a(\square)+\ell(\square)+1$.  Define $h_r(\lambda)$ to be the number of $\square\in\lambda$ with hooklength $h(\square)$ divisible by $r$.  Partitions with $h_r(\lambda)=0$ are called $r$-cores.  The classical notion of cores and quotients proves the following product formula

\[\sum_{\lambda\in\PP} q^{|\lambda|}t^{h_r(\lambda)}=\prod_{m\geq 1} \frac{(1-q^{rm})^r}{(1-q^m)}\prod_{m\geq 1}\frac{1}{(1-q^{rm}t^m)^r}\]
The first product here is the generating function for $r$-cores, and the second product is a generating function for $r$-quotients; $h_r(\lambda)$ is the size of the $r$-quotient of $\lambda$.  

Es
For $k, r$ integers, we define
\[\cdim^+_{k/r}(\lambda)=\#\left\{\square\in\lambda \Big | \ell(\square)-k a(\square)=-1 \mod r\right\}. \]
We see this family of statistics is related to cores and quotients because $h_r(\lambda)=\cdim^+{-1/r}(\lambda)$.

\begin{conjecture}
  The generating function
  \[ \Gcomb_{k/r}(q,t)=\sum_{\lambda\in\PP}q^{|\lambda|}t^{\cdim^+_{k/r}(\lambda)}\]
has a product formula that is a $t$ analog of the Euler product when $(k+1,r)=1$, and of the $s$-cores and quotient product when $(k+1,r)=s$.
\end{conjecture}

The precise form of this conjecture, and a geometric interpretation of it, is given in.  

\subsection{Quadratic forms}
Another part of the cores and quotients story is that $r$-core partitions are in bijection with a lattice $\Z^{r-1}$, so that the size $|\lambda|$ of a given $r$ core becomes a quadratic function on the lattice.




Using this geometry as motivation, we introduce here some new partitions statistics, and use a generalization of cores and quotients of partitions to study them.  Our first statistic turns out out to be the size of our generalized quotient, and it counts cells where certain linear combinations of the arm and leg lengths satisfy a congruence condition.  We conjecture that $(q,t)$ counting partitions with respsect to size and this new statistic satisfy explicit product formulas that are $t$-analogs of the Euler product (or more generally, $t$-analogs of the formula for partitions coming from cores and quotients).  



We prove that ``$G$-core'' partitions are in bijection with lattice points, and their size is given by a piecewise quadratic function.  




Buryak conjectured \cite{Buryak}, and later proved with Feigin \cite{BF}, and simplified the proof with Nakajima \cite{BFN}, the following version. 

\begin{theorem}[Buryak-Feigin-Nakajima] \label{thm:BFN}
Let $\C^*$ act on $\C^2$ by $t\cdot (x,y)=(t^ax, t^by), a,b\geq 0$.  Then

$$\sum_{k,n} h^k(\Hilb_n(\C^2)^{\C^*})q^nt^k=\prod_{a+b } \frac{1}{1-q^\ell}\prod_{(a+b)|\ell} \frac{1}{1-q^\ell t}$$
\end{theorem}

The proofs given by Buryak-Feigin, and Buryak-Feigin-Nakajima are algebraic geometric, and go through quiver varieties.  An obvious first step in trying to prove our conjectures combinatorially would be to prove Theorem \ref{thm:BFN} combinatorially.  





\begin{tikzpicture}[scale=.4]
\begin{scope}
\draw (3.5,8.5) node{(1/5,1/5)};
       \clip (0,0) rectangle (7,7);
\foreach \x in {0,...,6}{%
    \foreach \y in {-2,...,2}{%
       \draw[fill=red!20]  (\x, -\x+5*\y) rectangle (\x+1, -\x+5*\y+1); 
       \draw[fill=green!20]  (\x, -\x+5*\y+1) rectangle (\x+1, -\x+5*\y+2); 
       \draw[fill=blue!20]  (\x, -\x+5*\y+2) rectangle (\x+1, -\x+5*\y+3); 
       \draw[fill=black]  (\x, -\x+5*\y+3) rectangle (\x+1, -\x+5*\y+4);
        \draw  (\x, -\x+5*\y+4) rectangle (\x+1, -\x+5*\y+5); 
}}
\end{scope}     

\begin{scope}[xshift=10cm]
\draw (3.5,8.5) node{(1/5,2/5)};
\begin{scope}[xscale=-1, rotate=90]
       \clip (0,0) rectangle (7,7);
\foreach \x in {0,...,6}{%
    \foreach \y in {-4,...,2}{%
       \draw[fill=red!20]  (\x, 3*\x+5*\y) rectangle (\x+1, 3*\x+5*\y+1); 
       \draw[fill=green!20]  (\x, 3*\x+5*\y+1) rectangle (\x+1, 3*\x+5*\y+2); 
       \draw[fill=blue!20]  (\x, 3*\x+5*\y+2) rectangle (\x+1, 3*\x+5*\y+3); 
       \draw[fill=black]  (\x, 3*\x+5*\y+3) rectangle (\x+1, 3*\x+5*\y+4);
        \draw  (\x, 3*\x+5*\y+4) rectangle (\x+1, 3*\x+5*\y+5); 
}}
\end{scope}
\end{scope}     

\begin{scope}[xshift=20cm]
\draw (3.5,8.5) node{(1/5,-1/5)};
\begin{scope}[xscale=-1, rotate=90]
       \clip (0,0) rectangle (7,7);
\foreach \x in {0,...,6}{%
    \foreach \y in {-2,...,2}{%
       \draw[fill=red!20]  (\x, \x+5*\y) rectangle (\x+1, \x+5*\y+1); 
       \draw[fill=green!20]  (\x, \x+5*\y+1) rectangle (\x+1, \x+5*\y+2); 
       \draw[fill=blue!20]  (\x, \x+5*\y+2) rectangle (\x+1, \x+5*\y+3); 
       \draw[fill=black]  (\x, \x+5*\y+3) rectangle (\x+1, \x+5*\y+4);
        \draw  (\x, \x+5*\y+4) rectangle (\x+1, \x+5*\y+5); 
}}
\end{scope}     
\end{scope}

\end{tikzpicture}

By adding a $1\times r$ strip to the bottom right corner of a partition, it is immediate that if a class $v\in K(G)$ is represented by a partition, then $v+V_G$ is as well. 

\begin{definition}
The $\overline{K}(G)=K(G)/V_G$ 
\end{definition}

It is not hard to see that any class in $\overline{K}(G)$ is represented by a partition: a block staircase partition with blocks of size $|G|$ will be a multiple of $V_G$, and then modifications on each step can add a multiple of $G$ plus one of any given color.  Below, the bottom to stairs each have the effect of adding one white box in $\overline{K}(G)$, while the top two stairs each add one red box.

\begin{tikzpicture}[scale=.3]
\begin{scope}[xscale=-1, rotate=90]
       \clip (0,0) rectangle (12,12);
\foreach \x in {0,...,11}{%
    \foreach \y in {-4,...,4}{%
       \filldraw[red!20]  (\x, \x+3*\y) rectangle (\x+1, \x+3*\y+1); 
       \filldraw[green!20]  (\x, \x+3*\y+1) rectangle (\x+1, \x+3*\y+2); 
       \draw[gray] (0,0) grid (12,12);
}}
\end{scope}     
\draw[very thick] (0,9)--(3,9)--(3,6)--(6,6)--(6,3)--(9,3)--(9,0);
\draw[very thick, dashed] (0,10)--(1,10)--(1,9);
\draw[very thick, dashed] (3,7)--(4,7)--(4,6);
\draw [very thick, dashed] (6,6)--(8,6)--(8,4)--(9,4)--(9,3);
\draw [very thick, dashed] (9,3)--(11,3)--(11,1)--(12,1)--(12,0);

\end{tikzpicture}


\begin{question}
Given a class $\overline{v}\in \overline{K}(G)$, what is the smallest class $v\in K(G)$ represented by a partition?
\end{question}


\subsection{Acknowledgements}

Research on this work was supported by NSF grant NUMBER HERE, and CITE TOM's GRANT?

Thanks to Amin Gholampour, Yunfeng Jiang, and Martijn Kool for asking a question that began this work, and to Tom Bridgeland for useful conversations.

\section{Background: Partitions}


\subsection{Partitions}


The other combinatorial lens we will use to view partitions is the fermionic viewpoint, so named in .  

The boundary path is the directed lattice path starting at postive infinity on the $y$ axis, descending along the $y$ axis, tracing the boundary of $\lambda$, and then going along the $x$-axis to positive infinity.  One way to view this is as a bi-infinte word consisting of the letters $S$s and $E$; we will use the \emph{Maya diagram}, this is, on the simplest level, just translating $S$'s into empty circles, and the $E$'s into filled in circles, or stones.  



The cells $\square\in\lambda$ are in bijection with the
\emph{inversions} of the boundary path; that is, by pairs of segments
$(\text{step}_1, \text{step}_2)$, where $\text{step}_1$ occurs before $\text{step}_2$, but $\text{step}_1$ is traveling E and $\text{step}_2$ is traveing S.  The bijection sends $\square$ to the segments at the end of its arm and leg.





\section{Background: Hilbert schemes of points on smooth surfaces}
\label{sec:smoothbackground}



\subsection{Introduction to Hilbert schemes}

Throughout, $R=\C[x,y]$ will be the polynomial ring in two variables.  

The Hilbert scheme of points in the plane parameterizes ideals $\II$ of $R$ of codimension $n$:
$$\Hilb_n(\C^2)=\{\II\subset R | \dim_\C R/\II=n\}$$
The space $\Hilb_n(\C^2)$ is smooth and connected of dimension $2n$.

Geometrically, $\Hilb_n(\C^2)$ should is a resolution of the set of $n$ points in the plane, as we now describe.

 We write $\Sym^n(\C^2)=(\C^2)^n/S_n$ for the set of $n$ unordered points in $\C^2$.  The space $\Sym^n(\C^2)$ is singular for $n>1$, with singularities occuring where the points are not unique, as $S_n$ does not act freely here.  

Let $P=\{p_1,p_2,\dots, p_n\}$ be a set of $n$ distinct points in the plane, and hence a smooth point point of $\Sym^n(\C^2)$.  

Given $P$, we can also form the ideal $\II_P$ of polynomials vanishing on $P$:
 $$\II_{P}=\{f\in R| f(p)=0 \text{ for }p\in \mathbf{p}\},$$  
this is a point of $\Hilb_n(\C^2)$.  Thus, there is a locus in $\Hilb_n(\C^2)$ that agrees with the smooth locus of $\Sym^n(\C^2)$. 

 If two of the points collide, there is still a limiting ideal $\II$.  This ideal has some non-reduced structure that ``remembers how the points collided''.  This is best illustrated with an example.


\begin{example}
Let $(a,b)\neq (0,0)\in\C^2$, and let $P_t$ be the pair of distinct points $(0,0)$ and $(at,bt)$.  What is the limit $\II_{P_0}$ of $\II_{P_t}$ as $t\to 0$?

The ideal $\II_{P_t}$ can be written as follows: 
$$(\{x,y\})(\{x-at,y-bt\})=(\{x(x-at),x(y-bt), y(x-at), y(y-bt)\}$$
Setting $t=0$, we see that $\II_{P_0}$ should contain $x^2, xy$ and $y^2$.  Since $(x^2,xy,y^2)$ has codimension three instead of two, $\II_{P_0}$ must contain a linear term.

Taking the difference of the the middle two generators of $\II_{P_t}$, we see that it contains $btx-aty$. As $\II_{P_t}$ is an ideal and $t$ is a nonzero scalar, it must contain $bx-ay$.  Now as $bx-ay\in \II_{P_t}$ for $t\neq 0$, it certainly should be contained in the limit as well.  

Another way to describe $\II_{P_0}$ is
$$\II_{P_0}=\{f\in R| f(0)=\partial_vf(0)=0\}$$
where $v=(a,b)$ is the direction the two points collided in.  Thus, the non-reduced scheme structure at $\II_{P_0}$ \emph{remembers} the direction the in which the points collided. 
\end{example}

We have seen that $\Hilb_n(\C^2)$ and $\Sym^n(\C^2)$ agree for the smooth locus of $\Sym^n(\C^2)$, but that $\Hilb_n(\C^2)$ is larger over the singular locus of $\Sym^n(\C^2)$, where the points are not distinct.  It turns out that $\Hilb_n(\C^2)$ is smooth, and a resolution of singularities of $\Sym^n(\C^2)$; the resolution map is the Hilbert-Chow morphism:

\begin{definition}
The \emph{Hilbert-Chow morphism} $HC:\Hilb_n(S)\to\Sym^n(S)$ sends an ideal $\II$ to the support of $R/\II$, weighted by its multiplicity
$$HC:\II\mapsto \sum_{p\in S} \dim_p (R/\II)$$
\end{definition}









\subsection{Topology of Hilbert schemes}


We now summarize the results on the Hilbert schemes of smooth surfaces that motivate our conjecture.  A key point is that this structure is simpler to state when looking at the Hilbert schemes together for all $n$, rather than just for any fixed $n$.  

\begin{definition}
A \emph{graded space} $X$ is a disjoint union $$X=\coprod_{n=0}^\infty X_n$$
We will be lazy and write graded spaces as formal power series with coefficients in $\Var, \Top$, etc.
$$X=\sum_{n=0}^\infty X_nq^n$$
\end{definition}

The Hilbert schemes of points on a surface $S$ naturally form a graded space.

\begin{definition}
Let $S$ be a smooth surface.  We define:
$$\HG_S=\sum_{n=0}^\infty \Hilb_n(S)q^n$$
\end{definition}

\begin{definition}
Recall that for a space $X$, the Poincare polynomial $P_t(X)$ is the generating series of its Betti numbers:
$$P_t(X)=\sum_{k=0}^\infty b_k(X)$$
If $X$ is a graded space with finite dimensional graded pieces, then $P_t(X)$ is naturally an element of $\Z[t][[q]]$.
\end{definition}
 
Though for familiarity we will work with cohomology, most of our work could be done in the Grothendieck ring of varieties.

\begin{definition}
The \emph{Grothendieck ring of varieties} $K_0(\Var_k)$ is a quotient of the free abelian group on the set of isomorphism classes of varities over $k$.  We quotient out by relations of the form
$$[X]=[Y]+[X\setminus Y]$$
whenever $Y$ is a closed subvariety of $X$.   

The product structure is given by
$$[X]\times [Y]=[X\times Y]$$

Again, if $X$ is a graded space with positively graded pieces, then $[X]$ will be an element of $K_0(\Var_k)[[q]]$.

In $K_0(\Var_k)$, affine spaces are denoted $\LL^k=[\C^k]$.

\end{definition}

Before we turn to Hilbert schemes, we illustrate the use of graded spaces and the idea that the topology is simpler when considering all $n$ at once in the simple case of projective spaces.


\begin{example}[Projective space]
The fact that the Riemann sphere is obtained from $\C$ by adding a point at infinity translates to the identity $[\proj^1]=\LL+1$.

More generally, the decomposition $\proj^n=\C^n\amalg \proj^{n-1}$ gives
$$[\proj^n]=\LL^n+[\proj^{n-1}]=\LL^n+\LL^{n-1}+\cdots 1=\frac{1-\LL^{n+1}}{1-\LL}$$


We now put all projective spaces together into a graded space 
$$\proj=\sum_{n\geq 0} \proj^nq^n$$ 
A short computation with the previous line gives
$$[\proj]=\frac{1}{(1-q)(1-q\LL)}$$

\end{example}


The study of the topology of $\Hilb_n(S)$ began with the following result of Ellingsrud and Str\o mme:
\begin{theorem}[Ellingsrud and Str\o mme \cite{ES}] \label{thm:ES}

$$P_t(\HG_{\C^2})=\prod_{\ell=1}^\infty \frac{1}{1-t^{2\ell-2}q^\ell}$$
$$[\HG_{\C^2}]=\prod_{\ell=1}^\infty \frac{1}{1-\LL^{\ell +1}q^\ell}$$
\end{theorem}

Building on this, we have:

\begin{theorem}[G\"ottsche, \cite{gottsche}] \label{thm:gprod}
Let $S$ be a smooth quasiprojective surface, and let $b_i$ denote $b_i(S)$.  Then:
$$P_t(\HG_S)=\prod_{\ell\geq 1} \frac{(1+t^{2\ell-1}q^\ell)^{b_1}(1+t^{2\ell+1}q^\ell)^{b_3}}{(1-t^{2\ell-2}q^\ell)^{b_0}(1-t^{2\ell}q^\ell)^{b_2}(1-t^{2\ell+2}q^\ell)^{b_4}}$$
\end{theorem}

When $S=\C^2$, G\"ottsche's theorem reduces to Ellingsrud-Str\o mme's result.  G\"ottsche's original proof used the Weil conjectures to reduce to the local model of the smooth surfaces, $S=\C^2$.  

Our first conjecture is an analog of Theorem \ref{thm:gprod} for $S=[\C^2/G]$; our main theorem is an analog of the following corollary to G\"ottsche's formula:

\begin{corollary} \label{cor:gstab} Suppose further that $S$ is connected.
Then for fixed $k$ and large $n$, $b_{k}(\Hilb_n(S))$ stabilizes.
\end{corollary}

\begin{proof}
The point is that there is exactly one factor in the product formula that has no $t$'s, namely $1/(1-q)^{b_0}$.  When $S$ is connected, $b_0=1$.  If we remove this term, we may expand the rest of the product as a series \emph{in $t$}, and the coefficient of $t^k$ will be a polynomial $p_k(q)$ in $q$.  

Adding the $1/(1-q)$ term back in means that once a monomial $q^kt^m$ occurs, it will now also occur for all higher powers of $k$.  Hence, we see that once $n\geq\text{deg}(p_k)$, we will have $b_k(\Hilb_n(S))=p_k(1)$.
\end{proof}

Our third conjecture is an analog of the work of Grojnowski and Nakajima \cite{grojnowski, nakajimaheisenberg}. who reproved Theorem \ref{thm-gprod} by categorifying it.  They observed that power series on the right hand side of Theorem \ref{thm-gprod} is the $(q,t)$ character of the highest weight representation of the Heisenberg algebra generated by $H^*(S)$.  This lead them to hope that the vector space 
$$H^*(\HG_S)=\bigoplus_{\substack{n\geq 0 \\ k\geq 0}} H^k(\Hilb_n(S))$$
has a natural action of this Heisenberg algebra.  This is exactly what Grojnowski and Nakajima did, using nested Hilbert schemes of points.  

\begin{theorem} \label{thm:grojnowskinakajima}
$H^*(\HG_S)$ is an irreducible heighest weight representation of the Heisenberg algebra generated by $H^*(S)$.
  \end{theorem}

Moreover, in case the quasiprojective surface $S$ is the minimal resolution of the ADE singularity, this Heisenberg action is part of a quantumn group action for the corresponding semisimple quantum group.

Further categorifications and extensions of this Heisenberg action are an area of active study, see e.g. \cite{CL, CL2, FT, SV}.







\section{Background: Hilbert schemes of points on orbifolds}


\subsection{Orbifolds and stacks}
We turn now to Hilbert schemes on orbifolds.  From a naive perspective, an orbifold is a space where every point has a neighborhood isomorphic to $\C^n/G$, where $G$ is a finite group.  An object on the local chart $\C^n/G$ is a $G$-invariant object on $\C^n$: for example, a map to $\C^n/G$ is a $G$ invariant map to $\C^n$, and a sheaf on $\C^n/G$ is a $G$-invariant sheaf on $\C^n$.  Viewed in this way, orbifolds are mildly singular spaces, with the singularities the points where $G$ does not act freely.

An alternate point of view is to enlarge our category and view orbifolds as stacks with finite stabilizers.  We will use brackets $[\C^n/G]$, or calligraphic letters $\mathcal{X}$ to denote that we are working with a stack rather than a variety.  

Stacks have a reputation for being technically formidable, but we will not need much stack machinery -- just the naive point of $[\C^n/G]$ is something so that objects on $[\C^n/G]$ are $G$-\emph{equivariant} objects on $\C^n$.  


One benefit of the stacky point of view is that $[\C^n/G]$ is a smooth object in the category of stacks, and the idea that theorems for smooth spaces should have analogs for smooth stacks.


\subsection{Orbifold Hilbert schemes}

With the above point of view, the Hilbert scheme of $n$ points on $[\C^2/G]$ should parameterize $G$-equivariant ideals of $R$.  Since $R$ has an action of $G$, and ideals are subsets of $R$, the $G$ action on an ideal is forced upon us, and $G$-equivariant ideals are just $G$-invariant ideals.  Thus, the Hilbert scheme of $n$ points on $[\C^2/G]$ is just the $G$-fixed locus of the Hilbert scheme of $n$ points on $\C^2$:

$$\Hilb_n([\C^2/G])=\{\II\subset R |\dim R/\II=n, g\cdot \II=\II\forall g\in G \} =\Hilb_n(\C^2)^G$$

This is potentially confusing, given the discussion in the previous section that passing from the singular space $\C^2/G$ to the smooth stack $[\C^2/G]$ is passing from $G$-invariant objects to $G$-equivariant objects.  The Hilbert scheme of $n$-points on the singular space $\C^2/G$ is the space of ideals in the ring of invariants $R^G$

$$\Hilb_n(\C^2/G_=\{\II\subset R^G | \dim R^G/\II=n\}$$


Viewing $\Hilb_n([\C^2/G])$ as the fixed point set of the $G$ action on $\C^2$ is much simpler than the stack theoretic viewpoint, and will be our main way of working with the Hilbert schemes.  We write $\Hilb_n([\C^2/G])$ instead of $\Hilb_n(\C^2)^G$ because our motivation comes from the results about Hilbert schemes of points on smooth surfaces in section.


We highlight that $\Hilb_n([\C^2/G])$ is just a space, even though $\C^2/G$ is a stack.  Furthermore, as the fixed point set of a finite group acting on a smooth space, we immediately see that $\Hilb_n([\C^2/G])$ are smooth, giving a first hint that they behave in a similar manner.



For $G$ any finite group acting on $\C^2$, we will write $\mathcal{S}_G$ to denote the stack $[\C^2/G]$. Let $G$ be the group $\mathbb{Z}/r\mathbb{Z}$, acting with weights $1/k, k/r$, with $k$ relatively prime to $r$.  Any cyclic group acting on $\C^2$ so that only the origin has a nontrivial stabilizer will be of this form after changing coordinates on $\mathbb{C}^2$ and changing generator of $\mathbb{Z}/r\mathbb{Z}$.  We will write $\mathcal{S}_{k/r}$ to denote $\mathcal{S}_G$ in this case.


\begin{example}
Let $X$ be any variety.  Then $\Hilb_1([X/G])=\Hilb_1(X)^G=X^G$, the fixed point set of $G$.  In particular, we have that $\Hilb_1(\Sur_{k/r})$ is a point.

Note this behaviour is very different than Hilbert schemes on spaces, where $\Hilb_1(X)=X$, and suggests that for orbifolds, it is wrong to think of $n$ as counting the number of points.
\end{example}

\begin{example}
We now look at $\Hilb_2(\Sur_{k/r})$.  

First, consider the $G$ action on the locus of $\Hilb_2(\C^2)$ consisting of two distinct points $\{p, q\}$.  To be $G$-invariant, we need this set of points to be a union of $G$-orbits.   As only the origin has nontrivial stabilizer, every $G$-orbit on $\C^2$ is either the origin, consisting of one point, or consits of $|G|=r$ distinct points.  So, if $r>2$, there are no points in $\Hilb_2(\C^2)^G$ consisting of distinct points.  

If $r=2$, we must have $k=1$, the nontrivial element of $G=\Z/2\Z$ acts as multiplication by -1, and any point of $\Hilb_2(\C^2)^G$ consisting of distinct points of $\C^2$ is of the form $\{p,-p\}$

Now we consider the action of the locus where the ideal is supported at one point.  For the ideal to be $G$ invariant, this point needs to be $G$-fixed, and so must be the origin.  Thus, we are considering ideals of the form $\II_{0, v}$ for $v\in T_0C^2$.  For this to be $G$ invariant, we must have that $v$ is an eigenvector for the $G$ action.  For $k=1$, the $G$ action is diagonal, and so every vector $v$ is an eigenvector, and there is a $\proj^1$ worth of points over the origin.  If $r>2$, these are the only points, and so we have $\Hilb_2(\mathcal{S}_{1/r})=\proj^1$.  


If $k\neq 1$, there are two distinct eigenvectors, and so in this case we see that $\Hilb_2(\mathcal{S}_{k/r})$ consists of two points.


\end{example}

The example showed that, in contrast to smooth Hilbert schemes, $\Hilb_n(\Sur_G)$ need not be connected.  There is an obvious discrete invariant that explains this.

 Since $R$ and $\II$ both have $G$ actions, the module $R/\II$ is not just a vector space but a representation of $G$.  Since $G$ is finite, the set of representations are discrete, and $\II$ and $\mathcal{J}$ can't be deformed into each other if $R/\II$ isn't isomorphic to $R/\mathcal{J}$ as a representation of $G$.  

Thus, instead of just recording the dimension of a $R/\II$ we should record which representation it is:

\begin{definition}
For $v$ a representation of $G$, we define
$$\Hilb_v(\C^2/G)=\{\II|R/\II=v\}$$
\end{definition}

It turns out the $\Hilb_v$ are connected.  





The discussion also helps explain why increasing $n$ a point does not correspond to a smooth point of $\C^2/G$.  Smooth points of $\C^2/G$ correspond to points in $\C^2$ where $G$ acts freely.  If $\mathcal{O}_{Gp}$ is the structure sheaf of the orbit of such a point, then $G$ acts on $\mathcal{O}_{Gp}$ as the regular representation.  Thus, a smooth point of $\C^2/G$ correspond to copies of the regular representation.

Thus, the generic point of $\Hilb_{\C[G]}({\C^2/G})$ -- often called $G\Hilb$ in the literature -- will correspond to a smooth point of $\C^2/G$.  As $G\Hilb$ is smooth from our discussion before, and since the image of the Hilbert-Chow morphism is the $G$ invariant sets, we really have a map $G\Hilb\to \C^2/G$ that is an isomorphism away from the singular points, and so we see $G\Hilb$ is a resolution of $\C^2/G$. 

\begin{center}
\begin{tikzpicture}[scale=.5]
\draw(0,-4) node{$\widetilde{S}_{1/2}$};
\draw[-latex] (3,0)--(9,0);

\draw (0,2.236) ellipse (2 and 1);
\draw (0,-2.236) ellipse (2 and 1);
\draw[red] (0,0) ellipse (2 and 1);
\fill[white] (-2,0)--(2,0)--(2,1.1)--(-2,1.1)--cycle;
\fill[white] (-2,-2.336)--(2,-2.336)--(2,-1.136)--(-2,-1.136)--cycle;
\draw[dashed] (0,-2.236) ellipse (2 and 1);
\draw[red,dashed] (0,0) ellipse (2 and 1);

\draw (2,2.236)--(2,-2.236);
\draw (-2,2.236)--(-2,-2.236);

\begin{scope}[xshift=12cm]
\draw (0,-4) node{$S_{1/2}$};
\draw (0,2.236) ellipse (2 and 1);
\draw (0,-2.236) ellipse (2 and 1);
\fill[white] (0,0)--(1.7888, -1.7888)--(-1.7888,-1.7888)--cycle;
\draw (-1.7888,-1.7888) -- (1.78888,  1.7888);
\draw (1.7888,-1.7888) -- (-1.78888,  1.7888);
\draw[dashed] (0,-2.236) ellipse (2 and 1);
\end{scope}
\end{tikzpicture}
\end{center}

In fact, singularities of the form $\C^2/G$ have a unique minimal resolution, and $G-\Hilb$ is this resolution:





\begin{theorem} \label{thm:GHilb-resolution}
The map $G-\Hilb\to \C^2/G$ is the minimal resolution.
\end{theorem}

Theorem \ref{thm:GHilb-resolution} is useful because it gives a modular interpretation of the minimal resolution.  It also shows the geometry of the minimal resolution appearing.

The resolution of the $\C^2/G$ singularity is rational; hence the exceptional locus is a tree of $\proj^1$'s joining together.  In case $G$ is cyclic, the resolution is actually a chain of rational curves with negative self intersection.  See Chapter 10 of \cite{toric} for a thorough discussion of this.

\begin{center}
\begin{tikzpicture}[xscale=.75]
\draw plot [smooth, tension=1] coordinates {(0,0) (2,.5) (4,0)};
\draw plot [smooth, tension=1] coordinates {(3,0) (5,.5) (7,0)};
\draw plot [smooth, tension=1] coordinates {(6,0) (8,.5) (10,0)};
\draw plot [smooth, tension=1] coordinates {(9,0) (11,.5) (13,0)};

\draw (2,1) node{$-2$};
\draw (5,1) node{$-4$};
\draw (8,1) node{$-3$};
\draw (11,1) node{$-3$};

\end{tikzpicture}
\end{center}

$$
\begin{bmatrix}
-2 & 1 & 0 & 0 \\
1 & -4 & 1 & 0 \\
0 & 1 & -3 & 1 \\
0 & 0 & 1 & -3 
\end{bmatrix}
$$

The number and self-intersection of the components of the exception divisor are given by the \emph{Hirzebruch-Jung} continued fraction expansion of $r/a$.  That is, if we write
$$
\frac{r}{a}=b_1-\cfrac{1}{b_2-\cfrac{1}{b_3-\cfrac{1}{\cdots - \cfrac{1}{b_r}}}}
$$  
then the exception divisor as $r$ components in a chain, and the $i$th component has self-intersection $-b_i$. 

\begin{example}[Diagonal action]
Consider the diagonal action of $\Z_r$ on $\C^2$; this corresponds to $a=1$, and so the Hirzebruch-Jung continued fraction is simply $r/1=r$.  The exceptional divisor consists of one component with self intersection $-r$.

\end{example}


\begin{example}[The anti-diagonal case]  
In this case, the Hirzebruch-Jung continued fraction is
$$r/(r-1)=\underbrace{2-\cfrac{1}{2-\cfrac{1}{\cdots-\cfrac{1}{2}}}}_{\text{$r-1$ times}}$$
Thus, the resolution consists of a chain $r-1$ -2 curves.
\end{example}

\begin{example}[The other $\Z_5$ case]
As a final example, consder $\Z_5$ acting with weights $(1,2)$.  Switching the role of $x$ and $y$, this is equivalent to $\Z_5$ acting with weights $(1,3)$.

We have the continued fraction expansions
$$5/2=3-\cfrac{1}{2}$$
$$5/3=2-\cfrac{1}{3}$$
The minimal resolution thus consists of a -3 curve meeting a -2 curve; the choice of which direction we choose as $x$ and $y$ corresponds to which edge of the chain we start at.
\end{example}

The intersection pairing of the minimal resolution $\widetilde{S}_{a/r}$ will play an important role later.  The components of the exception curve form a basis for 


\subsection{McKay Correspondence}
The McKay correspondence is relates the geometry of $\widetilde{S}_G$ to the representation theory of $G$.

\begin{definition}
Let $G$ be a finite group, and $W$ a representation of $G$.  The \emph{McKay graph} $\Gamma_V$ is the directed graph with vertex set $\irreps(G)$, and if $U$ and $V$ are two irreps of $G$, the multiplicity of the edges from $U$ to $V$ is the multiplicity of $V$ in $U\otimes W$.
\end{definition}

The subgroups of $SU_2$ have an ADE classification.  There are two infinite series: cyclic groups, binary dihedral groups, and double covers of the isometries of the symmetries of the platonic solids.

McKay observed \cite{McKay} that when $G\subset SL_2(\C)$, the McKay graph of the defining two dimension representation was the \emph{affine} version of the corresponding ADE Dynkin diagram.  Removing the vertex corresponding to the trivial representation gives the usual ADE Dynkin diagram.  Furthermore, the exceptional locus of the minimal resolution $\widetilde{S}_G$ of the corresponding singularity has all $-2$ curves, with dual graph the corresponding Dynkin diagram. 






A geometric explanation of this was first given by Gonzalez-Sprinberg and Verdier \cite{GSV}, by constructing vectors bundles $E_\rho$ on the minimal resolution $\widetilde{S}_G$, labeled by the irreducible representations $\rho$ of $G$, so that $c_1(E_\rho)$ was the corresponding component of the exception divisor.

Ito and Nakamura observed the $G\Hilb$ is the minimal resolution \cite{IN}.

The description of the minimal resolution $\widetilde{S}_G$ as $G-\Hilb$ gives a natural construction of these vector bundles: for each irrep $\rho$ of $G$, we have on the one hand a tautological bundle $\mathbb{E}_\rho$ over $G-\Hilb$, and on the other hand a vertex of the McKay graph.  

$$c_1(\mathbb{E}_\rho)=E_\rho$$

Furthermore, viewing the minimal resolution as a moduli space of objects on the orbifold gives rise to powerful tools using the derived category.  

In dimensions higher than two, Hilbert schemes of points on a smooth surface are not smooth, and we no longer have any reason to expect that $G-\Hilb$ would be smooth, or a crepant resolution of $\C^n/G$ -- indeed, in dimensions 4 or higher, gorenstein quotient singularities need not have crepant resolutions.

It turns out that in many cases of interest, $G-\Hilb$ is smooth in this case, and a crepant resolution of $\C^2/G$.  This was proven by Bridgeland, King and Reid in \cite{BKR} for dimension three, and Haiman's work on $\Hilb_n(\C^2)$ is essentially showing that it is $S_n-\Hilb(\C^n)$.



\subsection{The special McKay correspondence}

When $G$ is not in $SL_2$, the number of components in the exception divisor of $\widetilde{S}_G$ is strictly less than the number of nontrivial irreducible representations of $G$.  The \emph{special McKay correspondence} started by Wunram \cite{wunram} picks out a subset of the irreps of $G$, called the \emph{special} representations, and gives a labeling of the irreducible components of the exception divisor of $G$ by the special representations, so that we still have $c_1(\mathbb{E}_\rho)=E_\rho$.

Kidoh \cite{Kidoh} in the cyclic case and Ishii \cite{ishii} in the general case proved that we still have $G\Hilb$ is the minimal resolution.

Ito \cite{ito} gave the following combinatorial description of the special representations.  

\begin{theorem}[\cite{ito}, Theorem 3.7]
Defines $B(G)$ to be the set of monomials which are not divisible by any $G$-invariant monomial, and defines $L(G)$ to be the set of monomials not divisible by $xy, x^{|G|}, y^{|G|}$.  Then a representation $\rho$ is special if and only if the corresponding monomial is not contained in $B(G)\setminus L(G)$.
\end{theorem}


Although we will work mainly at the level of homology, it is worth noting the McKay and special McKay correspondences hold at the level of derived categories.  (KAPRANOV-VASSEROT, ISHII)


\subsection{Colored square counting}

The core construction helps us with the colored box square counting:

Consider the function
$$P_G(q_0,\dots,q_{r-1})=\sum_{\lambda\in\mathcal{P}} \mathbf{q}^{|\lambda|^G}$$
where 
$$\mathbf{q}^{|\lambda|_G}=\prod_{i=0}^{r-1} q_i^{|\lambda|^G_i}$$
that counts partitions according to their full colored square count as opposed to just their size; alternatively, the coefficient of $\mathbf{q}^v$ is the euler characteristic of $\Hilb_v([\C^2/G])$.

What can we say about $P_G$?  First, we consider the case where $G\subset SL_2$.  Let $Q=q_0q_1\cdots q_{r-1}$.  The core construction gives
$$P_G=\prod_{i=1}^\infty\frac{1}{(1-Q^i)^r} \sum_{w} Q^{A(w)}\mathbf{q}^w $$
Thus, we see in this case $P_G$ is a multivariable theta function.

In case $r=2$, this has an infinite product expansion using the Jacobi identity.  In fact, work of Boulet shows that actually the case $G=\Z_2\times \Z_2$ has an infinite product expansion.  Letting $q_{00}, q_{01}, q_{10}, q_{11}$ denote the variables, we have

\begin{theorem}[Boulet \cite{boulet}]
$$P_{\Z_2\times\Z_2}=\prod_{i=1}^\infty \frac{(1+q_{00}^jq_{01}^{j-1}q_{10}^{j-1}q_{11}^{j-1})(1+q_{00}^jq_{01}^jq_{10}^jq_{11}^{j-1})}{(1-q_{00}^jq_{01}^jq_{10}^jq_{11}^j)(1-q_{00}^jq_{01}^jq_{10}^{j-1}q_{11}^{j-1})(1-q_{00}^jq_{01}^{j-1}q_{10}^{j}q_{11}^{j-1})}$$
\end{theorem}

However, even the one variable specialization $P_{\Z_3}(q,1,1)$ does not have a nice infinite product expression, as observed by Bal\'azs Szendr\"oi \cite{Bmo}; as it has a root at $-e^{\pi/\sqrt{3}}$ \cite{Borwein2}.

\subsection{Dijkgraaf, orbifold partitions, and the topology of Quot schemes}

Specialization where one $q_i$ is set to $q$ and all others are set to one is studied in \cite{DS}, where they are all called \emph{orbifold partitions of the first type}.  There, they show that these specializations are characters of the affine Kac-Moody lie algebras.

We briefly mention that they also introduce \emph{orbifold partitions of the second type}, where only cells of the given color are coordinated.  Combinatorially, this can be viewed as putting an equivalence relation on partitions, so that two partitions are equivalent if they contain exactly the same boxes of color $i$.  Algebraically, instead of considering $G$-equivariant ideals of $R$, this is working $G$-invariantly.  That is, we have the ring $R^G$ of $G$ invariant elements of $R$.  The ring $R$ is a module over $R^G$, and splits into a direct sum of modules indexed by the irreducible representations of $R$.  Rather than counting ideals in $R$ with quotient having length $n$, they are counting submodules of $R_i$ of colength $n$.  

Geometrically, $R^G$ is the ring of functions on the singular space $\C^2/G$.  The modules $R_i$ are spaces of sections of sheave $\mathcal{F}_i$ that are .  These ideals are $T$ equivariant, and so they correspond to the euler characteristic of the quot scheme.  


\begin{question}
What can be said about the Quot?  Homology groups, decomposition in the grothendieck ring of varieites, etc.

How are these related to the topo

\end{question}



\subsection{Hilbert schemes of points in the plane}


That is, we look for 
$$\DC_G(q,t):=\sum_{n,k\geq 0 } b_k(\Hilb_n([\C^2/G])) t^kq^n$$







\subsubsection{Homological stability}

The analogs of stabilization and geometric representation theory work on the level of connected Hilbert scheme.

\begin{theorem} 
$P_t(\Hilb^{\delta+nG}_G)$ stabilizes to $1/(t,t)_\infty^{|G|}$
\end{theorem}
Note that the right hand side is independent of $m$ and $\delta$.
\begin{proof} Combinatorics -- a generalization of cores and quotients of partitions \end{proof}

\begin{conjecture}
The stable cohomology of $\Hilb^{\delta+nG}$ is freely generated by the Chern classes of the $|G|$ tautological bundles.
\end{conjecture}



\subsubsection{Heisenberg Action}


\begin{conjecture}
Let $\delta\in K_0(G)$ be small, and $G$ cyclic.  Then
$$\bigoplus_{k\geq 0} H_*(\Hilb^{\delta+kG}_G)$$ admits the action of a Heisenberg algebra based on the cohomology of the minimal resolution of $\C^2/G$.
\end{conjecture}

Evidence:
Let $c$ be the number of rational curves in the minimal resolution of $\C^2/G$.  Then
$$\mathcal{CH}^\delta_G\cdot(q,qt^2)_\infty\cdot (qt^2,qt^2)_\infty^c$$
has positive coefficients; but higher powers start giving negative coefficients.


\subsubsection{Resolutions of $(\C^2/G)^n/S_n$}
One family of resolutions

Let $S_G$ be the minimal resolution of $\C^2/G$.  Then $\Hilb_n(S_G)$ is a resolution of $(\C^2/G)^n/S_n$.


Another family of resolutions

Let $\delta\in K^0(G)$ be such that $\Hilb^\delta([\C^2/G])=pt$.  Then $\Hilb^{\delta+nG}([\C^2/G])$ is a resolution of $(\C^2/G)^n/S_n$.  



Stabilization implies that for fixed $\delta$, and large $n$, this second resolution will be bigger than the first resolution.

Changing Stability?
But, it seems as $\delta\to\infty$, this second family of resolutions converges to the first.


\section{Warm-up: Cores, quotients, and the antidiagonal case}

In this section, we focus on the anti-diagonal case $-1/r$; that is, where $A=\Z_r$ acting with weights $(1,-1)$.  In this case, much is already well known: combinatorially, cores and quotients, together with the abacus construction, give us much information.  Topologically, we are in the $SL_2$ case, and the components of the Hilbert scheme have a descipription as Nakajima quiver varieties. 

Though there has been some work connecting the combinatorial and geometric viewpoints, we suspect more c

\subsection{Colored boxes}

In the anti-diagonal case, the coloring scheme of the cells is a familiar object in the student of partitions, but not one as frequently connected to core and quotient partitions.

\begin{definition}
The \emph{content} $c(\square)$ of a cell $\square=(i,j)\in\lambda$ is $c(\square)=i-j$
\end{definition}

\begin{tikzpicture}

\draw[thin, gray] (0,0) grid (4.5,4.5);
\foreach \x in {0,...,4}
   \foreach \y in {0,...,4}
       \pgfmathsetmacro\result{\x-\y}
       \node at (\x+.5,\y+.5) {\pgfmathprintnumber{\result} };
\node at (10,2.5) {Each cell is labeled by its content};

\end{tikzpicture}


It is immediate from the definition, that if we let $\C^*_{1,-1}$ be the torus that acts on $x$ with weight 1 and $y$ with weight $0$, then the content $c(\square)$ is just the weight of the $\C^*$ action on the corresponding monomial in $R/\II_\lambda$.

Furthermore, the multiset of the contents of $\lambda \mod r$ is just the class of $[R/\II]\in K^0(\Z_r)$.


\subsection{Topology of orbifold Hilbert schemes}

The methods used to study the topology of Hilbert schemes of points on $\C^2$ adapts easily to Hilbert schemes of points on $[\C^2/G]$ when $G$ is abelian.  We view $\Hilb([\C^2/G])$ as $\Hilb(\C^2)^G$.  When $G$ is abelian, its action can be simultaneously diagonalized.  We use the $T=(\C^*)^2$ to be the one acting on the diagonal coordinates for $G$, so that $G\subset T$, and the actions of $G$ and $T$ commute. 

In particular, since $T$ commutes with $G$, its action will preserve $G$ fixed point sets, and in particular we get a $T$ action on $\Hilb_n(\C^2)^G$, and we can proceed exactly as we did in studying $\Hilb_n(\C^2)$.



\begin{lemma} Let $G$ be an abelian group acting on $\C^2$.  Then
\[ \chi( \Hilb_n([\C^/G]))=p(n) \]
\end{lemma}

\begin{proof}
Since $G\subset T$, we have $\Hilb_n(\C^2)^{\C^*}\subset\Hilb_n(\C^2)^G$, and so all the monomials ideals are $G$ invariant.  
\end{proof}

Similarly, we can use the Ellingsrud-Str\o mme's calculation of torus weights on $T_\lambda\Hilb_n$ to use the Bia\l ynicki-Birula decomposition to find the Betti numbers of $\Hilb_n([\C^2/G])$.  

In general, if $G$ is a group acting on a manifold $M$ and $p\in M^G$ is a fixed point, the tangent directions of $M^G$ at $p$ are just the $G$-fixed tangent directions to $p$ in $M$: $T_p(M^G)=(T_pM)^G$.  

Thus, if $\II\in \Hilb_n([\C^2/G])$ is a monomial ideal, the tangent space $T_{\II}\Hilb_n([\C^2/G])$ is just the $G$ invariant part of Ellingsurd and Str\o mme's calculation.  For $G$ a finite cycle group, a term in Lemma \ref{lem:torus-weights} will be $G$ invariant if and only if the arm and leg satisfy some linear congruence relation; for $G$ a product of two cyclic groups, it will be a system of linear relations.


\begin{definition}
Let $\lambda$ be a partition of $n$, and $k<r$ relatively prime.  We define 
\[\dim_{k/r}(\lambda)=\dim T_\lambda\Hilb_n([\C^2/G_{k/r}])\]
Furthermore, let $\C^*$ act on $\C^2$ with weights $(\varepsilon, 1)$, and consider the induced $\C^*$ action on $\Hilb_n([\C^2/G_{k/r}])$.  We define 
\[\dim^{\pm}_r \lambda =\dim T^\pm_\lambda(\Hilb_n([\C^2/G_{k/r}])\]
to be the dimensions of the positive and negative eigenspaces of this action. 
\end{definition}

\begin{lemma} We have

\begin{align*}
\dim_{k/r}(\lambda)&=\#\left\{\square\in\lambda \Big | \ell(\square)-k a(\square)\in\{-1,k\} \mod r\right\} \\
\dim^+_{k/r}(\lambda)& =\#\left\{\square\in\lambda \Big | \ell(\square)-k a(\square)=-1 \mod r \text{ and } a(\square)>0\right\} \\
\dim^-_{k/r} &= \dim_{k/r}-\dim^+_{k/r}
\end{align*}
\end{lemma}
\begin{proof}
Let $L$ denote the representation where the standard generator of $G_{k/r}$ acts as $\exp(2\pi i/r)$. Under the inclusion $i:G_{k/r}\in T$, we have $i^*(T_1)=L, i^*(T_2)=L^k$.  Thus, we see that as a $G_{k/r}$-rep, we have
$$T_\lambda \Hilb_n(\C^2)=\sum_{\square\in\lambda} L^{-\ell(\square)+ka(\square)+k}+L^{\ell(\square)+1-ka(\square)}$$
The tangent space to $\Hilb_n(\C^2/G_{k/r})$, being the $G$ invariant directions, will correspond to the exponents that are divisible by $r$.  This immediately gives the first line.

To find the dimension of positive and negative eigenspaces  we intersect our answer to the first line with Prop whatever.

The last line is equivalent to all fixed points being isolated.
\end{proof}

Combinatorially, it seems awkward that cells with $a(\square)=0$ are included in the negative eigenspace instead of the positive eigenspace, and it is natural to make the following definition:
\begin{definition}
The \emph{combinatorial} positive and negative dimensions, $\cdim^\pm_{k/r}(\lambda)$ are defined by
\begin{align*}
\cdim^+_{k/r}(\lambda)&=\#\left\{\square\in\lambda \Big | \ell(\square)-k a(\square)=-1 \mod r\right\} \\
\cdim^-_{k/r}(\lambda)&=\#\left\{\square\in\lambda \Big | \ell(\square)-k a(\square)=k \mod r\right\} \\
\end{align*}
\end{definition}

\begin{example}[Hook Lengths]
The case when $G\subset SL_2$ is $k=r-1$, or equivalently, $k=-1$.  In this case, we have
$$\cdim^+_{-1/r}(\lambda)=\cdim^-_{-1/r}(\lambda)=\#\left\{\square\in\lambda \Big | h(\square)=0\mod r\right\} \\$$
and so these statistics count the number of cells with hook lengths divisible by $r$.
\end{example}

The generating functions for $\cdim^+_{k/r}$ and $\dim^+_{k/r}$ are closely related.

\begin{lemma}
$$\prod_{m>0}(1-q^{mr}t^m)\sum_{\lambda\in\PP} q^{|\lambda|}t^{\cdim^+_{k/r}(\lambda)}=\prod_{m>0} (1-q^{mr}t^{m-1}) \sum_{\lambda\in\PP} q^{|\lambda|}t^{\dim^+_{k/r}(\lambda)}$$
\end{lemma}
\begin{proof}
The two statistics $\cdim^+_{k/r}$ and $\dim_{k/r}$ both count cells $\square$ satisfying the congruence relation $\ell(\square)-ka(\square)=-1\mod r$; the combinatorial dimension includes those cells with $a(\square)=0$, while the usual dimension does not.  

If a square satifies $a(\square)=0$ it is on the top of its column; when $a(\square)=0$ the congruence relation becomes simply asking that $\ell(\square)+1$ is divisible by $r$. Since $\square$ is at the top of its column, that means the two differ whenever we have a part of $\lambda$ appearing at least $kr$ times.  In particular, the two statistics agree on partitions $\lambda$ where no part has multiplicity $r$ or greater.

If there are $r$ or more parts of size $m$ in $\lambda$, one may remove $r$ parts of them, resulting in removing an $r\times k$ rectangle of squares from $\lambda$.  Doing so will not change $\ell(\square)-ka(\square)\mod r$ for any square in $\lambda$ -- we haven't changed the arm or leg of any square to the right of the rectangle removed, while we have reduced the leg of any square to the right by $r$.  Furthermore, each row of squares we removed will contain exactly one square contributing to $\cdim^+_{k/r}(\lambda)$, since each square in a row has the same arm length, and as we move across the row from right to left the leg lengths decrease by 1.  

Thus, we have seen that if we let $\PP^{<r}$ to denote the set of partitions with multiplicities of all parts less than $r$, we have
$$\prod_{m>0}\frac{1}{1-q^{mr}t^m}\sum_{\lambda\in\PP} q^{|\lambda|}t^{\cdim^+_{k/r}(\lambda)}=\sum_{\lambda\in\PP^{<r}}q^{|\lambda|}t^{\cdim^+_{k/r}(\lambda)}$$
The same analysis holds for $\dim^+_{k/r}$, except now the top row of each rectangle removed does not contribute to $\dim^+$, explaining why the powers of $t$ in the product are one lower.

The shaded cells are the cells contributing to $\cdim^+_{1/3}(\lambda)$; that is, those cells $\square$ with $\ell(\square)-a(\square)=-1\mod 3$.  The regions contained in the dashed red lines correspond to the parts removed -- notice that each row of each region contains one shaded box, and that removing the dashed regions does not changed whether the leftover cells are shaded or not.


\begin{tikzpicture}[scale=.5]
\begin{scope}
\fill[red!20!white] (0,1) rectangle (2,2);
\fill[red!20!white] (2,2) rectangle (3,5);
\fill[red!20!white] (2,1) rectangle (3,0);
\fill[red!20!white] (3,1) rectangle (4,2);
\fill[red!20!white] (4,2) rectangle (5,4);
\fill[red!20!white] (4,0) rectangle (5,1);
\fill[red!20!white] (6,1) rectangle (9,2);
\fill[red!20!white] (9,1) rectangle (10,0);
\fill[red!20!white] (11,1) rectangle (12,2);
\fill[red!20!white] (12,1) rectangle (13,0);
\fill[red!20!white] (16,1) rectangle (17,0);

\draw[thick, dashed, red] (4.1,.1)--(4.1,3.9)--(6.9,3.9)--(6.9,.1)--cycle;
\draw[thick, dashed, red] (8.1,.1)--(8.1,1.9)--(10.9,1.9)--(10.9,.1)--cycle;
\draw[thick, dashed, red] (11.1,.1)--(11.1,1.9)--(13.9,1.9)--(13.9,.1)--cycle;
\draw[thick, dashed, red] (16.1,.1)--(16.1,.9)--(18.9,.9)--(18.9,.1)--cycle;


\path[clip, preaction={thick, draw}] (0,0)--(0,7)--(1,7)--(1,6)--(3,6)--(3,4)--(7,4)--(7,3)--(8,3)--(8,2)--(14,2)--(14,1)--(19,1)--(19,0)--cycle;
\draw[thin, gray] (0,0) grid (20,7);
\end{scope}

\begin{scope}[yshift=-10cm]
\fill[red!20!white] (0,1) rectangle (2,2);
\fill[red!20!white] (2,2) rectangle (3,5);
\fill[red!20!white] (2,1) rectangle (3,0);
\fill[red!20!white] (3,1) rectangle (5,2);


\path[clip, preaction={thick, draw}] (0,0)--(0,7)--(1,7)--(1,6)--(3,6)--(3,4)--(4,4)--(4,3)--(5,3)--(5,1)--(7,1)--(7,0)--cycle;
\draw[thin, gray] (0,0) grid (7,7);
\end{scope}

\end{tikzpicture}
\end{proof}


Combinatorially, our motivation lies in the notion of core partitions.
\begin{definition}
Let $\lambda$ be a partition.  The \emph{hook length} $h(\square)$ of $\square\in\lambda$ is $a(\square)+\ell(\square)+1$.

We say $\lambda$ is a \emph{$t$-core} if it has no cells $\square\in\lambda$ with $h(\square)=t$.
\end{definition}

There are many equivalent definitions of $t$-core partitions; when we generalize them some of these definitions will still hold and some will be different.

\begin{lemma}
The following are equivalent:
\begin{enumerate}
\item $\lambda$ is a $t$-core
\item $\lambda$ has no boundary strips of length $t$
\item $\lambda$ has no cells $\square$ with $t|h(\square)$
\item It is not possible to remove one square of each content from $\lambda$
\item $\lambda$ is the only partition with its set of contents
\item $\II_\lambda$ is an isolated point in $\Hilb_n([\C^2/G_{-1/t}])$
\end{enumerate}


\end{lemma}


Many of these

\subsection{Abacus construction}

The abacus construction builds on the boundary path description of partitions in Section .

Rather than working with bi-infinite strings of letters, we will prefer to work with \emph{Maya diagrams}, which we now describe.  The conventions in the following discussion are based on the ``fermionic'' viewpoint of partitions and lattice paths, based on Dirac's electron sea model, see \cite{KR} for more. 


\subsection{Paths}

We will use $\Z_{1/2}$ to denote the set of half integers $\Z+1/2$, i.e., $-1/2$ and $3/2$ are in $\Z_{1/2}$, but $2$ is not.  $\Z_{1/2}^+$ will denote the positive half integers, and $\Z_{1/2}^-$ will denote the negative.



\begin{definition}
A state $S$ is a subset $S\subset \Z^{1/2}$ so that the symmetric difference of $S$ with $\Z^-_{1/2}$ is finite; that is $S\cap\Z^+_{1/2}$ and $S^c\cap \Z^-_{1/2}$ are both finite.  We call $|S\cap \Z^-_{1/2}|-|S\cap \Z^+_{1/2}|$ the \emph{charge} of $S$.

We will typically represent a state by a \emph{Maya diagram} -- this is a sequence of circles labeled by $\Z_{1/2}$, with the positive entries going to the left and the negative entries to the right.  A black bead is placed at each of the entries of $S$, and the entries not in $S$ are displayed as white circles.
\end{definition}

We now describe a bijection between the set of partitions $\mathcal{P}$ to the set of charge 0 states/


\subsubsection{}
We draw partitions in ``Russian notation'' -- rotated $\pi/4$ radians counterclockwise and scaled up
by a factor of $\sqrt{2}$, so that each segment of the border path of $\lambda$ is centered above a half integer on the $x$-axis, with origin above the square 0.



\begin{example} \label{ex:electronstopartitions}

We illustrate the bijection in the case of $\lambda=3+2+2$.  The corresponding state $S_\lambda$ consists of two
electrons with energy $5/2$ and $1/2$, and two positrons with energy
$3/2$ and $5/2$.

\begin{center}
\begin{tikzpicture}
\begin{scope}[gray, very thin, scale=.6]
\clip (-5.5, 5.5) rectangle (5.5, -5);
\draw[rotate=45, scale=1.412] (0,0) grid (6,6);
\end{scope}

\begin{scope}[rotate=45, very thick, scale=.6*1.412]
\draw (0,5.5) -- (0, 3) -- (1,3) -- (1,2) -- (3,2) -- (3,0) -- (5.5,0);
\end{scope}

\begin{scope}[scale=.6, dotted]

\draw (-4.5,0) -- (-4.5, 4.5);
\draw (-3.5,0) -- (-3.5, 3.5);
\draw (-2.5,0) -- (-2.5, 3.5);
\draw (-1.5,0) -- (-1.5, 3.5);
\draw (-.5,0) -- (-.5, 3.5);
\draw (.5,0) -- (.5, 4.5);
\draw (1.5,0) -- (1.5, 4.5);
\draw (2.5,0) -- (2.5, 3.5);
\draw (3.5,0) -- (3.5, 3.5);
\draw (4.5,0) -- (4.5, 4.5);
\end{scope}


\begin{scope}[scale=.6, yshift=-.5cm]
\draw[solid] (0,-.5) -- (0,.5);
\draw (-5.5,0) node{$\cdots$};
\draw (-4.5,0) circle (.3) node[below=3pt]{$\frac{9}{2}$};
\draw (-3.5,0) circle (.3) node[below=3pt]{$\frac{7}{2}$};
\filldraw (-2.5,0) circle (.3) node[below=3pt]{$\frac{5}{2}$};
\draw (-1.5,0) circle (.3) node[below=3pt]{$\frac{3}{2}$};
\filldraw (-.5,0) circle (.3) node[below=3pt]{$\frac{1}{2}$};
\filldraw (.5,0) circle (.3) node[below=3pt]{$\frac{-1}{2}$};
\draw (1.5,0) circle (.3) node[below=3pt]{$\frac{-3}{2}$};
\draw (2.5,0) circle (.3) node[below=3pt]{$\frac{-5}{2}$};
\filldraw (3.5,0) circle (.3) node[below=3pt]{$\frac{-7}{2}$};
\filldraw (4.5,0) circle (.3) node[below=3pt]{$\frac{-9}{2}$};
\draw (5.5,0) node{$\cdots$};
\end{scope}
\end{tikzpicture}
\end{center}

\end{example}







The bijection between partitions and states of charge zero may be
modified to give a bijection between partitions and states of charge $c$ for any $c\in\Z$.   Simply translate the partition to the right by $c$.



\subsection{Abaci}

Rather than view the Maya diagram as a series of stones in a line, we
now view it as beads on the runner of an abacus.  Sliding the beads
to be right justified allows the charge of the state to be read off,
as it is easy to see how many electrons have been added or are missing
from the vacuum state.

In what follows, we mix our metaphors and talk about electrons and protons on runners of an abacus.

\begin{example} \label{ex:mayabijection}
Consider Example \ref{ex:particles}, where the Maya diagram consists of
two positrons and an electron.  Pushing the beads to be right
justified, we see the first bead is one step to the right of zero, and
hence the original state had charge 1.

\begin{center}
\begin{tikzpicture}[scale=.6]
\draw (0,-.5) to (0,.5);
\draw (-5.5,0) node{$\cdots$};
\draw (-4.5,0) circle (.3) node[below=3pt]{$\frac{9}{2}$};
\draw (-3.5,0) circle (.3) node[below=3pt]{$\frac{7}{2}$};
\draw (-2.5,0) circle (.3) node[below=3pt]{$\frac{5}{2}$};
\filldraw (-1.5,0) circle (.3) node[below=3pt]{$\frac{3}{2}$};
\draw (-.5,0) circle (.3) node[below=3pt]{$\frac{1}{2}$};
\draw (.5,0) circle (.3) node[below=3pt]{$\frac{-1}{2}$};
\filldraw (1.5,0) circle (.3) node[below=3pt]{$\frac{-3}{2}$};
\draw (2.5,0) circle (.3) node[below=3pt]{$\frac{-5}{2}$};
\filldraw (3.5,0) circle (.3) node[below=3pt]{$\frac{-7}{2}$};
\filldraw (4.5,0) circle (.3) node[below=3pt]{$\frac{-9}{2}$};
\draw (5.5,0) node{$\cdots$};

\begin{scope}[yshift=-2.5cm]
\draw[-triangle 45, snake=snake,line after snake=1mm] (-4,0)--(4,0) node [above,  text centered,midway] {Push beads};

\end{scope}

\begin{scope}[yshift=-4cm]
\draw (0,-.5) to (0,.5);
\draw (-5.5,0) node{$\cdots$};
\draw (-4.5,0) circle (.3) node[below=3pt]{$\frac{9}{2}$};
\draw (-3.5,0) circle (.3) node[below=3pt]{$\frac{7}{2}$};
\draw (-2.5,0) circle (.3) node[below=3pt]{$\frac{5}{2}$};
\draw (-1.5,0) circle (.3) node[below=3pt]{$\frac{3}{2}$};
\draw (-.5,0) circle (.3) node[below=3pt]{$\frac{1}{2}$};
\draw (.5,0) circle (.3) node[below=3pt]{$\frac{-1}{2}$};
\filldraw (1.5,0) circle (.3) node[below=3pt]{$\frac{-3}{2}$};
\filldraw (2.5,0) circle (.3) node[below=3pt]{$\frac{-5}{2}$};
\filldraw (3.5,0) circle (.3) node[below=3pt]{$\frac{-7}{2}$};
\filldraw (4.5,0) circle (.3) node[below=3pt]{$\frac{-9}{2}$};
\draw (5.5,0) node{$\cdots$};
\end{scope}


\end{tikzpicture}
\end{center}
\end{example}

\subsubsection{Cells and hook lengths}

The cells $\square\in\lambda$ are in bijection with the
\emph{inversions} of the boundary path; that is, by pairs of segments
$(\text{step}_1, \text{step}_2)$, where $\text{step}_1$ occurs before $\text{step}_2$,
but $\text{step}_1$ is traveling NE and $\text{step}_2$ is traveing SE.  The bijection
sends $\square$ to the segments at the end of its arm and leg.

Translating to the fermionic viewpoint, cells of $\lambda$ are in
bijection with pairs 
$$\left\{(e, e-k)\big | e\in\Z_{1/2}, k>0\right\}$$ of a filled energy level $e$ and an


The arm $a(\square)$ corresponds to the number of energy levels between $e$ and $e-k$ that are empty; the leg $\ell(\square)$ is the number of energy levels between $e$ and $e-k$ that are full.  The hook length $h(\square)$ of the corresponding cell is $k$, the distance between the two sets with 

Recall that a \emph{rim hook} of length $a$ of $\lambda$ is a size $a$ subset of the cells of $\lambda$ so that removing the rim hook gives a small partition, and the rim hook does not contain a $2\times 2$ box.  

\begin{lemma} \label{lem:rimhooks}
Rimhooks of size $a$ in $\lambda$ are in bijection with cells $\square\in\lambda$ with $h(\square)=a$.  Removing the rimhook corresponds to moving the stone at $e$ and playing at $e-k$ instead.
\end{lemma}

This is illustrated in the following picture:

\begin{tikzpicture}[scale=.8]
\begin{scope}[gray, very thin, scale=.6]
   \clip (-5.5, 5.5) rectangle (5.5, -1);
   \draw[rotate=45, scale=1.412, fill, red!20!white] (1,1)--(1,2)--(3,2)--(3,0)--(2,0)--(2,1)--cycle;
   \draw[rotate=45, scale=1.412] (0,0) grid (6,6);
\end{scope}
\begin{scope}[rotate=45, very thick, scale=.6*1.412]
   \draw (0,5.5) -- (0, 3) -- (1,3) -- (1,2) -- (3,2) -- (3,0) -- (5.5,0);
\end{scope}
\begin{scope}[scale=.6, yshift=-.5cm]
   \draw (-5.5,0) node{$\cdots$};
   \draw (-4.5,0) circle (.3);
   \draw (-3.5,0) circle (.3);
   \filldraw (-2.5,0) circle (.3);
   \draw (-1.5,0) circle (.3);
   \filldraw[gray] (-.5,0) circle (.3);
   \filldraw (.5,0) circle (.3);
   \draw (1.5,0) circle (.3);
   \draw (2.5,0) circle (.3);
   \filldraw (3.5,0) circle (.3);
   \filldraw (4.5,0) circle (.3);
   \draw (5.5,0) node{$\cdots$};
   \node (start) at (-.3, -.3) {};
   \node(finish) at (2.3, -.3) {};
   \draw[->] (start) to [out=-30, in=210] (finish);
\end{scope}
\begin{scope}[xshift=8cm]
   \begin{scope}[gray, very thin, scale=.6]
      \clip (-5.5, 5.5) rectangle (5.5, -1);
      \draw[rotate=45, scale=1.412] (0,0) grid (6,6);
   \end{scope}
   \begin{scope}[rotate=45, very thick, scale=.6*1.412]
      \draw (0,5.5) -- (0, 3) -- (1,3) -- (1,1) -- (2,1) -- (2,0) -- (5.5,0);
   \end{scope}
   \begin{scope}[scale=.6, yshift=-.5cm]
      \draw (-5.5,0) node{$\cdots$};
      \draw (-4.5,0) circle (.3);
      \draw (-3.5,0) circle (.3);
      \filldraw (-2.5,0) circle (.3);
      \draw (-1.5,0) circle (.3);
      \draw (-.5,0) circle (.3);
      \filldraw (.5,0) circle (.3);
      \draw (1.5,0) circle (.3);
      \filldraw (2.5,0) circle (.3);
      \filldraw (3.5,0) circle (.3);
      \filldraw (4.5,0) circle (.3);
      \draw (5.5,0) node{$\cdots$};
   \end{scope}
\end{scope}
\end{tikzpicture}


If $(e,e-k)$ is such a pair, reducing the energy of the electron from
$e$ to $e-k$ changes $\lambda$ by removing the
rim hook corresponding to the cell $\square$.  This rim-hook has
length $k$.


\begin{definition} 
The $r$-core of $\lambda$, denoted $\core_r(\lambda)$, the partition obtained from $\lambda$ by iteratively choosing a rim-hook of size $r$ and removing it.
\end{definition}

It is not clear from the above definition that $\core_r(\lambda)$ is well defined -- it seems possible that removing rim hooks in different orders could result in different partitions.  We will see later that it is in fact well defined.  Assuming that $\core_r(\lambda)$ is well defined, it follows from Lemma \ref{lem:rimhooks} that $\core_r(\lambda)$ is an $r$-core.


\begin{example}
The cell $\square=(2,1)$ of $\lambda=3+2+2$ (See Example \ref{ex:electronstopartitions}).
Here, $h(\square)=3$, and corresponds to the electron in energy state $1/2$
and the empty energy level $-5/2$; which are three apart.

\end{example}

\subsection{Bijections}

The essence of the cores and quotient construction is that, rather than place the electrons corresponding to $\lambda$ on one runner, we place them on $r$ different runners, putting the energy levels $ka-i-1/2$ on runner $i$ -- i.e., every $r$th bead goes to the same runner, as in Figure \ref{ }

The charge $c_i$ on the $i$th runner need not be 0, but using the change, we may still view the beads on the $i$th runner as a Maya diagram of a partition.  


\begin{definition} \label{def:quotients}
The \emph{$r$-quotient} of $\lambda$, denoted $\quot_r(\lambda)$ is an $r$-tuple of partitions, where the $i$th partition $\quot_r^i(\lambda)$ is obtained by reading off the $i$th runner of the $r$-abacus of $\lambda$ as a partition.

The size of the $r$-quotient is the sum of the sizes of the individual partitions: 
$$|\quot_r(\lambda)|=\sum_{i=1}^r |\quot_r^i(\lambda)|$$
\end{definition}

The abacus construction also gives us another way to view the $r$-core of a partition $\lambda$.  Since removing an $r$-border strip corresponds to moving a bead $r$ spots to the right to an empty position, on the $r$ abacus this corresponds to moving a bead one step to the right on its runner.  Thus, 


\begin{lemma}
We have
$$|\lambda|=|\core_r{\lambda}|+r|\quot_r{\lambda}|$$
$$\cdim^+_{-1/r}(\lambda)=|\quot_r{\lambda}|$$
\end{lemma}

\begin{proof}
To form the core of $\lambda$ from $\lambda$, we slide the beads on each runner of the abacus.  Each time we slide a bead one step, we are removing an $r$-strip from $\lambda$, and hence decreasing the size of $\lambda$ by one.  On the other hand, we are removing a single cell from $\tilde{\lambda}$.


\end{proof}


\begin{corollary} \label{cor:product-formulas-traditional-cores}

Let $\mathcal{C}_r$ denote the set of $r$-core partitions.  We have
$$\sum_{\lambda\in\mathcal{C}_a} q^{|\lambda|}=\frac{(q^r;q^r)_\infty^r}{(q;q)_\infty}$$
Moreover, we have 
$$\sum_{\lambda\in\mathcal{P}} q^{|\lambda|} t^{\cdim^+{-1/r}(\lambda)}=\frac{(q^r;q^r)_\infty^r}{(q;q)_\infty}\frac{1}{(q^rt;q^rt)_\infty^r}$$
And hence:
$$P_t(\Hilb_{[\C^2/G]})=\frac{(q^r;q^r)_\infty^r}{(q;q)_\infty}\frac{1}{(q^rt^2;q^rt^2)_\infty^{r-1}(q^r;q^rt^2)_\infty}$$
\end{corollary}





The process is reversible -- if we have $a$-different Maya diagrams whose charge sums to $0$, we may interleave the beads of them together as in Figure \ref{ }, and merge them to get a partition.  


The $a$-quotient of $\lambda$ records information about the $a$-hooks of $\lambda$.  If we have a cell $\square\in\lambda$,  with hooklength $h(\square)=ka$ divisible by $a$, then the two energy levels of $\textrm{inversion}(\square)$ lie on the same runner.  Similarly, any inversion of energy states on the same runner corresponds to a cell with hook length divisible by $a$.  

Thus, $\lambda$ is an $a$-core if and only if the beads on each runner of the $a$-abacus are right justified, that is, if the $a$-quotient of $\lambda$ is zero.  Moving a bead on the $a$-abacus, corresponds to removing a border strip of length $a$, and so sliding all the beads on each runner all the way to the right corresponds to removing all $a$-hooks in $\lambda$.  We see that the order we make the moves doesn't matter, as the end result will always be the same.

  Although the total charge of all the runners must be zero, the charge need not be evenly divided among the runners.  Let
$c_i$ be the charge on the $i$th runner; then we have $\sum c_i=0$, and the $c_i$ determine $\lambda$.

Similarly, given any $\mathbf{c}=(c_0,\dots,c_{a-1})\in\Z^a$ with $\sum c_i=0$, there is a unique right justified abacus with charge  $c_i$ on the $i$th runner.  The coresponding partition is an $a$-core which we denote $\core_a(\mathbf{c})$.

We have shown:

\begin{lemma}
There is a bijection $$\core_a:\{(c_0,\dots,c_{a-1}|c_i\in\Z, \sum c_i=0\}\to \{\lambda | \lambda \text{ is in $a$-core} \}$$
\end{lemma}


\begin{example}
We illustrate that $\core_3(0,3, -3)=7+5+3+3+2+2+1+1$.  The numbers on the boundary path illustrate which runner of the abacus that step belongs to.

\begin{center}
\begin{tikzpicture}

\begin{scope} %drawing of original partition
\draw (0,5) node{$\lambda=10+8+8+4+4+3+1+1$};


   \begin{scope}[gray, very thin, scale=.4]
      \clip (-11.5, 11.5) rectangle (11.5, -1);
      \draw[rotate=45, scale=1.412, gray!20!white, fill] (0,0)--(0,10)--(1,10)--(1,8)--(3,8)--(3,4)--(5,4)--(5,3)--(6,3)--(6,1)--(8,1)--(8,0)--cycle;
      \begin{scope}[rotate=45, scale=1.412, black!50!green] 
         \draw (.5,6.5)node{$\spadesuit$};
         \draw (1.5,6.5)node{$\spadesuit$};
      \end{scope}

      \begin{scope}[rotate=45, scale=1.412, blue] 
         \draw (2.5,5.5)node{$\clubsuit$};
         \draw (2.5,2.5)node{$\clubsuit$};
         \draw (4.5,2.5)node{$\clubsuit$};
      \end{scope}


      \draw[rotate=45, scale=1.412] (0,0) grid (12,12);
   \end{scope}

   \begin{scope}[rotate=45, ultra thick, scale=.4*1.412]
      \begin{scope}[blue]
         \draw (1,9.5) node{$1$};
         \draw (2.5,8)node{$1$};
         \draw (3,5.5)node{$1$};
         \draw (4.5,4)node{$1$};
         \draw (6,2.5)node{$1$};
         \draw (7.5,1)node{$1$};
         \draw (9.5,0)node{$1$};
      \end{scope}

      \begin{scope}[red]
         \draw (0,10.5)node{$2$};
         \draw (1,8.5)node{$2$};
         \draw (3,7.5)node{$2$};
         \draw (3,4.5)node{$2$};
         \draw (5,3.5)node{$2$};
         \draw (6,1.5)node{$2$};
         \draw (8,.5)node{$2$};
         \draw (10.5,0)node{$2$};
      \end{scope}

      \begin{scope}[black!50!green]
         \draw (.5,10)node{$3$};
         \draw (1.5, 8)node{$3$};
         \draw (3,6.5)node{$3$};
         \draw (3.5,4)node{$3$};
         \draw (5.5,3)node{$3$};
         \draw (6.5,1)node{$3$};
         \draw (8.5,0)node{$3$};
      \end{scope}

   \end{scope}

   \begin{scope}[scale=.4, yshift=-.5cm, blue]
      \draw(-11.5,0) circle(.3);
      \draw(-8.5, 0)  circle (.3);
      \filldraw(-5.5,0)  circle (.3);
      \draw(-2.5,0)  circle (.3);
      \filldraw(.5,0)  circle (.3);
      \draw(3.5,0)  circle (.3);
      \filldraw(6.5,0)  circle (.3);
      \filldraw(9.5,0)  circle (.3);
   \end{scope}

   \begin{scope}[scale=.4, yshift=-1.5cm, red]
      \draw(-10.5,0) circle(.3);
      \draw(-7.5, 0)  circle (.3);
      \draw(-4.5,0)  circle (.3);
      \draw(-1.5,0)  circle (.3);
      \draw(1.5,0)  circle (.3);
      \draw(4.5,0)  circle (.3);
      \draw(7.5,0)  circle (.3);
      \filldraw(10.5,0)  circle (.3);
   \end{scope}

   \begin{scope}[scale=.4, yshift=-2.5cm, black!50!green]
      \filldraw(-9.5,0) circle(.3);
      \filldraw(-6.5, 0)  circle (.3);
      \draw(-3.5,0)  circle (.3);
      \filldraw(-.5,0)  circle (.3);
      \filldraw(2.5,0)  circle (.3);
      \filldraw(5.5,0)  circle (.3);
      \filldraw(8.5,0)  circle (.3);
      \filldraw(11.5,0)  circle (.3);
   \end{scope}

   \begin{scope}[scale=.4]
      \draw (0,-3)--(0,0);
   \end{scope}
\end{scope}

\begin{scope}[yshift=-8cm] %drawing of the core
\draw (0,5) node{$\core_3(\lambda)=7+5+3+3+2+2+1+1$};


   \begin{scope}[gray, very thin, scale=.4]
      \clip (-11.5, 11.5) rectangle (11.5, -1);
      \draw[rotate=45, scale=1.412, gray!20!white, fill] (0,0)--(0,7)--(1,7)--(1,5)--(2,5)--(2,3)--(4,3)--(4,2)--(6,2)--(6,1)--(8,1)--(8,0)--cycle;
      \draw[rotate=45, scale=1.412] (0,0) grid (12,12);
   \end{scope}

   \begin{scope}[rotate=45, ultra thick, scale=.4*1.412]
      \begin{scope}[blue]
         \draw (0,8.5) node{$1$};
         \draw (1,6.5)node{$1$};
         \draw (2,4.5)node{$1$};
         \draw (3.5,3)node{$1$};
         \draw (5.5,2)node{$1$};
         \draw (7.5,1)node{$1$};
         \draw (9.5,0)node{$1$};
      \end{scope}

      \begin{scope}[red]
         \draw (0,10.5)node{$2$};
         \draw (0,7.5)node{$2$};
         \draw (1,5.5)node{$2$};
         \draw (2,3.5)node{$2$};
         \draw (4,2.5)node{$2$};
         \draw (6,1.5)node{$2$};
         \draw (8,.5)node{$2$};
         \draw (10.5,0)node{$2$};
      \end{scope}

      \begin{scope}[black!50!green]
         \draw (0,9.5)node{$3$};
         \draw (0.5, 7)node{$3$};
         \draw (0.5,7)node{$3$};
         \draw (1.5,5)node{$3$};
         \draw (2.5,3)node{$3$};
         \draw (4.5,2)node{$3$};
         \draw (6.5,1)node{$3$};
         \draw (8.5,0)node{$3$};
      \end{scope}

   \end{scope}

   \begin{scope}[scale=.4, yshift=-.5cm, blue]
      \draw(-11.5,0) circle(.3);
      \draw(-8.5, 0)  circle (.3);
      \draw(-5.5,0)  circle (.3);
      \draw(-2.5,0)  circle (.3);
      \filldraw(.5,0)  circle (.3);
      \filldraw(3.5,0)  circle (.3);
      \filldraw(6.5,0)  circle (.3);
      \filldraw(9.5,0)  circle (.3);
   \end{scope}

   \begin{scope}[scale=.4, yshift=-1.5cm, red]
      \draw(-10.5,0) circle(.3);
      \draw(-7.5, 0)  circle (.3);
      \draw(-4.5,0)  circle (.3);
      \draw(-1.5,0)  circle (.3);
      \draw(1.5,0)  circle (.3);
      \draw(4.5,0)  circle (.3);
      \draw(7.5,0)  circle (.3);
      \filldraw(10.5,0)  circle (.3);
   \end{scope}

   \begin{scope}[scale=.4, yshift=-2.5cm, black!50!green]
      \draw(-9.5,0) circle(.3);
      \filldraw(-6.5, 0)  circle (.3);
      \filldraw(-3.5,0)  circle (.3);
      \filldraw(-.5,0)  circle (.3);
      \filldraw(2.5,0)  circle (.3);
      \filldraw(5.5,0)  circle (.3);
      \filldraw(8.5,0)  circle (.3);
      \filldraw(11.5,0)  circle (.3);
   \end{scope}

   \begin{scope}[scale=.4]
      \draw (0,-3)--(0,0);
   \end{scope}

\end{scope}


\begin{scope}[yshift=-12cm] %the quotients

   \begin{scope}[xshift=-4cm] %blue quotient


\draw (0,2) node{$\quot_3^1(\lambda)=2+1$};
      \begin{scope}[gray, very thin, scale=.4]
         \clip (-5, 3.5) rectangle (5, -1);
         \draw[rotate=45, scale=1.412, gray!20!white, fill] (0,0)--(0,2)--(1,2)--(1,1)--(2,1)--(2,0)--cycle;
         \draw[rotate=45, scale=1.412] (0,0) grid (4,4);
      \end{scope}

      \begin{scope}[scale=.4, yshift=-.5cm, blue]
         \draw(-3.5,0) circle(.3);
         \draw(-2.5, 0)  circle (.3);
         \filldraw(-1.5,0)  circle (.3);
         \draw(-.5,0)  circle (.3);
         \filldraw(.5,0)  circle (.3);
         \draw(1.5,0)  circle (.3);
         \filldraw(2.5,0)  circle (.3);
         \filldraw(3.5,0)  circle (.3);
      \end{scope}

   \end{scope}



   \begin{scope} %red quotient
\draw (0,2) node{$\quot_3^2(\lambda)=\emptyset$};
      \begin{scope}[gray, very thin, scale=.4]
         \clip (-5, 3.5) rectangle (5, -1);
         \draw[rotate=45, scale=1.412] (0,0) grid (4,4);
      \end{scope}

      \begin{scope}[scale=.4, yshift=-.5cm, red]
         \draw(-3.5,0) circle(.3);
         \draw(-2.5, 0)  circle (.3);
         \draw(-1.5,0)  circle (.3);
         \draw(-.5,0)  circle (.3);
         \filldraw(.5,0)  circle (.3);
         \filldraw(1.5,0)  circle (.3);
         \filldraw(2.5,0)  circle (.3);
         \filldraw(3.5,0)  circle (.3);
      \end{scope}

   \end{scope}


   \begin{scope}[xshift=4cm] %green quotient 
\draw (0,2) node{$\quot_3^3(\lambda)=1+1$};

      \begin{scope}[gray, very thin, scale=.4]
         \clip (-5, 3.5) rectangle (5, -1);
         \draw[rotate=45, scale=1.412, gray!20!white, fill] (0,0)--(0,1)--(2,1)--(2,0)--cycle;
         \draw[rotate=45, scale=1.412] (0,0) grid (4,4);
      \end{scope}

      \begin{scope}[scale=.4, yshift=-.5cm, black!50!green]
         \draw(-3.5,0) circle(.3);
         \draw(-2.5, 0)  circle (.3);
         \draw(-1.5,0)  circle (.3);
         \filldraw(-.5,0)  circle (.3);
         \filldraw(.5,0)  circle (.3);
         \draw(1.5,0)  circle (.3);
         \filldraw(2.5,0)  circle (.3);
         \filldraw(3.5,0)  circle (.3);
      \end{scope}

   \end{scope}

\end{scope}


\end{tikzpicture}
\end{center}
\end{example}



\subsection{Leaving and arriving word description}

We now describe a slight shift in perspective on cores that will help when we generalize them.  One way to label the colors of the paths is in terms of the contents of the cells they border.  



\subsection{Size of an \texorpdfstring{$a$}{a}-core}
The core and quotient bijection above in particular gives a bijection between $a$-cores and points in a lattice $a-1$ dimensional lattice.  We have parametrized the lattice above in the \emph{charge} coordinates.  


\begin{definition}
The \emph{$r$-charge} of a partition $\lambda$ is the $c^G(\lambda)$ vector in $\Lambda_a=\{(c_i)|\sum c_i=0\}$ where $c_i^G(\lambda)$ is the charge of the $i$th runner of $\lambda$ when written on the $a$-abacus.
\end{definition}




\begin{theorem} \label{thm:quadratic-sl2}
The size of a core is a quadratic function in the charge coordinates;
$$|\core_a(\mathbf{c})|=\frac{a}{2}\sum_{k=0}^{a-1} c_k^2+ kc_k$$
In fact, each colored cell count is a quadratic function in the charge coordinates.
$$|\core_a(\mathbf{c})|^G_i=\frac{1}{2}\sum_{k=0}^{a-1} c_k^2+ \sum_{k>i}c_k$$
\end{theorem}

We are not sure where Theorem \ref{thm:quadratic-sl2} originates; it is proved independently in \cite{GKS, DS}.  The proof is not difficult and uses the interaction of Frobenius coordinates and the abacus, see ; the first statement is proved in \cite{jsimultaneous} and is easily extended to the second statement. 


Note that since the quadratic part of $|\core_a(\mathbf{c})|^G_i$ is independent of $i$, the quadratic part is adding a number of copies of the regular representation.  Hence, Theorem \ref{thm:quadratic-sl2} implies:




\begin{corollary} 
There is a linear relation between the charge functions $c_i(\lambda)$ the class $[\lambda]^G$.
\end{corollary}

In fact, an explicit form of this linear change of variables will hold in general, even though



 that there is a linear change of variables between the charge coordinates and the colored box coordinates.  This second fact we will prove this second fact directly in Theorem \ref{ }



\section{Conjectured product formula}


In this section we state the conjectured product formula.  Section \ref{sec:prod-start} gives some examples and describes the general form the product formula takes.  In Section \ref{sec:prod-CR} we give an explicit conjectural formula in terms of Chen-Ruan cohomology, while in Section \ref{sec:prod-comb} we give an explicit conjectural formula relating interchanging arms and co-legs.  Section \ref{sec:prod-equiv} proves the equivalence of the two formulations of the conjecture.


\subsection{Initial observations toward the product formula}
\label{sec:prod-start}

To state our conjectural product formula, we will use the Pochhammer symbol 
$$(a;x)_\infty:=\prod_{\ell\geq 0} (1-ax^\ell)$$
We will also use the following extension of the Pochhammer symbol:
$$(a_1,\dots, a_n; x)_\infty=\prod_{i=1}^n (a_i;x)_\infty$$


\begin{example}{G\"ottsche's formula}
Using the Pochhammer symbol, G\"ottsche's formula becomes:
$$\sum_{n\geq 0} b_k(\Hilb_n(S))t^kq^n=\frac{(-qt;qt^2)_\infty^{b_1}(-qt^3;qt^2)_\infty^{b_3}}{(q;qt^2)_\infty^{b_0}(qt^2;qt^2)_\infty^{b_2}(qt^4;qt^2)_\infty^{b_4}}$$
\end{example}



Using the Bia\l ynicki-Birula calculation of Section \ref{ }, it is easy to write code to compute the generating function; see for instance the Sage code at \cite{ }.  Having gathered this data, one can then take log of the generating function and begin to look for product formulas.  For a given group $G$, such formulas are readily found, the difficulty is then finding a general pattern.  We begin with some initial observations in this direction.  

\begin{example}{GLM's $\Z_3$ example}
In \cite{GLM}, Gusein-Zade, Luengo, Melle-Hern\'andez conjectured
$$\DC_{1/3}=\frac{1}{(1-q)}\frac{1}{(1-qt^2)}\frac{1}{(1-q^3)}\frac{1}{(1-q^4t^2)}\frac{1}{(1-q^5t^4)}\frac{1}{(1-q^6t^2)}\cdots$$
Using the Pochhammer symbol, this becomes
\begin{equation*}
\begin{split}
\DC_{1/3}&=\frac{1}{(q;t^2q^3)_\infty}\frac{1}{(q^2t^2;t^2q^3)_\infty}\frac{1}{(q^3;t^2q^3)_\infty} \\
&=\frac{1}{(q,q^2t^2,q^3;t^2q^3)_\infty}
\end{split}
\end{equation*}
\end{example}

\begin{example}{$\Z_5$ actions}
For $\Z_5$, there are two groups not contained in $SL_2$.  The diagonal action has a similar form to the diagonal $\Z_3$:

$$\DC_{1/5}=\frac{1}{(q,q^2t^2,q^3,q^4t^2, q^5; q^5t^2)_\infty}$$
That is, if we set $t=-1$, we obtain the usual Euler product.  The term $q^n$ term in the product occurs with coefficient $t^{2w(n)}$, where $w(n)=\lfloor n/5\rfloor + \epsilon_{1/5}(n)$, and 

$$\epsilon_{1/5}(n)=\left\{\begin{array}{ll} 0 & n\cong 0,1,3 \pmod 5 \\ 1 & n\cong 2,4 \pmod 5\end{array}\right.$$

For the other action, we have:
$$\DC_{2/5}=\frac{1}{(q,q^2,q^3t^2,q^4t^2, q^5; q^5t^2)_\infty}$$
Equivalently, the term $q^n$ term in the product occurs with coefficient $t^{2w(n)}$, where $w(n)=\lfloor n/5\rfloor + \epsilon_{2/5}(n)$, and 

$$\epsilon_{2/5}(n)=\left\{\begin{array}{ll} 0 & n\cong 0,1,2 \pmod 5 \\ 1 & n\cong 3,4 \pmod 5\end{array}\right.$$

\end{example}

\begin{example}{Diagonal $\Z_4$-conjecture of GLM}
In the previous two examples the intersection of $G$ with $SL_2$ was trivial; while in Section \ref{ } we saw how cores and quotients gave a product formula for the case when $G\subset SL_2$.  When $G$ is not contained in $SL_2$ but has a nontrivial intersection, the generating function $\DC_G$ appears to contain elements from both.

When $\Z_4$ acts diagonally, Gusein-Zade, Luengo, Melle-Hern\'andez made the following conjecture:

$$\DC_{1/4}=\frac{(q^2;q^2)^2_\infty}{(q;q^2)_\infty} \frac{1}{(q^2, t^2q^2, q^4, t^2q^4;t^2q^4)_\infty}$$

\end{example}






\subsubsection{Summary}
It seems in general that if $G\cap SL_2=\{1\}$ then $\DC_G$ has a product formula that is a simple $t$-deformation of the usual Euler-product.  More specificly, it seems that
$$\DC_{G}=\prod_{h=1}^r \frac{1}{(q^h t^{\epsilon(h)}; q^r t^2)_\infty}$$
with $\epsilon(h)$ either 0 or 2; the question is then to describe $\epsilon(h)$.

When $G\cap SL_2=\Z_k$, then it appears that $\DC_G$ is a $t$-deformation of the usual $k$-core and quotient product formula, with the $t$-terms only affecting the $k$-quotient terms.  More specifically, it seems that
$$\DC_G=\frac{(q^k;q^k)^k_\infty}{(q;q)_\infty}
\prod_{m=1}^{r/k}\frac{1}{(q^{km}t^{\epsilon_{1,m}},q^{km}t^{\epsilon_{2,m}},\dots,q^{km}t^{\epsilon_{k,m}} ;q^rt^2))}$$
with $\epsilon_{i,j}$ either 0 or 2; the question is then to describe $\epsilon_{i,j}$.


In G\"ottsche's formula, terms with $\epsilon(h)=0$ correspond to elements of $J_0(S)$, and terms with $\epsilon(h)=2$ corresponds to elements of $H_2(S)$. Thus, one might hope for a description of the $\epsilon(h)$ in terms of the cohomology of the stack $[\C^2/G]$.  The correct cohomology theory to use appears to be Chen-Ruan cohomology.

\subsection{Product formula: geometry}
\label{sec:prod-CR}

\subsubsection{Chen-Ruan cohomology}

The Chen-Ruan cohomology $H_{CR}^*(\mathcal{X})$ of an orbifold $\mathcal{X}$ was discovered as a byproduct of defining the quantum cohomology of such orbifolds \cite{chenruancohomology}.  As a vector space, the Chen-Ruan cohomology is the usual cohomology of the inertial orbifold $\mathcal{IX}$ of $\mathcal{X}$.  As a set $\mathcal{IX}$ is the space of constant maps from $S^1$ to $\mathcal{X}$; more algebraically, 
$$\mathcal{IX}=\{(x, (g)| x\in \mathcal{X}, (g)\in\textrm{conj}(G_x)\}$$
Every isotropy group $G_x$ has an identity element $e_x$, the subset of $(x, e_x)$ naturally forms a copy of $X$.  Elements $(x,(g))$ with $(g)$ nontrivial form other components of $\mathcal{IX}$ called \emph{twisted sectors}.

Thus, for $\mathcal{X}=[\C^2/G]$, we see that 
$$H_{CR}^*(\mathcal{X}) =H^*(\mathcal{IX})=\oplus_{g\in\textrm{conj}(G)} (\C^2)^g$$
has a basis indexed by conjugacy classes of $G$.

However, Chen-Ruan cohomology has a different product and grading than $H^*(\mathcal{IX})$.  Each twisted sector has a a degree shifting number $\iota(g)$, obtained as follows.

Since $g\in G_x$, and $g$ acts on $T_x$.  It acts trivially on the tangent directions to fix($g$), and nontrivial on the normal directions.  Diagonalizing the action on the normal bundle, we see that $g$ is a diagonal matrix with entries 
$$(\exp(2\pi i a_1/r), \exp(2\pi i a_2/r), \cdots \exp(2\pi i a_m))$$
Then $\iota(g)=\sum a_i/r$; sometimes called the logarithmic trace of $g$.

The grading shift number is in general only a rational number -- $\iota(g)\in \Z$ if and only if the determinant of $g$ is trivial.  Thus, in case all isotropy groups $G_x$ are in $SL_2$, then the Chen-Ruan cohomology is integrally graded.

\begin{example}[Antidiagonal action]
When $G\subset SL_2$, we have that 
$$\iota(g)=\left\{\begin{array}{rl} 0 & g=0 \\
1 & g\neq 0 \end{array}\right.
$$

Thus, the dimension of $H_{CR}^*([\C^2/G])$ is equal to the number of conjugacy classes of $G$ (which is the number of irreps of $G$.  We have $H_{CR}^0$ is one dimensional, with the rest of the classes being two dimensional.  Thus we have

$$H_{CR}^*([\C^2/G])\cong H^*(\widetilde{S}_G)$$
as graded vector spaces (actually, as rings in this case).  
\end{example}


\begin{theorem}[Yasuda \cite{yasuda}] \label{thm:yasuda}
Let $\mathcal{X}$ an effective orbifold, and $\widetilde{X}\to|\mathcal{X}|$ a crepant resolution of the coarse moduli space.  Then :
$$H_{CR}^*(\mathcal{X})=H^*(\widetilde{X})$$
as graded vector spaces.  
\end{theorem}

Note that the products do not necessarily agree; quantumn corrections are needed; the \emph{Crepant Resolution Conjecture} states that, properly understood, the quantumn cohomology of $\mathcal{X}$ and $\widetilde{X}$ should agree.  See \cite{CoatesRuan} for discussion of the details and reference to other sources.


\begin{example}
Let $\Z_r$ act diagonally, with element $g_k$ acting as $(\exp(k2\pi i/r),\exp(k2\pi i/r))$.  Then we have $\iota(g_k)=2k/r$.
\end{example}




The Chen-Ruan cohomology of $[\C^2/G]$ is rationally graded, with $d$ with $0\leq d < 4$


$$0\to G\cap SL_2\to G \to \C^*$$








Our conjectural product formula for $\DC_G$ is easiest to state in case $G\cap SL_2=1$.  In this case, the action of $G$ on $\wedge T^*\C^2$ is faithful; taking $r$ times the logarithmic trace of this action gives a bijection between $G$ and $\{0,\dots, r-1\}$.


Let $F(g)$ and $I(g)$ denote the fractional and integral parts of $\iota(g)$. 

If $G\cap SL_2=\{1\}$, then $F(G)$ gives a bijection between $G$ and $\{0, 1/r,\dots, (r-1)/r\}$.
\begin{conjecture}[Johnson] \label{conj:CR}
Let $G$ be cyclic, and define $k=|G\cap SL_2|$

$$\mathcal{H}_G(q,t)= \frac{(q^k;q^k)^k_\infty}{(q,q)_\infty} \prod_{g\in G}\frac{1}{(q^{r(1-F(g))} t^{2I(g)},q^rt^2)_\infty}$$

\end{conjecture}


\begin{example}[Diagonal action, $r$-odd]
Let $r=2k+1$.   The element of $\Z_r$ that acts on $K$ as $\exp(2\pi i/r)$ acts 1on the tangent space as $\exp(2\pi i k/r)$, and thus as $\iota=2k/(2k+1)<1$.  

Odd powers of this element will have $\iota<1$, while even powers will have $\iota>1$, giving 

In the limit as $r\to\infty$ odd, we get Buryak-Feigin's result

$$\prod_{n \text{ odd}} \frac{1}{1-q^n}\prod_{m \text{ even}}\frac{1}{1-tq^m}$$
\end{example}


\begin{example}[$S_{2/5}$]
We have
$$\begin{array}{r|c|l}
g & r(1-F(g)) & I(g) \\
\hline
1 & 2 & 0 \\
2 & 4 & 1 \\
3 & 1 & 0 \\
4 & 3 & 1 \\
\end{array}
$$

Thus, we have


$$
\frac{1}{(q, q^2,q^3t, q^4t, q^5t^2; q^5t^2)_\infty}
$$
\end{example}


\subsection{Product formula: combinatorics}
\label{sec:prod-comb}

We now give a different description of the product formula, at least for the case when $G\subset SL_2=1$, with motivation coming from combinatorics.  In particular, our conjecture is a strengthening of results of Bacher-Manivel \cite{BM} and Bessenrodt \cite{Bess}, which show that the total number cells in all partitions of $n$ with arm length $A$ and leg length $L$ is equal to the total number of parts of size $A+L+1$ in partitions of $n$.

\subsubsection{ }

The combinatorial motivation for the distribution of $\cdim_{k/r}(\lambda)$ is to define a new statistic, $\dusty_{k/r}$ that simply replaces arm with the co-leg.
\begin{definition}
$$\dusty_{k/r}(\lambda)=\#\left\{\square\in\lambda\Big| \ell(\square)-k\coleg(\square)=-1\mod r\right\}$$
\end{definition}

The immediate benefit of this definition is that this both $\ell(\square)$ and $\coleg(square)$ depend only on the row $\square$ is in, and not on the global shape of the partition.  Thus, viewing the rows of the partition as the parts of the partition, it is immediate that the $(q,t)$-enumeration of partitions has a product formula that is just a $t$-deformation of the standard Euler product, that is, we have

$$\sum_{\lambda\in \PP} q^{|\lambda|} t^{|\dusty_{k/r}(\lambda)|}=\prod_{m\geq 1} \frac{1}{1-q^mt^{z_{k/r}(m)}}$$
for some function $z(m)$.

It is an easy observation that when $k+1$ is relatively prime to $r$, then $z_{k/r}(m)$ has the same basic structure as the weights $w_{k/r}(m)$ appearing in the distribution of 

\begin{lemma} \label{lem:dusty-lemma}
If $(k+1,r)=1$, then 
$$w_{k/r}(m)=\varepsilon_{k/r}(m)+\lfloor m/r\rfloor$$
where $\varepsilon_{k/r}(m)$ is either 0 or 1 and only depends on $m\mod r$.
\end{lemma}

\begin{proof}
This follows from the fact that, as we move one box to the left in a row, $\ell(\square)-k\coleg(\square)$ increases by $k+1$.  Thus, when $k+1$ is relatively prime to $r$, in any string of $r$ boxes in a single row, $\ell(\square)-k\coleg(\square)$ will take on every residue class $\mod r$ exactly once. 
\end{proof}

\begin{conjecture} \label{conj:dusty}
$$\sum_{\lambda\in \PP} q^{|\lambda|} t^{|\dusty_{k/r}(\lambda)|}=\sum_{\lambda\in \PP} q^{|\lambda|} t^{|\cdim_{k/r}(\lambda)|}$$
\end{conjecture}


\begin{example}
Consider $\dusty_{1/r}$, for $r$ 

\end{example}


\begin{proposition}
If we take the first derivative of Conjecture \ref{conj:dusty} with respect to $t$, and then set $t=1$, then the resulting equation is true.  
\end{proposition}

\begin{proof}
First, consider the left hand side; $\sum_{\lambda\in\PP} q^{|\lambda|} t^{\cdim(\lambda)}$.  If we take the derivative of $t$ and set $t=1$, we get 
$$\sum_{n} q^n\sum_{\lambda\in\PP_n}\sum_{\square\in\lambda} \delta(\ell(\square)-k\coleg(\square)=-1\mod r),$$
that is, the coefficient of $q^n$ is the total number of squares in all partions of $n$ that have $\ell(\square)-k\coleg(square)=-1\mod r$.  

  Here, a part of length $L$ will contribute 

Similarly, when we take the derivative with respect to $t$ and set $t=1$ on the right hand side, the coefficient of $q^n$ is the total number of squares in all partitions of $n$ with $\ell(\square)-k a(\square)=-1\mod r$. 

For any possible non-negative values $L$ and $A$ satisfying $L-kA=-1\mod r$, we can count the number of squares $\square$ in all partitions of $n$ with $a(\square)=A$ and $\ell(\square)=L$.  By the results of Bacher Manivel and Bessenrodt, this is equal to the number of all parts of length $A+L+1$ in all partitions of $n$.  

But for any such values of $L$ and $A$, in every part of size $A+L+1$ there will be a unique $\square$ with $\coleg(\square)=A$ and $\ell(\square)=L$, and and this square will contribute to the left hand side.


\end{proof}



\subsection{Equivalence of the two product formulas}
\label{sec:prod-equiv}

\begin{proposition}
Conjecture \ref{conj:CR} is equivalent to Conjecture \ref{conj:dusty}
\end{proposition}

\begin{proof}
By Lemma \ref{prop:dusty-lemma}, we only need to prove that the powers of $t$ in the two conjectural product formulas agree for the first $r$ terms; that is, we need to show that $\epsilon_{k/r}(m)=\varepsilon_{k/r}(m)$ for $1\leq m\leq r$.  

Now, for $\varepsilon_{k/r}(m)=1$, we need 
\begin{align*}
0 &\in \{m, m-(k+1),\dots, m-(m-1)(k+1) \} \\
-m &\in \{0, -(k+1),\dots,-(m-1)(k+1) \} \\
\frac{-m}{k+1} &\in \{0, -1, \dots, -(L-1) \} \\
\end{align*}
Where we have worked in $\mathbb{Z}_r$.  Equivalently, if we choose the representative of $[-m/(k+1)]\in \mathbb{Z}/r\mathbb{Z}$ to be in in $\{0,\dots, r-1\}$, then we need $-m/(k+1)>-m$.

From the Chen-Ruan point of view, a part of length $m<r$, with $ m\cong-(k+1)s\mod r$ will contribute if we have to carry when we add $s$ and $ks$, that is 
\begin{align*}
k s &\in\{ -1,\dots, -s \}\\
(k+1)s &\in\{ 0,\dots, s-1\} \\
m&\in\{ 0,-1,\dots, -s+1\} \\
m& \in \{0, -1,\dots, m/(k+1)+1\}
\end{align*}
That is, we need $m>m/(k+1)$, if we choose the representative of $m/(k+1)\in\Z_r$ to be $<r$.

\end{proof}


\section{Generalized cores and applications}

This section introduces \emph{rational cores and quotients}, amd applies them to the study of $|\lambda|_G$ and $\cdim(\lambda)$.  In particular, we prove two results: the topology of $\Hilb_v$ stabilizes, and that the size of $k/r$-cores is piecewise quadratic.

The basic construction of $k/r$-cores and quotients is the $k/r$-abacus.  This parallels the usual abacus construction of regular cores and quotients, but is more complicated: rather than regularly cycling through the $r$ runners, the runners the electrons/beads are placed on depends on whether the previous state was filled with an electron or left empty.  

The construction is also essentially the arriving/departing word construction of \cite{LW}.  We begin by making this explicit, and explaining usual $r$-cores in terms of leaving words.


\subsection{The regular abacus in terms of leaving words} \label{sec:leaving-words}

To conne




The construction of $G$-cores and quotients is exactly parallel to the 


\begin{center}
\begin{tikzpicture}

\begin{scope} %drawing of original partition
\draw (0,5) node{$\lambda=$};
\begin{scope}[scale=.4]

\begin{scope}[rotate=45, scale=1.412]
\clip (0,0)--(13,0)--(0,13)--cycle;
\foreach \x in {0,...,13}{%
    \foreach \y in {0,...,4}{%
       \draw[fill=green!7]  (\x, -\x+3*\y) rectangle (\x+1, -\x+3*\y+1); 
       \draw[fill=blue!7]  (\x, -\x+3*\y+1) rectangle (\x+1, -\x+3*\y+2); 
       \draw[fill=red!7]  (\x, -\x+3*\y+2) rectangle (\x+1, -\x+3*\y+3); 

}}

\begin{scope}
    \clip (0,0)--(0,10)--(1,10)--(1,9)--(3,9)--(3,5)--(4,5)--(4,3)--(6,3)--(6,2)--(7,2)--(7,1)--(8,1)--(8,0)--cycle;
\foreach \x in {0,...,13}{%
    \foreach \y in {0,...,4}{%
       \draw[fill=green!23]  (\x, -\x+3*\y) rectangle (\x+1, -\x+3*\y+1); 
       \draw[fill=blue!23]  (\x, -\x+3*\y+1) rectangle (\x+1, -\x+3*\y+2); 
       \draw[fill=red!23]  (\x, -\x+3*\y+2) rectangle (\x+1, -\x+3*\y+3); 

}}
\end{scope}


      \draw[gray, very thin] (0,0) grid (13,13);
\end{scope}


      \begin{scope}[rotate=45, scale=1.412, black!50!green] 
         \draw (.5,6.5)node{$\spadesuit$};
         \draw (1.5,6.5)node{$\spadesuit$};
      \end{scope}

      \begin{scope}[rotate=45, scale=1.412, blue] 
         \draw (2.5,7.5)node{$\clubsuit$};
         \draw (2.5,3.5)node{$\clubsuit$};
         \draw (3.5,3.5)node{$\clubsuit$};
      \end{scope}



   \end{scope}

   \begin{scope}[rotate=45, ultra thick, scale=.4*1.412]
      \begin{scope}[blue]
         \draw (0,10.5) node{$1$};
         \draw (1,9.5)node{$1$};
         \draw (2.5,9)node{$1$};
         \draw (3,7.5)node{$1$};
         \draw (3.5,5)node{$1$};
         \draw (4,3.5)node{$1$};
         \draw (5.5,3)node{$1$};
         \draw (6.5,2)node{$1$};
         \draw (7.5,1)node{$1$};
         \draw (8.5,0)node{$1$};
         \draw (11.5,0)node{$1$};
      \end{scope}

      \begin{scope}[red]
         \draw (0,11.5)node{$2$};
         \draw (3,8.5)node{$2$};
         \draw (3,5.5)node{$2$};
         \draw (4,4.5)node{$2$};
         \draw (6,2.5)node{$2$};
         \draw (7,1.5)node{$2$};
         \draw (8,.5)node{$2$};
         \draw (9.5,0)node{$2$};
      \end{scope}

      \begin{scope}[black!50!green]
         \draw (.5,10)node{$3$};
         \draw (1.5, 9)node{$3$};
         \draw (3,6.5)node{$3$};
         \draw (4.5,3)node{$3$};
         \draw (10.5,0)node{$3$};

      \end{scope}

   \end{scope}

   \begin{scope}[scale=.4, yshift=-.5cm, blue]
      \draw(-10.5,0) circle(.3);
      \draw(-8.5, 0)  circle (.3);
      \filldraw(-6.5,0)  circle (.3);
      \draw(-4.5,0)  circle (.3);
      \filldraw(-1.5,0)  circle (.3);
      \draw(.5,0)  circle (.3);
      \filldraw(2.5,0)  circle (.3);
      \filldraw(4.5,0)  circle (.3);
      \filldraw(6.5,0)  circle (.3);
      \filldraw(8.5,0)  circle (.3);
      \filldraw(11.5,0)  circle (.3);
   \end{scope}

   \begin{scope}[scale=.4, yshift=-1.5cm, red]
      \draw(-11.5,0) circle(.3);
      \draw(-5.5, 0)  circle (.3);
      \draw(-2.5,0)  circle (.3);
      \draw(-.5,0)  circle (.3);
      \draw(3.5,0)  circle (.3);
      \draw(5.5,0)  circle (.3);
      \draw(7.5,0)  circle (.3);
      \filldraw(9.5,0)  circle (.3);
      \filldraw(12.5,0)  circle (.3);
   \end{scope}

   \begin{scope}[scale=.4, yshift=-2.5cm, black!50!green]
      \draw(-12.5,0)  circle (.3);
      \filldraw(-9.5,0) circle(.3);
      \filldraw(-7.5, 0)  circle (.3);
      \draw(-3.5,0)  circle (.3);
      \filldraw(1.5,0)  circle (.3);
      \filldraw(10.5,0)  circle (.3);
   \end{scope}

   \begin{scope}[scale=.4]
      \draw (0,-3)--(0,0);
   \end{scope}
\end{scope}

\begin{scope}[yshift=-8cm] %drawing of the core
\draw (0,5) node{$\core_3(\lambda)=7+5+3+3+2+2+1+1$};



\begin{scope}[rotate=45, scale=.4*1.412]

\clip (0,0)--(13,0)--(0,13)--cycle;
\foreach \x in {0,...,13}{%
    \foreach \y in {0,...,4}{%
       \draw[fill=green!7]  (\x, -\x+3*\y) rectangle (\x+1, -\x+3*\y+1); 
       \draw[fill=blue!7]  (\x, -\x+3*\y+1) rectangle (\x+1, -\x+3*\y+2); 
       \draw[fill=red!7]  (\x, -\x+3*\y+2) rectangle (\x+1, -\x+3*\y+3); 
}}

\begin{scope}
    \clip (0,0)--(0,7)--(1,7)--(1,6)--(2,6)--(2,5)--(4,5)--(4,4)--(5,4)--(5,3)--(6,3)--(6,2)--(7,2)--(7,1)--(8,1)--(8,0)--cycle;
\foreach \x in {0,...,13}{%
    \foreach \y in {0,...,4}{%
       \draw[fill=green!23]  (\x, -\x+3*\y) rectangle (\x+1, -\x+3*\y+1); 
       \draw[fill=blue!23]  (\x, -\x+3*\y+1) rectangle (\x+1, -\x+3*\y+2); 
       \draw[fill=red!23]  (\x, -\x+3*\y+2) rectangle (\x+1, -\x+3*\y+3); 
}}
\end{scope}


      \draw[gray, very thin] (0,0) grid (13,13);
\end{scope}


   \begin{scope}[rotate=45, ultra thick, scale=.4*1.412]
      \begin{scope}[blue]
         \draw (0,10.5) node{$1$};
         \draw (0,7.5)node{$1$};
         \draw (1,6.5)node{$1$};
         \draw (2,5.5)node{$1$};
         \draw (3.5,5)node{$1$};
         \draw (4.5,4)node{$1$};
         \draw (5.5,3)node{$1$};
         \draw (6.5,2)node{$1$};
         \filldraw (7.5,1)node{$1$};
         \filldraw (8.5,0)node{$1$};
         \filldraw (11.5,0)node{$1$};
      \end{scope}

      \begin{scope}[red]
         \draw (0,11.5)node{$2$};
         \draw (0,8.5)node{$2$};
         \draw (4, 4.5)node{$2$};
         \draw (5, 3.5)node{$2$};
         \draw (6, 2.5)node{$2$};
         \draw (7, 1.5)node{$2$};
         \draw (8,.5)node{$2$};
         \draw (9.5,0)node{$2$};
      \end{scope}

      \begin{scope}[black!50!green]
         \draw (0,9.5)node{$3$};
         \draw (0.5, 7)node{$3$};
         \draw (1.5,6)node{$3$};
         \draw (2.5,5)node{$3$};
         \draw (10.5,0)node{$3$};
      \end{scope}

   \end{scope}

   \begin{scope}[scale=.4, yshift=-.5cm, blue]
      \draw(-10.5,0) circle(.3);
      \draw(-7.5, 0)  circle (.3);
      \draw(-5.5,0)  circle (.3);
      \draw(-3.5,0)  circle (.3);
      \filldraw(-1.5,0)  circle (.3);
      \filldraw(.5,0)  circle (.3);
      \filldraw(2.5,0)  circle (.3);
      \filldraw(4.5,0)  circle (.3);
      \filldraw(6.5,0)  circle (.3);
      \filldraw(8.5,0)  circle (.3);
      \filldraw(11.5,0)  circle (.3);
   \end{scope}

   \begin{scope}[scale=.4, yshift=-1.5cm, red]
      \draw(-11.5,0) circle(.3);
      \draw(-8.5, 0)  circle (.3);
      \draw(-.5,0)  circle (.3);
      \draw(1.5,0)  circle (.3);
      \draw(3.5,0)  circle (.3);
      \draw(5.5,0)  circle (.3);
      \draw(7.5,0)  circle (.3);
      \filldraw(9.5,0)  circle (.3);
      \filldraw(12.5,0)  circle (.3);
   \end{scope}

   \begin{scope}[scale=.4, yshift=-2.5cm, black!50!green]
          \draw(-12.5,0) circle(.3);
      \draw(-9.5,0) circle(.3);
      \filldraw(-6.5, 0)  circle (.3);
      \filldraw(-4.5,0)  circle (.3);
      \filldraw(-2.5,0)  circle (.3);
       \filldraw(10.5,0)  circle (.3);
   \end{scope}

   \begin{scope}[scale=.4]
      \draw (0,-3)--(0,0);
   \end{scope}

\end{scope}


\begin{scope}[yshift=-12cm] %the quotients

   \begin{scope}[xshift=-4cm] %blue quotient


\draw (0,2) node{$\quot_3^1(\lambda)=2+1$};
      \begin{scope}[gray, very thin, scale=.4]
         \clip (-5, 3.5) rectangle (5, -1);
         \draw[rotate=45, scale=1.412, gray!20!white, fill] (0,0)--(0,2)--(1,2)--(1,1)--(2,1)--(2,0)--cycle;
         \draw[rotate=45, scale=1.412] (0,0) grid (4,4);
      \end{scope}

      \begin{scope}[scale=.4, yshift=-.5cm, blue]
         \draw(-3.5,0) circle(.3);
         \draw(-2.5, 0)  circle (.3);
         \filldraw(-1.5,0)  circle (.3);
         \draw(-.5,0)  circle (.3);
         \filldraw(.5,0)  circle (.3);
         \draw(1.5,0)  circle (.3);
         \filldraw(2.5,0)  circle (.3);
         \filldraw(3.5,0)  circle (.3);
      \end{scope}

   \end{scope}



   \begin{scope} %red quotient
\draw (0,2) node{$\quot_3^2(\lambda)=\emptyset$};
      \begin{scope}[gray, very thin, scale=.4]
         \clip (-5, 3.5) rectangle (5, -1);
         \draw[rotate=45, scale=1.412] (0,0) grid (4,4);
      \end{scope}

      \begin{scope}[scale=.4, yshift=-.5cm, red]
         \draw(-3.5,0) circle(.3);
         \draw(-2.5, 0)  circle (.3);
         \draw(-1.5,0)  circle (.3);
         \draw(-.5,0)  circle (.3);
         \filldraw(.5,0)  circle (.3);
         \filldraw(1.5,0)  circle (.3);
         \filldraw(2.5,0)  circle (.3);
         \filldraw(3.5,0)  circle (.3);
      \end{scope}

   \end{scope}


   \begin{scope}[xshift=4cm] %green quotient 
\draw (0,2) node{$\quot_3^3(\lambda)=1+1$};

      \begin{scope}[gray, very thin, scale=.4]
         \clip (-5, 3.5) rectangle (5, -1);
         \draw[rotate=45, scale=1.412, gray!20!white, fill] (0,0)--(0,1)--(2,1)--(2,0)--cycle;
         \draw[rotate=45, scale=1.412] (0,0) grid (4,4);
      \end{scope}

      \begin{scope}[scale=.4, yshift=-.5cm, black!50!green]
         \draw(-3.5,0) circle(.3);
         \draw(-2.5, 0)  circle (.3);
         \draw(-1.5,0)  circle (.3);
         \filldraw(-.5,0)  circle (.3);
         \filldraw(.5,0)  circle (.3);
         \draw(1.5,0)  circle (.3);
         \filldraw(2.5,0)  circle (.3);
         \filldraw(3.5,0)  circle (.3);
      \end{scope}

   \end{scope}

\end{scope}


\end{tikzpicture}
\end{center}



\begin{lemma}
  The map class $[\lambda]^G\in\overline{K}(G)$ depends only on $c^G(\lambda)$.  
More specifically, let $e_i$ be the vector that adds 1 to the $i$th charge, $w_i=e_{i+a}-e_i\in\Lambda^G$ and $f_i\in\overline{K}(G)$ the vector that adds 1 to the $i$th color, and define an isomorphism of lattices $\varphi:\Lambda^G\to\overline{K}(G)$ by $\varphi(w_i)=f_i$.

Then $[\lambda]^G=\varphi(c^G(\lambda))$.
\end{lemma}

\begin{proof}
First, note $\varphi$ is an isomorphism of lattices, as the $w_i$ and $f_i$ generate each lattice subject to the relation $\sum w_i=0, \sum f_i=0$.

It suffices to show that if $\mu$ is obtained from $\lambda$ by removing a single box $\square$, then $[\lambda]^G-[\mu]^G=\varphi(c^G(\lambda)-c^G(\mu)$.  Suppose that $\square$ had color $i$; then in $\lambda$ the border path travels above $\square$ and has color $i$, while in $\mu$ the border path travels below $\square$ and has color $i+a$.  All other up-sloping steps of $\lambda$ and $\mu$ agree.  In removing the box, we deleted a box of color $i$ from $\lambda$, and thus $[\lambda]^G-[\mu]^G=f_i$.  We also removed one bead from the $i$th runner, increasing $c^G_i$ by 1, and added a bead to the $i+a$ runner, decreasing $c^G_{i+a}$ by 1.


\end{proof}


\begin{proposition} \label{prop:connected}
If $G$ is abelian, and $v$ any representation of $G$, then $\Hilb_v([\C^2/G])$ is connected.  
\end{proposition}
\begin{proof}

\end{proof}
For $G=\C^*$, this Theorem was proven by Evain \cite{evain1}.


\section{Riemann-Roch calculation}

\subsection{Riemann-Roch Theorem}
Although the ext groups $\Ext^i(\mathcal{F},\mathcal{G})$ between two sheaves depend delicately on the sheaves involved, the Euler pairing
$$\chi(\mathcal{F},\mathcal{G})=\sum_{i\geq 0} \Ext^i(\mathcal{F},\mathcal{G})$$
only depends upon the $K$-theory classes of $\mathcal{F}$ and $\mathcal{G}$.  

\begin{example}
Let $p$ and $q$ be two points of $\C^2$.  If $p\neq q$, then the structure sheaves $\mathcal{O}_p$ and $\mathcal{O}_q$ have disjoint support, and so $\Ext^i(\mathcal{O}_p,\mathcal{O}_q)=0$ for all $i$.

If $p=q$ we have:
\begin{eqnarray*}
\Ext^0(\mathcal{O}_p,\mathcal{O}_p)&=&1\\
\Ext^1(\mathcal{O}_p,\mathcal{O}_p)&=&2\\
\Ext^2(\mathcal{O}_p,\mathcal{O}_p)&=&1 
\end{eqnarray*}

In both cases, $\chi(\mathcal{O}_p,\mathcal{O}_q)=0$.
\end{example}

The Riemann-Roch formula gives an evaluation of the Euler pairing strictly in terms of the $K$-theory class.  Since our orbifolds are global quotients we can just use an equivariant Riemann-Roch formula, and since there are isolated fixed points it has a particularly nice form

In particular, for $\C^2/G$, the Euler pairing will be equivalent to 

The exterior algebra of $\C^2$ is a graded $G$ representation, with $\bigwedge^0\C^2$ the trivial representation, and $\bigwedge^2\C^2$ in piece 0, and $\bigwedge^1\C^2=\C^2$ in degree 1.  

\begin{definition}
The super McKay pairing is $Q_{SM}(V,W)=\dim (V\otimes W^\vee\otimes\bigwedge^*\C^2)^G$.
\end{definition}


\begin{theorem}[Equivariant Riemann-Roch]
$$\chi(\mathcal{F},\mathcal{G})=Q_{SM}([\mathcal{F}],[\mathcal{G}])$$
\end{theorem}


\subsection{Application to cores}

We now apply the Riemann-Roch theorem to the dimension and $K$-theory classes of cores.  Specifically, we will address the following question: given a class $v\in \overline{K}(G)$, what can we say about the dimension and $K$-theory class of the core Hilbert scheme in class $v$?

First, note that pairing with $[V_G]$ just counts the dimension, and since the super dimension of $\bigwedge^*\C^2$ is zero, it is in the kernel of the McKay pairing.  Geometrically, this corresponds to the fact that adding the structure sheaf of a smooth point on $[\C^2/G]$ will change the $K^0$ theory class by $V_G$, but the smooth point may be added disjoint from the support of the rest of the sheaf and the sheaf it is being paired with and hence not change any of the ext groups.


Thus, the Euler pairing $\chi$ descends to a pairing on $\overline{K}(G)=K^0(G)/V_G$.  Given a class $v\in \overline{K}(G)$, what is the minimum lift to $\tilde{v}\in K^0(G)$ with $[\tilde{v}]=v$ and $\Hilb_{\tilde{v}}$ nonempty?

As vector spaces, $\overline{K}(G)$ is isomorphic to the subset of $K^0(G)$ with entries summing to zero, but quite as lattices.  Therefore for $G$ abelian we introduce the space 
$$S^0_G=\left\{v\in K^0(G; \Z[1/|G|])\Bigg | \sum_{\chi\in\irreps(G)}(\chi, v)=0\right\}$$
and we have an isomorphism $S^0_G$ and $\overline{K}(G)$.  


\begin{lemma}
Let $k$ be the index of $K_{[\C^2/G]}\in K(G)$.

Then
$$\tilde{v}=v+\frac{1}{2}\left(Q_M(v)+\dim\Hilb_{\tilde{v}} -v_0-v_{k}\right)[V_G]$$
\end{lemma}



\begin{proof}
We calculate $\chi(\tilde{v},\tilde{v})$ in two different ways, once from the definition, and once using the Riemann-Roch theorem.

Let $\OO_{\tilde{v}}$ be any quotient sheaf with the right $K$-theory class.  

We first use
$$\chi(\tilde{v}, \tilde{v})=\dim\Ext^0(\OO_{\tilde{v}},\OO_{\tilde{v}})-\dim\Ext^1(\OO_{\tilde{v}},\OO_{\tilde{v}})+\dim\Ext^2(\OO_{\tilde{v}},\OO_{\tilde{v}})$$

By definition $\Ext^0(\OO_{\tilde{v}},\OO_{\tilde{v}})=\Hom_R(R/\II,R/\II)^G$.  Now since $\OO_v$ is a quotient of $R$ it is generated by $1$, and so to define a homomorphism we just need to say where $1$ maps. Since $\II$ is an ideal, it can map to any entry of $R/\II$.  Since one is fixed by $G$, for the homomorphism to be $G$ invariant we must preserve weight and map to something invariant.  So $\dim\Ext^0(\OO_{\tilde{v}},\OO_{\tilde{v}})=\tilde{v}_0$.  

Using Serre-duality, $\dim\Ext^2(\OO_{\tilde{v}},\OO_{\tilde{v}})=\dim\Ext^0(\OO_{\tilde{v}},K\otimes\OO_{\tilde{v}})$.  The effect of tensoring by $K$ is just to change the $G$ action -- an invariant vector of $\OO_{\tilde{v}}\otimes K$ is just a vector that transforms as $K^\vee$ in $\OO_{\tilde{v}}$.  
So $\dim\Ext^2(\OO_{\tilde{v}},\OO_{\tilde{v}})=\tilde{v}_{-k}$.  

Finally $\dim\Ext^1(\OO_{\tilde{v}},\OO_{\tilde{v}})=\dim\Hilb_{\tilde{v}}$.  So $\chi(\tilde{v},\tilde{v})=\tilde{v}_0+\tilde{v}_{-k}-\dim\Hilb_{\tilde{v}}$.  If we write $\tilde{v}=v+M [V_G]$, then $\tilde{v}_i=v_i+M$, and so  
\begin{equation} \label{eq:EXT-evaluation}
\chi(\tilde{v},\tilde{v})=v_o+v_{-k}+2M-\dim\Hilb_{\tilde{v}} 
\end{equation}

On the other hand, using Euler pairing using Riemann-Roch works exactly the same way, and we have
\begin{equation} \label{eq:RR-evaluation}
\chi(\tilde{v},\tilde{v})=Q_{SM}(v+M[V_G],v+M[V_G])=Q_M(v,v)
\end{equation}
using the fact that $[V_G]$ is in the kernel of $Q_{SM}$.  

Setting Equations \ref{eq:EXT-evaluation} and \ref{eq:RR-evaluation} equal and solving for $M$ gives the desired result.

\end{proof}


\begin{corollary}
The map $v\mapsto \dim\Hilb_{\tilde{v}}$ is a piecewise quadratic function.
\end{corollary}

\begin{proof}
Using cores, we have already seen that $\tilde{v}$ is a piecewise quadratic function of $v$.  
\end{proof}


\begin{corollary}  If $G\in SL_2$. Then
$$\tilde{v}=v-v_0+\frac{1}{2}Q_M(v,v)[V_G]$$
In particular, we see $\tilde{v}$ is a quadratic function of $v$.
\end{corollary}

\begin{proof}
When $G\in SL_2$, we the canonical bundle $K_{[\C^2/G]}$ is trivial, and so $k=0$.  Furthermore, when $G\in SL_2$, the core partitions are isolated points and hence have dimension 0.
\end{proof}






\appendix

\section{Review of Ellingsrud-Str\o mme}
This appendix reviews Ellingsrud-Str\o mme's computation of the poincare polynomial of $\Hilb_n(\C^2)$; it is aimed at combinatorialists.  

The main tool is the $(\C^*)^2$ action on $\Hilb_n(\C^2)$ induced by the $(\C^*)^2$ action on $\C^2$.  The torus $(\C^*)^2$ acts on the plane $\C^2$ in the obvious way: $(s,t)\cdot (\alpha,\beta)=(s\alpha,t\beta)$.  This induces an action on $R=\C[x,y]$ and on $\Hilb_n(\C^2)$.  

\begin{warning} \label{warning:action-sign}

 The elements of $R$ are functions on $\C^2$, and thus $(\C^*)^2$ acts with \emph{opposite} weights as might naively be expected, that is 
$$(s,t)\cdot x^ny^m=s^{-n}t^{-m}x^ny^m$$
\end{warning}




Many proofs of this are in the literature; for instance \cite{cheah} (IS IT IN THIS OR ONLY THE THESIS?) gives a largely combinatorial one, and \cite{nakajimabook}   Our proof is essentially the combinatorial one in \cite{cheah}, or , but made more so in that we use the boundary path of the partition instead of ``systems of arrows''.  






\subsection{Warm-up: Euler Characteristic of $\Hilb_n(\C^2)$} \label{sec:Teuler}

To illustrate how the torus action is used, we first compute the Euler characteristic of $\Hilb_n(\C^2)$.




\begin{proposition} \label{prop:hilbchi}
$$\chi(\Hilb_n(\C^2))=p(n).$$
Or, in terms of generating functions:
$$\chi(\Hilb_{\C^2})=\prod_{k\geq 0} \frac{1}{1-q^k}.$$
\end{proposition}

As a consistency check, this is what's obtained by setting $t=-1$ in Theorem \ref{thm:ES}.


Proposition \ref{prop:hilbchi} follows immediately from the following three lemmas.

\begin{lemma} \label{lem:toruschi}
Suppose that $X$ has a $T=(\C^*)^n$ action, with fixed point set $X^T$.  Then $\chi(X)=\chi(X^T)$.
\end{lemma}

\begin{lemma} \label{lem:Tmonomials}
The $T$-fixed points on $\Hilb_n(\C^2)$ are the monomial ideals.
\end{lemma}

\begin{lemma} \label{lem:monomialspartitions}
There is a bijection between monomial ideals $\II\subset R$ with $\dim_\C R/\II=n$ and partitions of $n$.
\end{lemma}

\begin{proof}[Sketch of Lemma \ref{lem:toruschi}]

  The pertinent fact about the Euler characteristic that makes it easy to compute is that it is additive over subvarieties: if $X\subset Y$ is a closed subvariety, then we have
  \[\chi(Y)=\chi(X)+\chi(Y\setminus X)\]
Note that this is not true of real submanifolds (take, e.g. $\{1\}\subset S^1$ as $X\subset Y$), and in some view is essentially because complex submanifolds have even codimension.


%Note that this is not true for real manifolds; take, the example of a point in $S^1$, both the point and its complement have euler characteristic 1, while $S^1$ has euler characteristic 0.  We briefly sketch the proof in case $X$ is a complex \emph{submanifold}.  If $U$ and $V$ are an open covering, then the Mayer-Vietoris sequence and the definition of $\chi$ give
%$$\chi(Y)=\chi(U)+\chi(V)-\chi(U\cap V)$$
%Taking $U=Y\setminus X$, and $V$ a tubular neighborhood of $X$, we see that $U\cap V$ we will be a homotopy equivalent to the normal sphere bundle of $X$ in $Y$.  Suppose that $X$ has complex codimension $k$; then $\chi(U\cap V)=\chi(X)\chi(S^{2k-1})=0$, since odd dimensional spheres have euler characteristic 0.

To apply this fact to the case when $T$ acts on $X$, we stratify $X$ according to its stabilizer subgroup.  For any subgroup $H\subset G$, we let $X^{(H)}\subset X$ denote the subset with stabilizer group $H$.  Then we have
\[\chi(X)=\sum_{H\subset G} \chi\left(X^{(H)}\right)\]
is a union of locally closed sets, to hence Lemma \ref{lem:toruschi} reduces to showing $\chi(X^{(H)})=0$ unless $H=G$.

The point now is that $X^{(H)}$ has a free action of $G/H$, and so is a $G/H$ bundle over the quotient $X^{(H)}/(G/H)$.  For any fiber bundle $Y$ with fiber $F$ and base space $B$ we have $\chi(Y)=\chi(B)\cdot \chi(F)$.  Since $G/H$ is itself a torus it has $\chi(G/H)=0$ unless $G=H$.  
\end{proof}


We now prove the $T$ fixed ideals are precisely the monomial ideas.
\begin{proof}[Proof of Lemma \ref{lem:Tmonomials}]
The torus $T$ acts on a monomial $x^\alpha y^\beta$ by scaling.  If $m\in \II$ is a monomial, and $g\in T$ then $g\cdot m=am\in \II$ for some $a\in \C$.  Hence an ideal generated by monomials will be $T$ fixed.  

In the other direction, suppose that $\II$ is $T$ fixed, and $f\in\II$.  We show all monomials in $f$ are in $\II$.  Suppose that $f$ is a linear combination of $k$ monomials $m_i$. Then there are $t_1,\cdots, t_k\in T$ so that $f_i=t_i \cdot f\in \II$ are linearly independent over $\C$. Inverting the matrix that expresses $f_i$ in terms of the monomials $m_i$, we see that $m_i$ is a linear combination of the $f_i$, and hence $m_i\in\II$.
\end{proof}



\begin{proof}[Proof of Lemma \ref{lem:monomialspartitions}]

The bijection between monomial ideals with $\dim R/\II=n$ and partitions of $n$ is best illustrated by example: 

\begin{center}
\begin{tikzpicture}
\draw[fill=red!20] (0,2) rectangle (1,3);
\draw[fill=red!20] (1,1) rectangle (2,2);
\draw[fill=red!20] (3,0) rectangle (4,1);


\draw[thin, gray] (0,0) grid (4.5,3.5);
\foreach \x in {0,...,4}
   \foreach \y in {0,...,3}
       \node at (\x+.5,\y+.5) {$x^\x y^\y$};
\draw[ultra thick, black] (0,0)--(0,2)--(1,2)--(1,1)--(3,1)--(3,0)--cycle;

\draw node at (7.5, 2) {$\begin{array}{cc} \II & \lambda \\ (x^3,xy,y^2) & (2,1,1) \end{array}$};
\end{tikzpicture}
\end{center}

The monomials \emph{not} in $\II$ form a basis for $R/\II$, and correspond to the cells of a partition $\II_\lambda$.  The generators of $\II_\lambda$ correspond to the monomials in the exterior corners of $\lambda$, the squares in red above.

\end{proof}

Thus, we have proven Ellingsrud-Str\o mme on the level of Euler characteristic.  To updgrade this poincare polynomial, we must use the Bia\l ynicki-Birula decomposition.



\subsection{Bia\l ynicki-Birula decomposition}

The Bia\l ynicki-Birula decomposition is essentially an algebraic version of Morse theory.

Morse theory describes the topology of a smooth manifold $X$ of dimension $n$ in terms of a generic smooth function $f:X\to\R$.  The critical points of such a function will be nondegenerate, that is, have non-degenerate Hessian matrix, and hence be locally of the form
\[f=a+x_1^2+x_2^2+\cdots +x_k^2-x_{k+1}^2-x_{k+2}^2-\cdots -x_n^2\]

The Morse flow is the downward gradient flow of $f$; the stable manifold $S_p$ of a critical point $p$ is the set of all points of $X$ that flow to $p$.  At each critical point $p$, the tangent space $T_pX$ splits into $T^+_pX$, the tangent directions flowing into $p$, and $T_p^-X$, the tangent directions flowing out of $p$.


The role of the Morse flow will be played by the flow $x\mapsto \varepsilon x$ as $\varepsilon\in\C^*$ tends toward zero.  We assume that the $\C^*$ action on $X$ is such that this limit point exists for all $x\in X$.  

Let $p$ be a fixed point of the $\C^*$ action.  Then $\C^*$ acts on $T_pX$, and so $T_pX$ is not just a vector space but a $\C^*$ representation.   and hence decomposes into a direct sum of irreducible representations.  Let $V_a$ denote the irreducible representation of $\C^*$ where $\varepsilon\in\C^*$ acts as $\varepsilon^a$.  


Let $T_p^+X$ (respectively $T_p^-X$) denote the subspace of $T_pX$ where the $\C^*$ action acts with a positive (respectively negative) exponent, and let $T_p^0(X)$ denote the subspace where $\C^*$ acts trivially.  Thus, if we consider the flow $X$ sending $\lambda$ toward 0, $T_p^+X$ are the directions that are flowing twoard $p$, and $T_p^-X$ are the directions flowing away from $p$.  

Note that if $p$ is an isolated fixed point, then $T_p^0(X)=0$.

$$T_pX=\bigoplus_{n\in\Z} V_n^{e_n}$$
then
$$T_p^+X=\bigoplus_{n>0} V_n^{e_n}$$

Let 
$$\mathcal{S}_p=\{x\in X|\lim_{\varepsilon\to 0} \varepsilon x=p\}$$
Then the Bia\l ynicki-Birula decomposition states that $X=\sqcup_{p}\mathcal{S}_p$; the point is that $\mathcal{S}_p$ is a subvariety isomorphic to $A_k$.


We saw in Section \ref{sec:Teuler} that if $X$ is a variety with a $\C^*$ action with $k$ isolated fixed points, then $\chi(X)=k$.  If we know the weights of the $\C^*$ action on the tangent spaces of the fixed points, the Bia\l ynicki-Birula decomposition leverages this extra information to give the Betti numbers of $X$, or further, the class of $X$ in the Grothendieck ring of varieties.

\subsection{Applying Bia\l ynicki-Birula to $\Hilb_n(\C^2)$}
To apply the Bia\l ynicki-Birula decomposition to $\Hilb_n(\C^2)$, we need to determine the torus weights on each tangent space $T_\lambda\Hilb_n(\C^2)$.

\begin{lemma}[Ellingsrud and Str\o mme \cite{ES}] \label{lem:torus-weights}
$$T_\lambda \Hilb_n(\C^2)=\sum_{\square\in\lambda} \left(x^{-\ell(\square)} y^{a(\square)+1}+x^{\ell(\square)+1}y^{-a(\square)}\right)$$
\end{lemma}

Before giving a proof, we show how Lemma \ref{lem:torus-weights} implies Ellingsrud and Str\o mme's result. 

\begin{proof}[Ellingsrud-Str\o mme]

  We have a described a $(\C^*)^2$ action on $\Hilb_n(\C^2)$.  To apply Bia\l ynicki-Birula, however, we need a $\C^*$ action with isolated fixed points.  The solution is to pass to a generic one dimensional subtorus -- we use the subtorus $\C^*$ that acts on $\C^2$ with weights $(\varepsilon,1)$, for $\varepsilon$ very small.

  The choice of subtorus weights should really be integer numbers, and by $(\varepsilon, 1)$ we really mean an action $(1, N)$ with $N>>n$. Since $a(\square)$ and $\ell(\square)$ are bounded by $n$, this means that we will always have $a(\square)<<N$, and so thinking about $\varepsilon$ as $1/N$, we have $a(\square)\varepsilon<<1$, and similarly with $\ell(\square)$.

In the flow $\lambda\to 0$ the positive directions are the stable directions.
  
Recall Warning \ref{warning:action-sign} that $\C^*$ will act with \emph{opposite} weights on $x$ and $y$, so that $$t\cdot x^\alpha y^\beta=t^{-\alpha\varepsilon-\beta}x^\alpha y^\beta.$$  

We now determine $\dim^+(\lambda)$ using this subtorus.  The term $x^{-\ell(\square)} y^{a(\square)+1}$ has weight $\varepsilon\ell(\square)-a(\square)-1$, which will never be positive, and hence will never contribute to $\dim^+(\lambda)$.

The term $x^{\ell(\square)+1}y^{-a(\square)}$ has weight $a(\square)-\varepsilon\ell(\square)-\varepsilon$, which will be positive if and only if $a(\square)>0$.  The squares with $a(\square)=0$ are exactly those at the top of eacg column of $\lambda$. Viewing the columns as the parts of $\lambda$, we see that the $i$th column contributes $\lambda_i-1$ to $\dim^+(\lambda)$.


\begin{center}
\begin{tikzpicture}[scale=.5]
\draw[fill=red!20] (0,0) rectangle (1,2);
\draw[fill=red!20] (0,0) rectangle (3,1);


\draw[thin, gray] (0,0) grid (1,3);
\draw[thin, gray] (1,0) grid (3,2);

\draw[thick] (0,0)--(0,3)--(1,3)--(1,2)--(3,2)--(3,1)--(4,1)--(4,0)--cycle;

\draw (2,-1) node {$\dim^+(\lambda)$ is number of shaded squares};
\end{tikzpicture}
\end{center}

\end{proof}

\begin{remark}
  While we picked the subtorus $(\varepsilon, 1)$, any generic $\C^*$ action (say, $(s,1)$, with $s\in\R\setminus\Q$) will have isolated fixed points, and hence can be used to compute the cohomology of $\Hilb_n(\C^2)$.  Using a different $\C^*$ action will give a different dimension statistic $\dim_s^-(\lambda)$:

\[ \dim_s^-(\lambda)=\left|\left\{\square\in\lambda : \frac{a(\square)}{\ell(\square)+1}<s<\frac{a(\square)+1}{\ell(\square)}\right\}\right| \]

However, since each of these statistics compute $P_t(\Hilb_n(\C^2)$, we see that
\begin{equation} \label{eq:lw} \sum_{\lambda\in\PP} q^{|\lambda|}t^{\dim_s^-(\lambda)}=\prod_{m=1}^\infty \frac{1}{1-tq^m}
  \end{equation}

This was apparently first observed by Haiman, and in \cite{LW} Loehr and Warrington gave a combinatorial proof of Equation \ref{eq:lw}, and furthermore extending the statistic to the non-isolated case -- that is, to rational $s$.  It may be interesting to have a geometric understanding of the bijections in \cite{LW}.
\end{remark}


\subsection{Determination of the tangent weights}

It remains to prove Lemma \ref{lem:torus-weights}, which will also be the main tool needed in the body of the text.  There are many proofs; the one we present is as bijective as possible, and as far as we know does not appear in the literature, although an expert told us: ``Everyone who studies this subject discovers this proof; nobody's written it down because it's awkward to draw the pictures''.


We begin with the following the description of the the tangent space to the Hilbert scheme:
\begin{lemma} \label{lem:tangent-hom}
$$T_{\II}\Hilb_n(\C^2)=\Hom_R(\II,R/\II)$$
\end{lemma}
Lemma \ref{lem:tangent-hom} actually holds in any dimension, for any smooth variety.

Before giving a proof, for intuition we give a plausibility argument in analogy with the Grassmannian.  If $V\subset W$ is a $k$ dimensional subspace, then the tangent space is the way to deform $V$ slightly inside of $W$, which we see is
\[T_V Gr_k(W)\cong \Hom(V, V^\perp) \cong \Hom(V, W/V).\]
Thus, we see that  \emph{$\C$-linear maps} from $\II$ to $R/II$ give deformations of $\II$ as a \emph{vector subspace} of $R$.  It is reasonable to guess that if we want the deformation to remain an ideal, we should require it to be $R$-linear instead of just $\C$-linear.

\begin{proof}
More formally, it is a general fact that first order deformations of objects are given by $\Ext^1(\mathcal{F},\mathcal{F})$, and obstructions to these deformations are given by $\Ext^2(\mathcal{F},\mathcal{F})$. 

The long exact sequence of $R$-modules obtained by applying $\Hom_R(\II, -)$ to the short exact sequence
$$0\to \II\to R\to R/\II\to 0$$
gives 
$$0\to\Hom_R(\II,\II)\to \Hom_R(\II, R)\to\Hom_R(\II,R/\II)\to\Ext_R^1(\II, \II)\to\Ext_R^1(\II,R)$$
Since $\II$ is an ideal that eventually contains $x^n, y^m$, one can check that  $\Hom_R(\II,\II)\cong\Hom_R(\II, R)$;  and $\Ext^1(\II,R)=0$, and so indeed we have $\Hom_R(\II,R/\II)\cong\Ext_R^1(\II, \II)$.
\end{proof}


SYSTEMS OF ARROWS

We can now complete the proof of the tangent weight statement.

As the isomorphism was, it will respect the $(C^*)^2$ action.  Let $T_\lambda^{a,b}$ denote the $(a,b)$-isotypical component of $T_{\II_\lambda}\Hilb_n(\C^2)$.  Then vectors $v\in T_\lambda^{a,b}$ correspond to module homomorphisms $f\in\Hom_R(\II_\lambda, R/\II_\lambda)$ of weight $(a,b)$, i.e., with
$$f(x^{\alpha}y^\beta)=c_{\alpha,\beta} x^{\alpha-a}y^{\beta-b}$$
for some constant $c_{\alpha, \beta}$ (that can be different for different monomials).  


Let $P_\lambda$ be the boundary path of $\lambda$; for $(a,b)$ in $\Z^2$, let $P_\lambda(a,b)$ denote the boundary path of $\lambda$ shifted to the right by $a$ and up by $b$.

\begin{lemma} Let $B_\lambda(a,b)$ be the set of bounded regions above $P_\lambda$ and below $P_\lambda(a,b)$.  For $U\in B_\lambda(a,b)$, let $f_U$ denote the map that multiplies monomials in $U$ by $x^{-a}y^{-b}$ and sends monomials not in $B_\lambda(a,b)$ to $0$.  Then the $f_U$ form a basis for $T_\lambda^{a,b}$.
\end{lemma}

\begin{proof}
To be a map of $R$-modules the map must commute with multipication by $x$ and $y$.  Thus, if $x^{\alpha-a+1}y^{\beta-b}\notin \II$, we must have $c_{\alpha+1,\beta}=c_{\alpha,\beta}$, and similarly with $y$.  Thus, to be a map of $R$-modules, we see that monomials in the same component must get multiplied by the same constant.

If a region is not bounded, then one of the generators of $\II$ contained in that region would be mapped to a monomial with negative exponents, which it can't do; therefore it must be multiplied by 0.
\begin{center}
\begin{tikzpicture}[scale=.5]

\fill[red!20!white] (0,5)--(0,3)--(1,3)--(1,5)--cycle;
\fill[red!20!white] (1,3)--(2,3)--(2,2)--(1,2)--cycle;
\fill[red!20!white] (3,2)--(4,2)--(4,1)--(3,1)--cycle;
\fill[red!20!white] (4,1)--(5,1)--(5,0)--(4,0)--cycle;
\draw[thin, gray] (0,0) grid (5,5);
\draw (.5,3.5) node{$y^3$};
\draw (1.5,2.5) node{$xy^2$};
\draw (3.5,1.5) node{$x^3y$};
\draw (4.5,.5) node{$x^4$};

\begin{scope}[xshift=-.1cm, yshift=-.1cm]
\draw[thick, red,  triangle 60 reversed-triangle 60] (0,5)--(0,3)--(1,3)--(1,2)--(3,2)--(3,1)--(4,1)--(4,0)--(5,0);
\end{scope}
\begin{scope}[xshift=1.1cm, yshift=.1cm]
\draw[thick, blue, dashed, triangle 60 reversed-triangle 60]  (0,5)--(0,3)--(1,3)--(1,2)--(3,2)--(3,1)--(4,1)--(4,0)--(5,0);
\end{scope}

\begin{scope}[xshift=8cm]
\fill[red!20!white] (0,5)--(2,5)--(2,3)--(3,3)--(3,2)--(1,2)--(1,3)--(0,3)--cycle;
\fill[red!20!white] (3,2)--(5,2)--(5,1)--(6,1)--(6,0)--(4,0)--(4,1)--(3,1)--cycle;

\draw[thin, gray] (0,0) grid (5,5);
\draw (.5,3.5) node{$y^3$};
\draw (1.5,2.5) node{$xy^2$};
\draw (3.5,1.5) node{$x^3y$};
\draw (4.5,.5) node{$x^4$};
\begin{scope}[xshift=-.1cm, yshift=-.1cm]
\draw[thick, red,  triangle 60 reversed-triangle 60] (0,5)--(0,3)--(1,3)--(1,2)--(3,2)--(3,1)--(4,1)--(4,0)--(5,0);
\end{scope}
\begin{scope}[xshift=2.1cm, yshift=.1cm]
\draw[thick, blue, dashed, triangle 60 reversed-triangle 60]  (0,5)--(0,3)--(1,3)--(1,2)--(3,2)--(3,1)--(4,1)--(4,0)--(5,0);
\end{scope}
\end{scope}



\end{tikzpicture}
\end{center}
The picture on the left shows $(1,0)$ is three dimensional.  The region containing $y^3$ is unbounded; indeed, $y^3$ would have to map to $x^{-1}y^3\notin R/\II$.  

The picture on the right shows $(2,0)$ is one dimensional; $y^3$ ad $xy^2$ would each map to things not in $R/\II$ because they are in an unbounded region.  We also have that $x^3y$ and $x^4$ must get multiplied by the same constant, since $xf(x^3y)=x^2y=yf(x^4)$ and $x^2y\in R\II$.
\end{proof}


We now prove Lemma \ref{lem:torus-weights}.  First, observe that $T_\lambda^{(a,b)}$ is empty if $(a,b)$ are both negative or both non-negative.  If both are non-negative, then there are no cells at all below $P_\lambda(a,b)$ and above $P_\lambda$.  If both are positive, then the cells below $P_\lambda(a,b)$ and $P_\lambda$ form a single unbounded region.  
\begin{center}
\begin{tikzpicture}[scale=.5]
\fill[blue!20!white] (0,5)--(1,5)--(1,4)--(2,4)--(2,3)--(4,3)--(4,2)--(5,2)--(5,0)--(4,0)--(4,1)--(3,1)--(3,2)--(1,2)--(1,3)--(0,3)--cycle;
\draw (.5,3.5) node{$y^3$};
\draw (1.5,2.5) node{$xy^2$};
\draw (3.5,1.5) node{$x^3y$};
\draw (4.5,.5) node{$x^4$};
\draw[thin, gray] (0,0) grid (5,5);
\begin{scope}[xshift=-.1cm, yshift=-.1cm]
\draw[thick, red,  triangle 60 reversed-triangle 60] (0,5)--(0,3)--(1,3)--(1,2)--(3,2)--(3,1)--(4,1)--(4,0)--(5,0);
\end{scope}
\begin{scope}[xshift=1.1cm, yshift=1.1cm]
\draw[thick, blue, dashed, triangle 60 reversed-triangle 60]  (0,5)--(0,3)--(1,3)--(1,2)--(3,2)--(3,1)--(4,1)--(4,0)--(5,0);
\end{scope}

\begin{scope}[xshift=9cm]
\draw[thin, gray] (0,0) grid (5,5);
\draw (.5,3.5) node{$y^3$};
\draw (1.5,2.5) node{$xy^2$};
\draw (3.5,1.5) node{$x^3y$};
\draw (4.5,.5) node{$x^4$};
\begin{scope}[xshift=-.1cm, yshift=-.1cm]
\draw[thick, red,  triangle 60 reversed-triangle 60] (0,5)--(0,3)--(1,3)--(1,2)--(3,2)--(3,1)--(4,1)--(4,0)--(5,0);
\end{scope}
\begin{scope}[xshift=-.9cm, yshift=.1cm]
\draw[thick, blue, dashed, triangle 60 reversed-triangle 60]  (0,5)--(0,3)--(1,3)--(1,2)--(3,2)--(3,1)--(4,1)--(4,0)--(5,0);
\end{scope}
\end{scope}

\end{tikzpicture}
\end{center}
The left shows that for $(1,1)$ are both positive, then there is a single unbounded region.  The right shows that for $(-1,0)$ there are no cells above $P_\lambda$ and below $P_\lambda(-1,0)$.
Thus, exactly one of $a,b$ is negative; assume it is $a$.  We must show that the elements of $B_\lambda(a,b)$ are in bijection with the cells $\square\in\lambda$ having  $a=\ell(\square)$ and $b=-a(\square)-1$ (recall the sign from Warning \ref{warning:action-sign}).

Consider a region of cells above $P_\lambda$ and below $P_\lambda(-a,-b)$.  There are two possible directions this region could be unbounded -- along the positive $y$ or $x$-axes.  Since $b$ is positive, there are no squares far along the $x$-axis both below $P_\lambda(-a,-b)$  and above $P_\lambda$, and so the region is automatically bounded in that direction.  

We must therefore guarantee the region is bounded at the bottom right.  This will happen when a South step of $P_\lambda$ starts at the same point as East step of $P_\lambda(-a,-b)$.  




Bounded cells well thus be in bijection with pairs of such cells.  Translating the East step back $(a,b)$ to its original place on the boundary strip, we find these two steps give an inversion in the $P_\lambda$ and hence a cell of $\lambda$.  The arm length of this cell will be $b+1$, and the leg will have size $a$. 

An analogous argument shows that tangent directions with $b$ negative correspond to the other terms in the sum.






\bibliographystyle{plain}
\bibliography{GHilb}

\end{document}
