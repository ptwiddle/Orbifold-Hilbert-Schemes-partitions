\documentclass{amsart}
\DeclareMathOperator{\Hilb}{Hilb}
\newcommand{\Z}{\mathbb{Z}}
\newcommand{\C}{\mathbb{C}}

\title{Bit of number theory from Zurich}

\begin{document}

Let $a,r$ be a pair of positive integers, so that $(a,r)=1$ and $(a+1,r)=1$, and let $G$ be $\Z_r$ acting with weights $(1/r, a/r)$.  The first condition is equivalent to nonidentiy elements of $G$ only fixing the origin, the second condition is equivalent to $G$ having trivial intersection with $SL_2$. 

In this case, the generating function for the topology of Hilbert schemes of points appears to have a product formula that is just the Euler formula with some $t$'s inserted in a particular nice format.  If $P_t$ denotes the poincare polynomial, then it appears

$$\sum_{n} P_t(\Hilb_n([\C^2/G]))q^n=\prod_{m=1}^r \frac{1}{(q^mt^{\varepsilon{m}};q^at^2)_\infty}$$






\end{document}
