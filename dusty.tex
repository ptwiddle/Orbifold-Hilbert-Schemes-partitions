\documentclass{amsart}
\usepackage{tikz}


\DeclareMathOperator{\Hilb}{Hilb}
\DeclareMathOperator{\dusty}{pdim}
\newcommand{\Z}{\mathbb{Z}}
\newcommand{\C}{\mathbb{C}}


\title{Bit of number theory from Zurich}

\begin{document}

My proposed formula for the product formula was in terms of the Chen-Ruan cohomology; Dusty instead changed arm to coleg to obtain a combinatorial formula.  Our goal here is to show these viewpoints are consistent in the case when the intersection with $SL_2$ is trivial.

$$\dusty_{k/r}(\lambda)=\#\left\{\square\in\lambda\Big| \ell(\square)-kco\ell(\square)=-1\mod r\right\}$$
 
Linear combinations of leg and co-leg depend only on the box's position within its row; hence every row of length $L$ will contribute the same amount to $\dusty_{k/r}(\lambda)$, and it is immediate that the joint distribution of $|\lambda|$ and $\dusty_{k/r}(\lambda)$ will be a $t$-deformation of the Euler product.  It remains to check that this is the $t$-deformation predicted by our conjecture.

Consider $\ell(\square)+1-kco\ell(\square)$; for the first square in a row of length $L$, this quantity will be $L$, and each square to the right we move the quantity will drop by $k+1$, as illustrated in the following dia

\begin{tikzpicture}[scale=.5]

\draw (0,0) grid (1,1);
\draw (.5,.5) node{$1$};


\begin{scope}[yshift=-1.5cm]
\draw (0,0) grid (2,1);
\draw (.5,.5) node{$2$};
\draw (1.5,.5) node{$0$};

\begin{scope}[yshift=-1.5cm]
\draw (0,0) grid (3,1);
\draw (.5,.5) node{$3$};
\draw (1.5,.5) node{$1$};
\draw (2.5,.5) node{$-1$};


\begin{scope}[yshift=-1.5cm]
\draw (0,0) grid (4,1);
\draw (.5,.5) node{$4$};
\draw (1.5,.5) node{$2$};
\draw (2.5,.5) node{$0$};
\draw (3.5,.5) node{$-2$};



\begin{scope}[yshift=-1.5cm]
\draw (0,0) grid (5,1);
\draw (.5,.5) node{$5$};
\draw (1.5,.5) node{$3$};
\draw (2.5,.5) node{$1$};
\draw (3.5,.5) node{$-1$};
\draw (4.5,.5) node{$-3$};
\end{scope}
\end{scope}
\end{scope}
\end{scope}



\begin{scope}[xshift=8cm]

\draw (0,0) grid (1,1);
\draw (.5,.5) node{$1$};


\begin{scope}[yshift=-1.5cm]
\draw (0,0) grid (2,1);
\draw (.5,.5) node{$2$};
\draw (1.5,.5) node{$-1$};

\begin{scope}[yshift=-1.5cm]
\draw (0,0) grid (3,1);
\draw (.5,.5) node{$3$};
\draw (1.5,.5) node{$0$};
\draw (2.5,.5) node{$-3$};


\begin{scope}[yshift=-1.5cm]
\draw (0,0) grid (4,1);
\draw (.5,.5) node{$4$};
\draw (1.5,.5) node{$1$};
\draw (2.5,.5) node{$-2$};
\draw (3.5,.5) node{$-5$};



\begin{scope}[yshift=-1.5cm]
\draw (0,0) grid (5,1);
\draw (.5,.5) node{$5$};
\draw (1.5,.5) node{$2$};
\draw (2.5,.5) node{$-1$};
\draw (3.5,.5) node{$-4$};
\draw (4.5,.5) node{$-7$};
\end{scope}
\end{scope}
\end{scope}
\end{scope}
\end{scope}
\end{tikzpicture}
When the intersection of $G$ and $SL_2$ is trivial, i.e. when $k+1$ is relatively prime to $r$, we see that adding $r$ boxes to a row will add one box of each residue class mod $r$, and thus proving the conjectures are equivalent reduces to checking the same number of boxes for rows of length less than $r$.  

For a row of length $L<r$ to contribute to the homological degree, we need 
\begin{align*}
0 &\in \{L, L-(k+1),\dots, L-(L-1)(k+1) \} \\
-L &\in \{0, -(k+1),\dots,-(L-1)(k+1) \} \\
\frac{-L}{k+1} &\in \{0, -1, \dots, -(L-1) \} \\
\end{align*}
Equivalently, if we choose the representative of $[-L/(k+1)]\in \mathbb{Z}/r\mathbb{Z}$ in $\{0,\dots, r-1\}$, then we need $-L/(k+1)>-L$.

From the Chen-Ruan point of view, for $\lambda \in \mathbb{Z}/r\mathbb{Z}$ a part with length $M<r, M\cong-(k+1)\lambda\mod r$ will contribute if we have to carry when we add $\lambda $ and $k\lambda$, that is
\begin{align*}
k \lambda &\in -1,\dots, -\lambda \\
(k+1)\lambda &\in 0,\dots, \lambda-1 \\
M\in 0,-1,\dots, -\lambda+1 \\
M \in 0, -1,\dots, M/(k+1)+1
\end{align*}
That is, we need $M>M/(k+1)$.









\end{document}
